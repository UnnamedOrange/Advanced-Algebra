% Licensed under the Creative Commons Attribution Share Alike 4.0 International.
% See the LICENCE file in the repository root for full licence text.

\chapter{矩阵的相抵与相似}

本章研究两个重要的等价关系。

\section{矩阵的相抵}

\begin{definition}{相抵}
	对于数域 $\mathbb K$ 上的 $s \times n$ 矩阵 $A$ 和 $B$,如果从 $A$ 经过一系列初等行变换和初等列变换能变成矩阵 $B$,那么称 $A$ 与 $B$ 是\emph{相抵}的,记作 $A \stackrel{\text{相抵}}{\sim} B$。
\end{definition}

\begin{theorem}
	相抵是 $M_{s \times n}(\mathbb K)$ 上的一个等价关系。
\end{theorem}

\begin{proof}
	\begin{enumerate}
		\item 显然,$A \stackrel{\text{相抵}}{\sim} A$;
		\item 显然,若 $A \stackrel{\text{相抵}}{\sim} B$,则 $B \stackrel{\text{相抵}}{\sim} A$;
		\item 显然,若 $A \stackrel{\text{相抵}}{\sim} B$,$B \stackrel{\text{相抵}}{\sim} C$,则 $A \stackrel{\text{相抵}}{\sim} C$。
	\end{enumerate}
\end{proof}

注意到,若 $\operatorname{rank}(A) = r$,则经过一系列的初等行变换,可得到一简化阶梯形矩阵,再经过一系列初等列变换,可得到矩阵 $\begin{bmatrix} I_r & 0 \\ 0 & 0 \end{bmatrix}$。于是我们可以得到两矩阵相抵的又一个定义。

\begin{theorem}
	数域 $\mathbb K$ 上 $s \times n$ 矩阵 $A$ 与 $B$ 相抵当且仅当它们的秩相等。
\end{theorem}

由初等矩阵,可以把以上初等行变换和初等列变换记作:
$$
P_s \cdots P_1 A Q_1 \cdots Q_t = PAQ = \begin{bmatrix} I_r & 0 \\ 0 & 0 \end{bmatrix}
$$

其中:
$$
P = P_s \cdots P_1
$$$$
Q = Q_1 \cdots Q_t
$$

由于 $P_s, \ldots, P_1, Q_1, \ldots, Q_t$ 都是初等矩阵,因此 $P$ 和 $Q$ 是可逆矩阵。于是我们可以得出以下结论。

\begin{theorem}
	对于任意的非零矩阵 $A$,若 $\operatorname{rank}(A) = r$,则总存在两可逆矩阵 $P$ 和 $Q$,使得 $PAQ = \begin{bmatrix} I_r & 0 \\ 0 & 0 \end{bmatrix}$。
\end{theorem}

由于任何一个秩为 $r$ 的矩阵都能经过初等行变换和初等列变换变成 $\begin{bmatrix} I_r & 0 \\ 0 & 0 \end{bmatrix}$(零矩阵保持为零矩阵),因此引入相抵标准形的概念。

\begin{definition}{相抵标准形}
	设数域 $\mathbb K$ 上的 $s \times n$ 矩阵 $A$ 的秩为 $r$。如果 $r > 0$,那么 $A$ 相抵于矩阵 $\begin{bmatrix} I_r & 0 \\ 0 & 0 \end{bmatrix}$,称它为 $A$ 的\emph{相抵标准形}。规定零矩阵的相抵标准形就是它自己。
\end{definition}