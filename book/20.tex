% Licensed under the Creative Commons Attribution Share Alike 4.0 International.
% See the LICENCE file in the repository root for full licence text.

\section{$\R^n$ 的子空间的正交补}

给定 $\R^n$ 的一个子空间,我们想要知道,在 $\R^n$ 中有哪些向量与该子空间的所有向量都正交。为此,我们提出以下概念。

\begin{definition}{正交}
	设 $U$ 是欧几里得空间 $\R^n$ 的一个子空间,如果向量 $\vec \alpha$ 与 $U$ 中的每一个向量都正交,那么称 $\vec \alpha$ 与 $U$ \emph{正交},记作 $\vec \alpha \perp U$。
\end{definition}

\begin{definition}{正交补}
	定义 $\R^n$ 的子空间 $U$ 的\emph{正交补} $U^\perp$ 为:
	$$
	U^\perp \triangleq \{ \vec \alpha \in \R^n \colon \vec \alpha \perp U \}
	$$
\end{definition}

可见,正交补即是上面的问题中要研究的对象。正交补满足什么性质?以下定理描述了正交补最基本的特征。

\begin{theorem}
	设 $U$ 是 $\R^n$ 的一个子空间,则 $U^\perp$ 也是 $\R^n$ 的一个子空间。
\end{theorem}

\begin{proof}
	由于 $\vec 0 \perp U$,因此 $\vec 0 \in U^\perp$,可知 $U^\perp$ 非空。任取 $\vec \alpha, \vec \beta \in U^\perp$,则 $\forall \vec \gamma \in U$,有:
	$$
	(\vec \alpha + \vec \beta, \vec \gamma) = (\vec \alpha, \vec \gamma) + (\vec \beta, \vec \gamma) = 0
	$$$$
	(k \vec \alpha, \vec \gamma) = k(\vec \alpha, \vec \gamma) = 0
	$$

	说明 $\vec \alpha + \vec \beta \in U^\perp$ 且 $k \vec \alpha \in U^\perp$,即 $U^\perp$ 是 $\R^n$ 的一个子空间。
\end{proof}

\begin{theorem}
	设 $U$ 是 $\R^n$ 的一个子空间,则:
	$$
	\dim U + \dim U^\perp = \dim \R^n = n
	$$
\end{theorem}

\begin{proof}
	设 $U = \langle \vec \alpha_1, \ldots, \vec \alpha_s \rangle$,其中 $\vec \alpha_1, \ldots, \vec \alpha_s$ 线性无关(规定这些向量都是列向量)。对于 $\R^n$ 中的向量 $\vec \alpha$,易证:
	$$
	\vec \alpha \in U^\perp \Longleftrightarrow (\vec \alpha, \vec \alpha_i) = 0 \pod{i = 1, \ldots, s}
	$$

	于是 $\vec \alpha_i^T \vec \alpha = 0 \pod{i = 1, \ldots, s}$。将 $s$ 个这样的等式合在一起,可以得到:
	$$
	\begin{bmatrix} \vec \alpha_1 & \cdots & \vec \alpha_s \end{bmatrix}^T \vec \alpha = \vec 0
	$$

	即可知 $\vec \alpha$ 是齐次线性方程组 $\begin{bmatrix} \vec \alpha_1 & \cdots & \vec \alpha_s \end{bmatrix}^T \vec x = \vec 0$ 的一组解,所以有:
	$$
	U^\perp = \operatorname{Ker} \begin{bmatrix} \vec \alpha_1 & \cdots & \vec \alpha_s \end{bmatrix}^T
	$$

	由重要维数公式\footnote{$\dim \operatorname{Ker} \phi = n - \operatorname{rank}(A)$,其中 $A$ 是线性映射 $\phi$ 对应的矩阵,$n$ 是 $A$ 的列数。}:
	$$
	\begin{aligned}
		\dim U^\perp &= n - \operatorname{rank} \begin{bmatrix} \vec \alpha_1 & \cdots & \vec \alpha_s \end{bmatrix}^T
		\\&=
		n - s
	\end{aligned}
	$$
\end{proof}