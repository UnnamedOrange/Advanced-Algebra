% Licensed under the Creative Commons Attribution Share Alike 4.0 International.
% See the LICENCE file in the repository root for full licence text.

\section{向量空间 $\mathbb K^n$ 及其子空间的基与维数}

$\mathbb K^n$ 中,设:
$$
\begin{aligned}
	\vec e_1 &= (1, 0, \ldots, 0)
	\\
	\vec e_2 &= (0, 1, \ldots, 0)
	\\
	\vdots
	\\
	\vec e_n &= (0, 0, \ldots, 1)
\end{aligned}
$$

显然 $\vec e_1, \vec e_2, \ldots, \vec e_n$ 是线性无关的。

又设 $\mathbb K^n$ 中的任一向量 $\vec \alpha$ 为:
$$
\vec \alpha = (a_1, a_2, \ldots, a_n)
$$

注意到,$\vec \alpha$ 可以使用 $\vec e_1, \vec e_2, \ldots, \vec e_n$ 表示为:
$$
\vec \alpha = \sum\limits_{i = 1}^n a_i \vec e_i
$$

由此思考,如果给定 $\mathbb K^n$ 中的一组向量 $\vec \beta_1, \vec \beta_2, \ldots, \vec \beta_m$,能否将 $\mathbb K^n$ 或其子空间中的任一向量表示为 $\sum\limits_{i = 1}^m b_i \vec \beta_i$ 的形式,以使用 $b_1, b_2, \ldots, b_m$ 这 $m$ 个数表示该向量?为了研究这一问题,我们首先引出以下概念。

\begin{definition}{基}
	设 $U$ 是 $\mathbb K^n$ 的一个子空间,如果 $\vec \alpha_1, \vec \alpha_2, \ldots, \vec \alpha_r \in U$,并且满足下述两个条件:
	\begin{enumerate}
		\item $\vec \alpha_1, \vec \alpha_2, \ldots, \vec \alpha_r$ 线性无关。
		\item $U$ 中的每一个向量都可以由 $\vec \alpha_1, \vec \alpha_2, \ldots, \vec \alpha_r$ 线性表出。
	\end{enumerate}

	那么称向量组 $\vec \alpha_1, \vec \alpha_2, \ldots, \vec \alpha_r$ 是 $U$ 的一个\emph{基}。
\end{definition}

由于子空间 $U$ 的基线性无关,因此用 $U$ 的任意一个基线性表出 $U$ 中向量的方式唯一。

前文中已多次出现 $\vec e_1, \vec e_2, \ldots, \vec e_n$,显然该向量组是 $\mathbb K^n$ 的一个基。由它的特殊性,我们给出以下定义。

\begin{definition}{标准基}
	在 $\mathbb K^n$ 中,称向量组 $\vec e_1, \vec e_2, \ldots, \vec e_n$ 为 $\mathbb K^n$ 的\emph{标准基}。
\end{definition}

\subsection{基的性质}

\begin{theorem}
	$\mathbb K^n$ 的任一非零子空间 $U$ 都有一个基。
\end{theorem}

\begin{proof}[构造法]
	取 $U$ 中的一个非零向量 $\vec \alpha_1$,向量组 $\vec \alpha_1$ 是线性无关的。若 $\langle \vec \alpha_1 \rangle \ne U$,则存在 $\vec \alpha \in U$,使得 $\vec \alpha_2 \not \in \langle \vec \alpha_1 \rangle$。由于 $\vec \alpha_2$ 不能由 $\vec \alpha_1$ 线性表出,因此 $\vec \alpha_1, \vec \alpha_2$ 线性无关。

	照此循环加入向量,直到 $\langle \vec \alpha_1, \vec \alpha_2, \ldots, \vec \alpha_s \angle = U$。由于 $\mathbb K^n$ 中任一线性无关的向量组中所含向量的个数不超过 $n$,因此一定存在一个有限的向量组使得它们生成的子空间为 $U$。证毕。
\end{proof}

以上证明过程表明,子空间 $U$ 的任意一个线性无关的向量组都可以扩充成 $U$ 的一个基。

\begin{theorem}
	$\mathbb K^n$ 的任一非零子空间 $U$ 的任意两个基所含向量的个数相等。
\end{theorem}

这是因为等价的线性无关的向量组含有相同个数的向量。根据这一性质,我们可以定义子空间的维数。

\begin{definition}{维数}
	$\mathbb K^n$ 的任一非零子空间 $U$ 的任意一个基所含向量的个数称为 $U$ 的\emph{维数},记作 $\dim_K U$ 或 $\dim U$。特别地,零子空间也存在维数的概念,规定为 $0$。
\end{definition}

基于本节已有的定义与性质,我们已经可以针对本节一开始提出的问题给出以下概念。

\begin{definition}{坐标}
	设 $\vec \alpha_1, \ldots, \vec \alpha_r$ 是 $\mathbb K^n$ 的一个子空间 $U$ 的一个基,则由基的定义,$U$ 的每一个向量 $\vec \alpha$ 都可以由 $\vec \alpha_1, \ldots, \vec \alpha_r$ 唯一地线性表出。设:
	$$
	\vec \alpha = a_1 \vec \alpha_1 + a_2 \vec \alpha_2 + \cdots + a_r \vec \alpha_r
	$$
	则把有序数组 $(a_1, \ldots, a_r)$ 称为 $\vec \alpha$ 在基 $\vec \alpha_1, \ldots, \vec \alpha_r$ 下的\emph{坐标}。
\end{definition}

\subsection{维数的性质}

下面给出一系列有关子空间维数的性质。这些性质均给出了严格证明,但都是直观的,可以按感性理解直接使用。

\begin{theorem}
	设 $\dim U = r$,则 $U$ 中任意 $r + 1$ 个向量都线性相关。
\end{theorem}

\begin{proof}
	取 $U$ 的一个基 $\vec \alpha_1, \vec \alpha_2, \ldots, \vec \alpha_r$,则 $U$ 中任意的 $r + 1$ 个向量 $\vec \beta_1, \vec \beta_2, \ldots, \vec \beta_{r + 1}$ 都可由 $\vec \alpha_1, \vec \alpha_2, \ldots, \vec \alpha_r$ 线性表出,即可知这 $r + 1$ 个向量组成的向量组线性相关\footnote{若向量组 $A$ 可由向量组 $B$ 线性表出,且 $|A| > |B|$,则向量组 $A$ 线性相关。}。
\end{proof}

\begin{theorem}
	设 $\dim U = r$,则 $U$ 中任意 $r$ 个线性无关的向量组成的向量组都是 $U$ 的一个基。
\end{theorem}

\begin{proof}
	任取 $U$ 中一线性无关的向量组 $\vec \alpha_1, \vec \alpha_2, \ldots, \vec \alpha_r$,再任取 $\vec \beta \in U$,则向量组 $\vec \alpha_1, \vec \alpha_2, \ldots, \vec \alpha_r, \vec \beta$ 线性相关,因此 $\vec \beta$ 可由 $\vec \alpha_1, \vec \alpha_2, \ldots, \vec \alpha_r$ 线性表出。综上, $\vec \alpha_1, \vec \alpha_2, \ldots, \vec \alpha_r$ 是 $U$ 的一个基。
\end{proof}

\begin{theorem}
	设 $\dim U = r$,$\vec \alpha_1, \vec \alpha_2, \ldots, \vec \alpha_r \in U$。如果 $U$ 中的每一个向量都可以由 $\vec \alpha_1, \vec \alpha_2, \ldots, \vec \alpha_r$ 线性表出,那么 $\vec \alpha_1, \vec \alpha_2, \ldots, \vec \alpha_r$ 是 $U$ 的一个基。
\end{theorem}

\begin{proof}
	取 $U$ 的一个基 $\vec \delta_1, \vec \delta_2, \ldots, \vec \delta_r$,可知它可被 $\vec \alpha_1, \vec \alpha_2, \ldots, \vec \alpha_r$ 线性表出,故两者等价,从而 $\vec \alpha_1, \vec \alpha_2, \ldots, \vec \alpha_r$ 也是 $U$ 的一个基。
\end{proof}

\begin{theorem}
	设 $U$ 和 $W$ 都是 $\mathbb K^n$ 的非零子空间,如果 $U \subseteq W$,那么 $\dim U \le \dim W$。
\end{theorem}

\begin{proof}
	在 $U$ 和 $W$ 中各取一个基 $A$ 和 $B$。因为 $U \subseteq W$,所以 $A$ 中的向量可被 $B$ 线性表出,从而 $|A| \le |B|$,进而 $\dim U \le \dim W$。
\end{proof}

\begin{theorem}
	设 $U$ 和 $W$ 是 $\mathbb K^n$ 的两个非零子空间,且 $U \subseteq W$。如果 $\dim U = \dim W$,那么 $U = W$。
\end{theorem}

\begin{proof}
	在 $U$ 中取一个基 $\vec \alpha_1, \vec \alpha_2, \ldots, \vec \alpha_r$。由于 $U \subseteq W$,因此 $\vec \alpha_1, \vec \alpha_2, \ldots, \vec \alpha_r \in W$。由于 $\dim W = \dim U = r$,因此 $\vec \alpha_1, \vec \alpha_2, \ldots, \vec \alpha_r$ 是 $W$ 的一个基,从而 $W$ 中任一向量 $\vec \beta$ 可以由 $\vec \alpha_1, \vec \alpha_2, \ldots, \vec \alpha_r$ 线性表出,于是 $\vec \beta \in U$,因此 $W \subseteq U$,从而 $U = W$。
\end{proof}

\begin{theorem}
	向量组 $\vec \alpha_1, \vec \alpha_2, \ldots, \vec \alpha_s$ 的一个极大线性无关组是这个向量组生成的子空间 $\langle \vec \alpha_1, \vec \alpha_2, \ldots, \vec \alpha_s \rangle$ 的一个基,从而:
	$$
	\dim \langle \vec \alpha_1, \vec \alpha_2, \ldots, \vec \alpha_s \rangle = \operatorname{rank} \{\vec \alpha_1, \vec \alpha_2, \ldots, \vec \alpha_s\}
	$$
\end{theorem}

\begin{proof}
	由于向量组 $\vec \alpha_1, \vec \alpha_2, \ldots, \vec \alpha_s$ 生成的子空间 $W = \langle \alpha_1, \vec \alpha_2, \ldots, \vec \alpha_s \rangle$ 中的每一个向量都可由 $\vec \alpha_1, \vec \alpha_2, \ldots, \vec \alpha_s$ 的一个极大线性无关组线性表出,因此 $\vec \alpha_1, \vec \alpha_2, \ldots, \vec \alpha_s$ 的一个极大线性无关组是 $W$ 的一个基,从而 $\dim W = r$。
\end{proof}

值得注意的是,维数和秩是两个完全不同的概念。维数是对子空间而言的,秩是对向量组而言的。子空间有无穷多个向量,而向量组只有有限多个向量。为了便于描述,我们补充以下定义。

\begin{definition}{列空间,行空间}
	数域 $\mathbb K$ 上 $s \times n$ 的矩阵 $A$ 的列向量组 $\vec \alpha_1, \vec \alpha_2, \ldots, \vec \alpha_n$ 生成的子空间称为 $A$ 的\emph{列空间}。$A$ 的行向量组 $\vec \gamma_1, \vec \gamma_2, \ldots, \vec \gamma_s$ 生成的子空间称为 $A$ 的\emph{行空间}。$A$ 的列(行)空间的维数等于 $A$ 的列(行)向量组的秩,即等于 $A$ 的秩。
\end{definition}