% Licensed under the Creative Commons Attribution Share Alike 4.0 International.
% See the LICENCE file in the repository root for full licence text.

\section{整除关系与带余除法}

从一元多项式环 $\mathbb K[x]$ 上的乘法出发,我们引出整除的概念,然后研究没有整除关系的两个多项式之间的带余除法。

\subsection{整除关系与带余除法的概念}

\begin{definition}{整除}
	设 $f(x), g(x) \in \mathbb K[x]$,如果存在 $h(x) \in \mathbb K[x]$,使得 $f(x) = h(x) g(x)$,那么称 $g(x)$ \emph{整除} $f(x)$,记作 $g(x) \mid f(x)$;否则,称 $g(x)$ 不能整除 $f(x)$,记作 $g(x) \nmid f(x)$。
\end{definition}

\begin{proposition}[带余除法]
	设 $f(x), g(x) \in \mathbb K[x]$,且 $g(x) \ne 0$,则在 $\mathbb K[x]$ 中存在唯一的一对多项式 $h(x), r(x)$,使得:
	$$
	f(x) = h(x) g(x) + r(x) \pod{\deg r(x) < \deg g(x)}
	$$

	带余除法中,$f(x)$ 称为\emph{\idx{被除式}},$g(x)$ 称为\emph{\idx{除式}},$q(x)$ 称为\emph{\idx{商式}},$r(x)$ 称为\emph{\idx{余式}}。
\end{proposition}

\begin{proof}[对 $f(x)$ 的次数作数学归纳法]
	存在性。若 $\deg g(x) = 0$,此时有 $f(x) = \biggl( \dfrac{1}{b} f(x) \biggr) b + 0$,显然成立。若 $\deg g(x) > \deg f(x) \ge 0$,此时有 $f(x) = 0 \cdot g(x) + f(x)$,显然成立。下面只用考虑 $\deg f(x) \ge \deg g(x) > 0$ 的情况。

	对 $f(x)$ 的次数作数学归纳法。假设对于次数小于 $n$ 的被除式命题的存在性部分成立,现在来看 $n$ 次多项式 $f(x)$。设 $f(x), g(x)$ 的首项分别是 $a_n x^n$、$b_m x^m$,于是 $a_n b_m^{-1} x^{n - m} g(x)$ 的首项是 $a_n x^n$。设:
	$$
	f_1(x) = f(x) - a_n b_m^{-1} x^{n - m} g(x)
	$$

	则 $\deg f_1(x) < n$。根据归纳假设,存在 $h_1(x), r_1(x) \in K[x]$,使得:
	$$
	f_1(x) = h_1(x) g(x) + r_1(x) \pod{\deg r_1(x) < \deg g(x)}
	$$

	代入到 $f(x)$ 中,得:
	$$
	f(x) = (h_1(x) + a_n b_m^{-1} x^{n - m}) g(x) + r_1(x)
	$$

	唯一性。若 $f(x) = g(x) q(x) + r(x) = g(x) q_1(x) + r_1(x)$,则可得 $g(x) (q(x) - q_1(x)) = r_1(x) - r(x)$。求两侧多项式的次数,根据:
	$$
	\max \{ \deg r(x), r_1(x) \} < g(x)
	$$

	易得:
	$$
	\deg(q(x) - q_1(x)) < 0 = -\infty \Longrightarrow q(x) = q_1(x)
	$$$$
	r(x) = r_1(x)
	$$
\end{proof}

\subsection{整除关系与带余除法的性质}

容易证明,整除关系满足以下事实:
\begin{enumerate}
	\item 有且仅有 $0$ 能被 $0$ 整除:
	$$
	0 \mid f(x) \Longleftrightarrow f(x) = 0
	$$

	\item 任何多项式都能整除 $0$:
	$$
	\forall f(x) \in \mathbb K[x], f(x) \mid 0
	$$
	\item 任何非零数都能整除所有多项式:
	$$
	\forall b \in \mathbb K \backslash \set{0}, \forall f(x) \in \mathbb K[x], b \mid f(x)
	$$
\end{enumerate}

作为一个二元关系,整除满足反身性和传递性,但不具有对称性。如果两个多项式之间的整除满足对称性,这两个多项式满足什么特殊条件?我们给出以下定义和定理。

\begin{definition}{相伴}
	在 $\mathbb K[x]$ 中,如果 $g(x) \mid f(x)$ 且 $f(x) \mid g(x)$,那么称 $f(x)$ 与 $g(x)$ \emph{相伴},记作 $f(x) \sim g(x)$。
\end{definition}

\begin{proposition}
	在 $\mathbb K[x]$ 中,$f(x) \sim g(x)$ 当且仅当存在 $c \in \mathbb K \backslash \set{0}$,使得 $f(x) = c g(x)$。
\end{proposition}

\begin{proof}
	充分性显然,只证必要性。已知 $f(x) \mid g(x)$ 且 $g(x) \mid f(x)$,所以存在 $h_1(x), h_2(x) \in \mathbb K[x]$,使得:
	$$
	g(x) = h_1(x) f(x)
	$$$$
	f(x) = h_2(x) g(x)
	$$

	从而有:
	$$
	f(x) = h_2(x) h_1(x) f(x)
	$$

	如果 $f(x) = 0$,那么 $g(x) = 0$,否则可得 $h_2(x) h_1(x) = 1$,由次数关系,可知 $h_2(x) = c \pod{c \in K \backslash \{0\}}$。
\end{proof}

\begin{proposition}
	在 $\mathbb K[x]$ 中,如果 $g(x) \mid f_i(x) \pod{i = 1, 2, \ldots, s}$,那么对于任意 $u_1(x), \ldots, u_s(x) \in \mathbb K[x]$,都有:
	$$
	g(x) \mid \bigl( u_1(x) f_1(x) + \cdots + u_s(x) f_s(x) \bigr)
	$$
\end{proposition}

\begin{proof}
	由已知条件得,存在 $h_i(x) \in \mathbb K[x]$,使得:
	$$
	f_i(x) = h_i(x) g(x) \pod{i = 1, 2, \ldots, s}
	$$

	从而原式右侧等于:
	$$
	\bigl( u_1(x) h_1(x) + \cdots + u_s(x) h_s(x) \bigr) g(x)
	$$

	因此 $g(x) \mid (u_1(x) f_1(x) + \cdots + u_s(x) f_s(x))$。
\end{proof}

\subsection{例题}

