% Licensed under the Creative Commons Attribution Share Alike 4.0 International.
% See the LICENCE file in the repository root for full licence text.

\section{矩阵乘法运算的性质}

本节中,我们介绍矩阵乘法一系列最基本的性质。

\begin{enumerate}
	\item 结合律。

	\begin{theorem}[矩阵乘法的结合律]
		设 $A = (a_{ij})_{s \times n}, B = (b_{ij})_{n \times m}, C = (c_{ij})_{m \times r}$。有:
		$$
		(AB)C = A(BC)
		$$
	\end{theorem}

	\begin{proof}
		显然运算结果都是 $s \times r$ 矩阵,考察该矩阵中的每一个元。
		$$
		\begin{aligned}&
			\bigl( (AB)C \bigr)(i; j) \pod{1 \le i \le s; 1 \le j \le r}
			\\=~&
			\sum\limits_{k = 1}^n (AB)(i; k) \cdot C(k; j)
			\\=~&
			\sum\limits_{k = 1}^n \biggl( \sum\limits_{l = 1}^s A(i; l) B(l; k) \biggr) C(k; j)
			\\=~&
			\sum\limits_{l = 1}^s A(i; l) \sum\limits_{k = 1}^n B(l; k) C(k; j)
			\\=~&
			\sum\limits_{l = 1}^s A(i; l) \cdot (BC)(i; j)
			\\=~&
			\bigl( A(BC) \bigr)(i; j)
		\end{aligned}
		$$
	\end{proof}

	\item 不满足交换律。

	矩阵乘法不满足交换律,以下三种情况均有可能发生:
	\begin{enumerate}
		\item $A$ 与 $B$ 可以做乘法,但 $B$ 与 $A$ 不可以做乘法。
		\item $A$ 与 $B$ 可以做乘法,$B$ 与 $A$ 也可以做乘法,但 $AB$ 与 $BA$ 的维数不同。
		\item $AB$ 与 $BA$ 的维数相同,但结果不同。
	\end{enumerate}

	\item 不满足消去律。

	即,从 $AB = 0$ 不同推出 $A = 0$ 或 $B = 0$。见下例:
	$$
	\begin{bmatrix} 0 & 0 \\ 0 & 1 \end{bmatrix}
	\begin{bmatrix} 0 & 1 \\ 0 & 0 \end{bmatrix}
	=
	\begin{bmatrix} 0 & 0 \\ 0 & 0 \end{bmatrix}
	$$

	不过,我们可以提出以下概念。

	\begin{definition}{左零因子}
		对于矩阵 $A$,如果存在一个矩阵 $B \ne 0$ 使得 $AB = 0$,那么称 $A$ 是一个\emph{左零因子}。
	\end{definition}

	\begin{definition}{右零因子}
		对于矩阵 $A$,如果存在一个矩阵 $C \ne 0$ 使得 $CA = 0$,那么称 $A$ 是一个\emph{右零因子}。
	\end{definition}

	\begin{definition}{零因子}
		左零因子和右零因子统称为\emph{零因子}。
	\end{definition}

	\begin{definition}{平凡的零因子}
		显然,零矩阵是零因子,称它为\emph{平凡的零因子}。
	\end{definition}

	\item 分配律。

	\begin{theorem}{矩阵乘法的左分配律,矩阵乘法的右分配律}
		矩阵的乘法适合\emph{左分配律}:
		$$
		A (B + C) = AB + AC
		$$

		也适合右分配律:
		$$
		(B + C) D = BD + CD
		$$
	\end{theorem}

	\begin{proof}
		只证明左分配律。
		$$
		\begin{aligned}&
			\bigl( A(B + C) \bigr)(i; j) \pod{1 \le i \le s; 1 \le j \le n}
			\\=~&
			\sum\limits_{k = 1}^m A(i; k) \cdot (B + C)(k; j)
			\\=~&
			\sum\limits_{k = 1}^m A(i; k) B(k; j) + A(i; k) C(k; j)
			\\=~&
			AB + AC
		\end{aligned}
		$$
	\end{proof}

	\item 存在单位元。

	\begin{definition}{$n$ 级单位矩阵}
		主对角线上元素都是 $1$,其余元素都是 $0$ 的 $n$ 级矩阵称为 \emph{$n$ 级单位矩阵},记作 $I_n$,或简记为 $I$。
	\end{definition}

	\begin{theorem}
		$n$ 级单位矩阵是矩阵乘法的单位元:
		$$
		I_s A_{s \times n} = A_{s \times n}, \qquad A_{s \times n} I_n = A_{s \times n}
		$$

		特别地,如果 $A$ 是 $n$ 级矩阵,则:
		$$
		IA = AI = A
		$$
	\end{theorem}

\end{enumerate}

我们再来看一系列特殊的矩阵或运算。

\subsubsection{数量矩阵}

根据矩阵乘法的定义式,很容易得到以下定理。

\begin{theorem}
	矩阵的乘法与数量乘法满足:
	$$
	k(AB) = (kA)B = A(kB)
	$$
\end{theorem}

虽然前面已经用到了这个性质,但因为它太简单了,所以此处仍然省略其证明。不过,下面这个在它基础上的概念却十分重要。

\begin{definition}{数量矩阵}
	主对角线上元素是同一个数 $k$,其余元素全为 $0$ 的 $n$ 级矩阵称为\emph{数量矩阵},它可以写成 $kI$。
\end{definition}

以下是关于数量矩阵的两个显然但重要的结论。

\begin{theorem}[数量矩阵的运算封闭性]
	$n$ 级数量矩阵组成的集合对于加法、数量乘法、矩阵乘法这三种运算都封闭。
\end{theorem}

\begin{theorem}
	数量矩阵 $kI$ 左乘或右乘矩阵 $A$ 等于 $k$ 乘 $A$,即:
	$$
	(kI)A = kA = A(kI)
	$$
\end{theorem}

\subsubsection{可交换的矩阵}

\begin{definition}{可交换}
	矩阵的乘法不适合交换律,但是对于具体的两个矩阵 $A$ 与 $B$,也有可能 $AB = BA$。如果 $AB = BA$,则称 $A$ 与 $B$ 可交换。
\end{definition}

\begin{theorem}
	数量矩阵与任一同级矩阵可交换。
\end{theorem}

\subsubsection{矩阵的非负整数次幂}

由于矩阵的乘法适合结合律,因而可以定义 $n$ 级矩阵的非负整数次幂。

\begin{definition}{$n$ 级矩阵的非负整数次幂}
	定义 \emph{$n$ 级矩阵 $A$ 的非负整数次幂}为:
	$$
	A^m = \underset{\text{$m$ 个}}{\underbrace{A \cdot A \cdots A}} \pod{m \in Z^+}
	$$

	特别地,定义 $A^0 = I$。
\end{definition}

\subsubsection{矩阵的转置}

我们已经提出过转置的概念,下面给出在矩阵运算中引入转置后的一些性质。

\begin{theorem}
	$$
	(A')' = A
	$$$$
	(A + B)' = A' + B'
	$$$$
	(kA)' = kA'
	$$$$
	(AB)' = B'A'
	$$
\end{theorem}