% Licensed under the Creative Commons Attribution Share Alike 4.0 International.
% See the LICENCE file in the repository root for full licence text.

\section{矩阵乘法运算的性质}

本节中,我们介绍矩阵乘法一系列最基本的性质。

\begin{enumerate}
	\item 结合律。

	\begin{theorem}[矩阵乘法的结合律]
		设 $A = (a_{ij})_{s \times n}, B = (b_{ij})_{n \times m}, C = (c_{ij})_{m \times r}$。有:
		$$
		(AB)C = A(BC)
		$$
	\end{theorem}

	\begin{proof}
		显然运算结果都是 $s \times r$ 矩阵,考察该矩阵中的每一个元。
		$$
		\begin{aligned}&
			\bigl( (AB)C \bigr)(i; j) \pod{1 \le i \le s; 1 \le j \le r}
			\\=~&
			\sum\limits_{k = 1}^n (AB)(i; k) \cdot C(k; j)
			\\=~&
			\sum\limits_{k = 1}^n \biggl( \sum\limits_{l = 1}^s A(i; l) B(l; k) \biggr) C(k; j)
			\\=~&
			\sum\limits_{l = 1}^s A(i; l) \sum\limits_{k = 1}^n B(l; k) C(k; j)
			\\=~&
			\sum\limits_{l = 1}^s A(i; l) \cdot (BC)(i; j)
			\\=~&
			\bigl( A(BC) \bigr)(i; j)
		\end{aligned}
		$$
	\end{proof}

	\item 不满足交换律。

	矩阵乘法不满足交换律,以下三种情况均有可能发生:
	\begin{enumerate}
		\item $A$ 与 $B$ 可以做乘法,但 $B$ 与 $A$ 不可以做乘法。
		\item $A$ 与 $B$ 可以做乘法,$B$ 与 $A$ 也可以做乘法,但 $AB$ 与 $BA$ 的维数不同。
		\item $AB$ 与 $BA$ 的维数相同,但结果不同。
	\end{enumerate}

	\item 不满足消去律。

	即,从 $AB = 0$ 不同推出 $A = 0$ 或 $B = 0$。见下例:
	$$
	\begin{bmatrix} 0 & 0 \\ 0 & 1 \end{bmatrix}
	\begin{bmatrix} 0 & 1 \\ 0 & 0 \end{bmatrix}
	=
	\begin{bmatrix} 0 & 0 \\ 0 & 0 \end{bmatrix}
	$$

	不过,我们可以提出以下概念。

	\begin{definition}{左零因子}
		对于矩阵 $A$,如果存在一个矩阵 $B \ne 0$ 使得 $AB = 0$,那么称 $A$ 是一个\emph{左零因子}。
	\end{definition}

	\begin{definition}{右零因子}
		对于矩阵 $A$,如果存在一个矩阵 $C \ne 0$ 使得 $CA = 0$,那么称 $A$ 是一个\emph{右零因子}。
	\end{definition}

	\begin{definition}{零因子}
		左零因子和右零因子统称为\emph{零因子}。
	\end{definition}

	\begin{definition}{平凡的零因子}
		显然,零矩阵是零因子,称它为\emph{平凡的零因子}。
	\end{definition}

	\item 分配律。

	\begin{theorem}{矩阵乘法的左分配律,矩阵乘法的右分配律}
		矩阵的乘法适合\emph{左分配律}:
		$$
		A (B + C) = AB + AC
		$$

		也适合右分配律:
		$$
		(B + C) D = BD + CD
		$$
	\end{theorem}



\end{enumerate}