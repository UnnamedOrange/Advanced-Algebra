% Licensed under the Creative Commons Attribution Share Alike 4.0 International.
% See the LICENCE file in the repository root for full licence text.

\section{行列式的等价定义}

除了行列式的第一定义,还可以直接使用行列式的完全展开式来定义行列式。

\begin{definition}{行列式}
	定义映射 $f: M_n(\mathbb F) \to \mathbb F$,其中 $\mathbb F$ 表示一个数域,$M_n(\mathbb F)$ 表示元素属于数域 $\mathbb F$ 的 $n$ 级矩阵组成的集合。若 $f$ 满足:
	$$
	f(A) = \sum_{j_1 j_2 \ldots j_n} (-1)^{\tau(j_1 j_2 \ldots j_n)} a_{1j_1} a_{2j_2} \cdots a_{nj_n}
	$$
	其中 $j_1 j_2 \ldots j_n$ 表示枚举 $123 \ldots n$ 的所有排列,$a_{ij}$ 是矩阵 $A$ 的 $(i, j)$ 元,则称 $f(A)$ 为方阵 $A$ 的\emph{行列式},记为 $\det A$ 或 $|A|$。
\end{definition}

现在只需验证按完全展开式定义的行列式满足归一化、多线性和反对称,即可证明这\textbf{两种定义的等价性}。

\begin{enumerate}
	\item 归一化。代入单位矩阵 $I_n$,显然有 $|I_n| = 1$。
	\item 多线性。考虑:
	$$
	\begin{aligned}
		f(A) &= \sum_{j_1 j_2 \ldots j_n} (-1)^{\tau(j_1 j_2 \ldots j_n)} (x a_{1j_1} + y b_{1 j_1}) a_{2j_2} \cdots a_{nj_n}
		\\&=
		x \sum_{j_1 j_2 \ldots j_n} (-1)^{\tau(j_1 j_2 \ldots j_n)} a_{1j_1} a_{2j_2} \cdots a_{nj_n} +
		\\&~~~~
		y \sum_{j_1 j_2 \ldots j_n} (-1)^{\tau(j_1 j_2 \ldots j_n)} b_{1j_1} a_{2j_2} \cdots a_{nj_n}
	\end{aligned}
	$$
	故成立。
	\item 反对称。交换矩阵 $A$ 的两行,则完全展开式中每一项的 $\tau(j_1 j_2 \ldots j_n)$ 的奇偶性改变,故成立。
\end{enumerate}

行列式的定义还有一个变种,可以将前两种定义中提到的行向量改为列向量。例如,按列向量定义的行列式的完全展开式被定义为:
$$
\sum_{k_1 k_2 \ldots k_n} (-1)^{\tau(k_1 k_2 \ldots k_n)} a_{k_1 1} a_{k_2 2} \cdots a_{k_n n}
$$
与行向量定义的唯一差别是 $i j_i$ 变为了 $k_i i$。

可以证明,按行向量和按列向量定义的行列式是等价的。

\begin{proof}
	只需证明两个完全展开式相等。注意到,两式都是对 $n!$ 项进行求和,因此只需证明这 $n!$ 项是相同的。

	行向量定义为:
	$$
	\sum_{k_1 k_2 \ldots k_n} (-1)^{\tau(k_1 k_2 \ldots k_n)} a_{1k_1} a_{2k_2} \cdots a_{nk_n}
	$$

	将上式中的每一项进行重排,按列编号排序,得:
	$$
	\sum_{k_1 k_2 \ldots k_n} (-1)^{\tau(k_1 k_2 \ldots k_n)} a_{i_1 k_{i_1}} a_{i_2 k_{i_2}} \cdots a_{i_n k_{i_n}}
	$$
	其中 $k_{i_j} = j$,即上式等于:
	$$
	\sum_{k_1 k_2 \ldots k_n} (-1)^{\tau(k_1 k_2 \ldots k_n)} a_{i_1 1} a_{i_2 2} \cdots a_{i_n n}
	$$

	由于排列 $k_1 k_2 \ldots k_n$ 与排列 $i_1 i_2 \ldots i_n$ 是一一对应关系(因为对于任意的 $j$,有 $k_{i_j} = j$),因此只需证明 $(-1)^{\tau(k_1 k_2 \ldots k_n)} = (-1)^{\tau(i_1 i_2 \ldots i_n)}$。

	\bigskip

	在重排前,行标号和列标号分别为:
	$$
	\begin{matrix}
		1 & 2 & 3 & \cdots & n & \pod{\text{行标号}}
		\\
		k_1 & k_2 & k_3 & \cdots & k_n & \pod{\text{列标号}}
	\end{matrix}
	$$

	重排的操作相当于同时交换行标号和列标号,例如:
	$$
	\begin{matrix}
		2 & 1 & 3 & \cdots & n & \pod{\text{行标号}}
		\\
		k_2 & k_1 & k_3 & \cdots & k_n & \pod{\text{列标号}}
	\end{matrix}
	$$

	由于行标号排列和列标号排列的奇偶性同时发生改变,因此\textbf{行标号排列与列标号排列的逆序数之和的奇偶性不变}。而行标号排列的初态与列标号排列的末态均为偶排列(逆序数为 $0$),所以列标号排列的初态与行标号排列的末态的奇偶性相同,即有 $(-1)^{\tau(k_1 k_2 \ldots k_n)} = (-1)^{\tau(i_1 i_2 \ldots i_n)}$。
\end{proof}

\begin{definition}{转置}
	互换矩阵 $A_{m \times n}$ 的行与列可得到一个 $n \times m$ 的矩阵,称之为 $A$ 的\emph{转置},记作 $A'$、$A^T$ 或 $A^t$。
\end{definition}

由行列式的等价定义,以下定理显然。

\begin{theorem}
	对于 $n$ 级方阵 $A$,满足 $|A^T| = |A|$。
\end{theorem}

受行列式等价定义的启发,我们可以得到以下定理。

\begin{theorem}
	对于任一 $1$ 到 $n$ 的 $n$ 元排列 $k_1 k_2 \ldots k_n$,$n$ 级矩阵 $A$ 的行列式 $\det A$ 满足:
	$$
	\begin{aligned}
		\det A &= (-1)^{\tau(k_1 k_2 \ldots k_n)} \sum\limits_{i_1 i_2 \ldots i_n} (-1)^{\tau (i_1 i_2 \ldots i_n)} a_{k_1 i_1} \cdots a_{k_n i_n}
		\\&=
		(-1)^{\tau(k_1 k_2 \ldots k_n)} \sum\limits_{i_1 i_2 \ldots i_n} (-1)^{\tau (i_1 i_2 \ldots i_n)} a_{i_1 k_1} \cdots a_{i_n k_n}
	\end{aligned}
	$$
\end{theorem}

证明略,思路与上文行列式等价定义的证明类似。

\section{行列式与高斯约当算法的关系}

高斯约当算法是一系列初等行变换的组合。考察方阵的初等行变换对其行列式的影响,可以得到以下定理。

\begin{theorem}
	$n$ 级矩阵的初等行变换不改变其行列式的非零性。即,若原行列式不为 $0$,则经过初等行变换后,行列式也不为 $0$。反之,若原行列式为 $0$,则经过初等行变换后也为 $0$。
\end{theorem}

\begin{proof}
	设原矩阵为 $A$,经过初等行变换后的矩阵为 $A_1$,分别考察初等行变换对行列式的影响:
	\begin{enumerate}
		\item 交换两行。由反对称可知,$|A_1| = -|A|$。
		\item 某一行乘上某一个非零数 $c$。由多线性可知,$|A_1| = c \cdot |A|$。
		\item 把第 $i$ 行乘以 $c$ 加至第 $j$ 行。$|A_1| = |A|$。
	\end{enumerate}

	其中,第三点的证明如下。利用行列式的多线性得 $|A_1| = |A| + |A'|$,其中 $A'$ 是把 $A$ 的第 $j$ 行替换为第 $i$ 行的 $c$ 倍的矩阵。又由行列式的多线性与反对称,得 $|A'| = 0$,即证得 $|A_1| = |A|$。
\end{proof}

将高斯约当算法作用于一个系数方阵,将得到一个简化阶梯形方阵。研究简化阶梯形方阵的行列式,可以得到以下定理。

\begin{theorem}
	系数矩阵为 $n$ 级方阵的线性方程组有唯一解的充分必要条件是系数矩阵的行列式非零。
\end{theorem}

这是因为当且仅当 $n \times n$ 的简化阶梯形矩阵的对角线全为 $1$ 时,其行列式为 $1$,否则为 $0$。进一步,可得以下定理。

\begin{theorem}
	若系数矩阵的行列式为 $0$,则对应的线性方程组无解或有无穷多解,反之亦然。
\end{theorem}