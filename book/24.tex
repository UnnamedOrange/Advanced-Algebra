% Licensed under the Creative Commons Attribution Share Alike 4.0 International.
% See the LICENCE file in the repository root for full licence text.

\section{矩阵的相似}

矩阵的相似是本笔记中最重要的内容之一,其研究动力如下。为了求 $n$ 级矩阵 $A$ 的方幂 $A^m$,如果能找到 $n$ 级可逆矩阵 $P$,使得 $P^{-1} AP = D$,其中 $D$ 是对角矩阵,那么 $A = PDP^{-1}$,从而 $A^m = PD^m P^{-1}$,而对角矩阵 $D$ 的方幂很容易计算,于是 $A^m$ 也就比较容易算出了。据此,我们定义相似的概念。

\subsection{相似的概念}

\begin{definition}{相似}
	设 $A$ 与 $B$ 都是数域 $K$ 上的 $n$ 级矩阵,如果存在数域 $\mathbb K$ 上一个 $n$ 级可逆矩阵 $P$,使得 $P^{-1} A P = B$,那么称 $A$ 与 $B$ 是\emph{相似}的,记作 $A \sim B$。
\end{definition}

\begin{theorem}
	相似是一个等价关系。
\end{theorem}

\begin{definition}{相似类}
	在相似关系下,$A$ 的等价类称为 $A$ 的\emph{相似类}。
\end{definition}

容易证明,矩阵的相似满足以下性质。

\begin{theorem}[相似的运算性质]
	如果 $B_1 = P^{-1} A_1 P$,$B_2 = P^{-1} A_2 P$,$m$ 是正整数,那么:
	\begin{enumerate}
		\item $B_1 + B_2 = P^{-1} (A_1 + A_2) P$。
		\item $B_1 B_2 = P^{-1} (A_1 A_2) P$。
		\item $B_1^m = P^{-1} A_1^m P$。
	\end{enumerate}
\end{theorem}

\begin{theorem}
	相似的矩阵的行列式是一个不变量,但不是完全不变量。
\end{theorem}

\begin{theorem}
	相似的矩阵或者都可逆,或者都不可逆;当它们可逆时,它们的逆矩阵也相似。
\end{theorem}

\begin{theorem}
	相似的矩阵的秩是一个不变量,但不是完全不变量。
\end{theorem}

下面我们提出矩阵的迹的概念,并证明相似的矩阵的迹也是一个不变量。

\begin{definition}{迹}
	$n$ 级矩阵 $A = (a_{ij})$ 的主对角线上的元素之和称为 $A$ 的\emph{迹},记作 $\operatorname{tr}(A)$, 即 $\operatorname{tr}(A) = a_{11} + a_{22} + \cdots + a_{nn}$。
\end{definition}

\begin{theorem}[矩阵的迹的运算性质]
	\begin{enumerate}
		\item 对 $n$ 级矩阵,$\operatorname{tr}(A + B) = \operatorname{tr}(A) + \operatorname{tr}(B)$。
		\item 对 $n$ 级矩阵,$\operatorname{tr}(kA) = k \operatorname{tr}(A)$。
		\item 对 $n \times m$ 矩阵 $A$ 和 $m \times n$ 矩阵 $B$,$\operatorname{tr}(AB) = \operatorname{tr}(BA)$。
	\end{enumerate}
\end{theorem}

\begin{proof}[只证明 3.]
	$$
	\begin{aligned}
		\operatorname{tr}(AB) &= \sum_{i = 1}^n (AB)(i; i)
		\\&=
		\sum_{i = 1}^n \sum_{k = 1}^m a_{ik} b_{ki}
		\\&=
		\sum_{k = 1}^m \sum_{i = 1}^n b_{ki} a_{ik}
		\\&=
		\sum_{k = 1}^m (BA)(k;k) = \operatorname{tr}(BA)
	\end{aligned}
	$$
\end{proof}

可见,矩阵的迹是从矩阵乘法的非交换性中提取的可交换的量。

\begin{theorem}
	相似的矩阵有相等的迹。
\end{theorem}

\begin{proof}
	$$
	\operatorname{tr}(B) = \operatorname{tr}(P^{-1} A P) = \operatorname{tr}(APP^{-1}) = \operatorname{tr}(A)
	$$
\end{proof}

综合以上定理,我们总结出相似不变量的概念。

\begin{definition}{相似不变量}
	矩阵的行列式、秩、迹都是相似关系下的不变量,简称为\emph{相似不变量}。
\end{definition}

% TODO: 补充例题。

\subsection{矩阵可对角化的概念}

