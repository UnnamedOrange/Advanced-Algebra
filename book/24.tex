% Licensed under the Creative Commons Attribution Share Alike 4.0 International.
% See the LICENCE file in the repository root for full licence text.

\section{矩阵的相似}

矩阵的相似是本笔记中最重要的内容之一,其研究动力如下。为了求 $n$ 级矩阵 $A$ 的方幂 $A^m$,如果能找到 $n$ 级可逆矩阵 $P$,使得 $P^{-1} AP = D$,其中 $D$ 是对角矩阵,那么 $A = PDP^{-1}$,从而 $A^m = PD^m P^{-1}$,而对角矩阵 $D$ 的方幂很容易计算,于是 $A^m$ 也就比较容易算出了。据此,我们定义相似的概念。

\subsection{相似的概念}

\begin{definition}{相似}
	设 $A$ 与 $B$ 都是数域 $K$ 上的 $n$ 级矩阵,如果存在数域 $\mathbb K$ 上一个 $n$ 级可逆矩阵 $P$,使得 $P^{-1} A P = B$,那么称 $A$ 与 $B$ 是\emph{相似}的,记作 $A \sim B$。
\end{definition}

\begin{theorem}
	相似是一个等价关系。
\end{theorem}

\begin{definition}{相似类}
	在相似关系下,$A$ 的等价类称为 $A$ 的\emph{相似类}。
\end{definition}

容易证明,矩阵的相似满足以下性质。

\begin{theorem}[相似的运算性质]
	如果 $B_1 = P^{-1} A_1 P$,$B_2 = P^{-1} A_2 P$,$m$ 是正整数,那么:
	\begin{enumerate}
		\item $B_1 + B_2 = P^{-1} (A_1 + A_2) P$。
		\item $B_1 B_2 = P^{-1} (A_1 A_2) P$。
		\item $B_1^m = P^{-1} A_1^m P$。
	\end{enumerate}
\end{theorem}

\begin{theorem}
	相似的矩阵的行列式是一个不变量,但不是完全不变量。
\end{theorem}

\begin{theorem}
	相似的矩阵或者都可逆,或者都不可逆;当它们可逆时,它们的逆矩阵也相似。
\end{theorem}

\begin{theorem}
	相似的矩阵的秩是一个不变量,但不是完全不变量。
\end{theorem}

下面我们提出矩阵的迹的概念,并证明相似的矩阵的迹也是一个不变量。

\begin{definition}{迹}
	$n$ 级矩阵 $A = (a_{ij})$ 的主对角线上的元素之和称为 $A$ 的\emph{迹},记作 $\operatorname{tr}(A)$, 即 $\operatorname{tr}(A) = a_{11} + a_{22} + \cdots + a_{nn}$。
\end{definition}

\begin{theorem}[矩阵的迹的运算性质]
	\begin{enumerate}
		\item 对 $n$ 级矩阵,$\operatorname{tr}(A + B) = \operatorname{tr}(A) + \operatorname{tr}(B)$。
		\item 对 $n$ 级矩阵,$\operatorname{tr}(kA) = k \operatorname{tr}(A)$。
		\item 对 $n \times m$ 矩阵 $A$ 和 $m \times n$ 矩阵 $B$,$\operatorname{tr}(AB) = \operatorname{tr}(BA)$。
	\end{enumerate}
\end{theorem}

\begin{proof}[只证明 3.]
	$$
	\begin{aligned}
		\operatorname{tr}(AB) &= \sum_{i = 1}^n (AB)(i; i)
		\\&=
		\sum_{i = 1}^n \sum_{k = 1}^m a_{ik} b_{ki}
		\\&=
		\sum_{k = 1}^m \sum_{i = 1}^n b_{ki} a_{ik}
		\\&=
		\sum_{k = 1}^m (BA)(k;k) = \operatorname{tr}(BA)
	\end{aligned}
	$$
\end{proof}

可见,矩阵的迹是从矩阵乘法的非交换性中提取的可交换的量。

\begin{theorem}
	相似的矩阵有相等的迹。
\end{theorem}

\begin{proof}
	$$
	\operatorname{tr}(B) = \operatorname{tr}(P^{-1} A P) = \operatorname{tr}(APP^{-1}) = \operatorname{tr}(A)
	$$
\end{proof}

综合以上定理,我们总结出相似不变量的概念。

\begin{definition}{相似不变量}
	矩阵的行列式、秩、迹都是相似关系下的不变量,简称为\emph{相似不变量}。
\end{definition}

% TODO: 补充例题。

\subsection{矩阵可对角化的概念}

\begin{definition}{可对角化}
	如果 $n$ 级矩阵 $A$ 能够相似于一个对角矩阵,那么称 $A$ \emph{可对角化}。
\end{definition}

我们直接给出矩阵可对角化的充分必要条件。

\begin{theorem}[矩阵可对角化的充分必要条件]
	数域 $\mathbb K$ 上 $n$ 级矩阵 $A$ 可对角化的充分必要条件是,$\mathbb K^n$ 中有 $n$ 个线性无关的列向量 $\vec \alpha_1, \ldots, \vec \alpha_n$,以及 $\mathbb K$ 中有 $n$ 个数 $\lambda_1, \ldots, \lambda_n$(它们之中可能相等),使得 $A \vec \alpha_i = \lambda_i \vec \alpha_i \pod{i = 1, \ldots, n}$。这时,令 $P = \begin{bmatrix} \vec \alpha_1 & \cdots & \vec \alpha_n \end{bmatrix}$,则 $P^{-1} AP = \operatorname{diag} \{ \lambda_1, \ldots, \lambda_n \}$。
\end{theorem}

\begin{proof}
	设 $D = \operatorname{diag} \{ \lambda_1, \ldots, \lambda_n \}$,则以下命题互为充分必要条件:

	\begin{itemize}
		\item $A$ 相似于一个对角矩阵,即 $P^{-1} AP = D$。
		\item $AP = PD$,设 $P = \begin{bmatrix} \vec \alpha_1 & \cdots & \vec \alpha_n \end{bmatrix}$。
		\item $\begin{bmatrix} A \vec \alpha_1 & \cdots & A \vec \alpha_n \end{bmatrix} =  \begin{bmatrix} \lambda_1 \vec \alpha_1 & \cdots & \lambda_n \vec \alpha_n \end{bmatrix}$
		\item $A \vec \alpha_i = \lambda_i \vec \alpha_i \pod{i = 1, \ldots, n}$,其中 $\vec \alpha_1, \cdots, \vec \alpha_n$ 线性无关。
	\end{itemize}
\end{proof}

\subsection{特征多项式}

根据以上关于矩阵可对角化的分析,我们可以提出以下概念。

\begin{definition}{特征值,特征向量}
	设 $A$ 是数域 $\mathbb K$ 上的 $n$ 级矩阵,如果 $\mathbb K^n$ 中有非零列向量 $\vec \alpha$,使得 $A \vec \alpha = \lambda_0 \vec \alpha \pod{\lambda_0 \in \mathbb K}$,那么称 $\lambda_0$ 是 $A$ 的一个\emph{特征值},称 $\vec \alpha$ 是 $A$ 的属于特征值 $\lambda_0$ 的一个\emph{特征向量}。
\end{definition}

注意,根据定义,零向量不是 $A$ 的特征向量。

关于特征值与特征向量,我们首先给出以下显然的结论。

\begin{theorem}
	如果 $\vec \alpha$ 是 $A$ 的属于 $\lambda_0$ 的一个特征向量,那么对于任意 $k \in \mathbb K \pod{k \ne 0}$,$k \vec \alpha$ 也是 $A$ 的属于 $\lambda_0$ 的特征向量。
\end{theorem}

\begin{proof}
	$$
	A(k \vec \alpha) = k(A \vec \alpha) = k(\lambda_0 \vec \alpha) = \lambda_0 (k \vec \alpha)
	$$
\end{proof}

根据以上定义,又受可对角化的充分必要条件的推导过程启发,我们可知以下命题互为充分必要条件:

\begin{itemize}
	\item $\lambda_0$ 是 $A$ 的一个特征值,$\vec \alpha$ 是 $A$ 的属于 $\lambda_0$ 的一个特征向量。
	\item $A \vec \alpha = \lambda_0 \vec \alpha \pod{\vec \alpha \ne \vec 0, \lambda_0 \in \mathbb K}$
	\item $(\lambda_0 I - A) \vec \alpha = \vec 0 \pod{\vec \alpha \ne \vec 0, \lambda_0 \in \mathbb K}$
	\item $\vec \alpha$ 是齐次线性方程组 $(\lambda_0 I - A) \vec x = \vec 0$ 的一个非零解($\lambda_0 \in \mathbb K$)。
	\item $|\lambda_0 I - A| = 0$,$\vec \alpha$ 是 $(\lambda_0 I - A) \vec x = \vec 0$ 的一个非零解($\lambda_0 \in \mathbb K$)。
	\item $\lambda_0$ 是\textbf{多项式} $|\lambda I - A|$ 在 $\mathbb K$ 中的一个根,$\vec \alpha$ 是 $(\lambda_0 I - A) \vec x = \vec 0$ 的一个非零解。
\end{itemize}

将行列式 $|\lambda I - A|$ 的完全展开式看作一个关于 $\lambda$ 的多项式,即得到了上文中的多项式 $|\lambda I - A|$。
$$
|\lambda I - A| =
\begin{vmatrix}
\lambda - a_{11} & -a_{12} & \cdots & -a_{1n}
\\
-a_{21} & \lambda - a_{22} & \cdots & -a_{2n}
\\
\vdots & \vdots & & \vdots
\\
-a_{n1} & -a_{n2} & \cdots & \lambda - a_{nn}
\end{vmatrix}
$$

\begin{definition}{特征多项式}
	把 $|\lambda I - A|$ 称为 $A$ 的\emph{特征多项式}。
\end{definition}

以下定理描述了多项式 $|\lambda I_n - A|$ 的结构。

\begin{theorem}
	设 $A$ 是数域 $\mathbb K$ 上的 $n$ 级矩阵,则 $A$ 的特征多项式 $|\lambda I - A|$ 是一个 $n$ 次多项式,$\lambda^n$ 的系数是 $1$,$\lambda^{n - 1}$ 的系数等于 $-\operatorname{tr}(A)$,常数项为 $(-1)^n |A|$,$\lambda^{n - k}$ 的系数为 $A$ 的所有 $k$ 阶主子式的和乘以 $(-1)^k \pod{1 \le k < n}$。
\end{theorem}

\begin{proof}
	设 $A = (a_{ij})$ 的列向量组是 $\vec \alpha_1, \ldots, \vec \alpha_n$,写出 $|\lambda I - A|$:
	$$
	|\lambda I - A| =
	\begin{vmatrix} \lambda \vec e_1 - \vec \alpha_1 & \lambda \vec e_2 - \vec \alpha_2 & \cdots & \lambda \vec e_n - \vec \alpha_n \end{vmatrix}
	$$
	利用行列式的多线性,$|\lambda I - A|$ 可以拆成 $2^n$ 个行列式的和,我们把它按全是 $\vec e$,全是 $\vec \alpha$,既有 $\vec e$ 又有 $\vec \alpha$ 分成三类。

	第一类,显然这个行列式等于 $\lambda^n$。第二类,显然这个行列式等于 $(-1)^n |A|$。对于第三类行列式,设第 $j_1, j_2, \ldots, j_{n - k}$ 列是 $\vec e$,于是可以将这个行列式按第 $j_1, j_2, \ldots, j_{n - k}$ 列展开。这 $n - k$ 列元素组成的 $n - k$ 阶子式只有一个不为 $0$,即选出第 $j_1, \ldots, j_{n - k}$ 行的那一个不为 $0$,还可知这个 $n - k$ 阶子式就是 $|\lambda I_{n - k}| = \lambda^{n - k}$。而剩下的 $k$ 阶子式对应的代数余子式为:
	$$
	(-1)^{(j_1 + \cdots + j_k) + (j_1 + \cdots + j_k)} (-A) \begin{pmatrix} j'_1, \ldots, j'_k \\ j'_1, \ldots, j'_k \end{pmatrix} = (-1)^k A \begin{pmatrix} j'_1, \ldots, j'_k \\ j'_1, \ldots, j'_k \end{pmatrix}
	$$

	于是第三类行列式的值为:$(-1)^k A \begin{pmatrix} j'_1, \ldots, j'_k \\ j'_1, \ldots, j'_k \end{pmatrix} \lambda^{n - k}$。特别地,当 $k = 1$ 时,得 $\lambda^{n - 1}$ 的系数为 $-\operatorname{tr}(A)$。综上,可以写出 $\lambda^{n - k}$ 的系数,即可得到 $A$ 的特征多项式的系数表示形式。
	$$
	\lambda^n - \operatorname{tr}(A) + \cdots + (-1)^k \sum_{1 \le j'_1 < \cdots < j'_k \le n} A \begin{pmatrix} j'_1, \ldots, j'_k \\ j'_1, \ldots, j'_k \end{pmatrix} \lambda^{n - k} + \cdots + (-1)^n |A|
	$$
\end{proof}

\subsection{利用特征多项式进行矩阵对角化}

要得到与矩阵 $A$ 相似的对角矩阵,我们需要找出 $A$ 的特征值 $\lambda_i$ 和特征向量 $\vec \alpha_i$。有了特征多项式,我们只需要找出 $A$ 的特征值 $\lambda_i$,并利用 $\vec \alpha_i$ 是齐次线性方程组 $(\lambda_i I - A) \vec x = \vec 0$ 的解求出 $\vec \alpha_i$(并验证 $\vec \alpha_i$ 是否线性无关)。于是可以将过程分成三步:
\begin{enumerate}
	\item 解多项式的零点 $|\lambda I_n - A| = 0$,得到 $n$ 个解 $\lambda_1, \ldots, \lambda_n$(包括重解)。
	\item 对于每个 $\lambda_i$,解齐次线性方程组 $(\lambda_i I - A) \vec x = \vec 0$。将 $n$ 个非零解合在一起构成列向量组。
	\item 检查列向量组 $\vec \alpha_1, \ldots, \vec \alpha_n$ 是否线性无关。
\end{enumerate}

如果以上步骤能顺利进行,那么 $\lambda_i$ 即为 $A$ 的一个特征值,$\vec \alpha_j$ 为 $A$ 属于 $\lambda_i$ 的一个特征向量。

\bigskip

下面以斐波那契数列 $a_n = a_{n - 1} + a_{n - 2}$ 为例,展示矩阵对角化的分析方法。

\begin{solve}
	用矩阵表示斐波那契数列的递推公式为:
	$$
	\begin{bmatrix} a_n \\ a_{n - 1}\end{bmatrix}
	=
	\begin{bmatrix} 1 & 1 \\ 1 & 0 \end{bmatrix}
	\begin{bmatrix} a_{n - 1} \\ a_{n - 2} \end{bmatrix}
	$$

	设 $A = \begin{bmatrix} 1 & 1 \\ 1 & 0 \end{bmatrix}$,下面对 $A$ 进行对角化。

	\begin{enumerate}
		\item 解方程 $|\lambda I - A| = 0$。
		$$
		\begin{vmatrix}\lambda I - A\end{vmatrix} =
		\begin{vmatrix} \lambda - 1 & -1 \\ -1 & \lambda \end{vmatrix} =
		\lambda^2 - \lambda - 1 = 0
		$$

		解得 $\lambda_1 = \dfrac{1 - \sqrt 5}{2}, \lambda_2 = \dfrac{1 + \sqrt 5}{2}$。

		\item 分别解关于 $\vec X$ 的方程组 $A \vec X = \lambda_1 \vec X$,$A \vec X = \lambda_2 \vec X$。

		分别移项,得 $(\lambda_1 I - A) \vec X = \vec 0$,$(\lambda_2 I - A) \vec X = \vec 0$。注意到,$\lambda_1 I - A = \begin{bmatrix} \lambda_1 - 1 & -1 \\ -1 & \lambda_1 \end{bmatrix}$,$\lambda_2 I - A = \begin{bmatrix} \lambda_2 - 1 & -1 \\ -1 & \lambda_2 \end{bmatrix}$,它们的行列式都为 $0$,而它们都不是零矩阵,故它们的解空间的维数为 $2 - 1 = 1$。于是,设 $\vec X_1$ 为 $(\lambda_1 I - A) \vec X = \vec 0$ 的基础解系,$\vec X_2$ 为 $(\lambda_2 I - A) \vec X = \vec 0$ 的基础解系。此处省略计算结果。

		\item 验证 $\vec X_1$ 和 $\vec X_2$ 线性无关。此处省略计算过程。
	\end{enumerate}

	最终可知,$A$ 可对角化,且满足:
	$$
	A \begin{bmatrix} \vec X_1 & \vec X_2 \end{bmatrix} = \begin{bmatrix} \vec X_1 & \vec X_2 \end{bmatrix} \begin{bmatrix} \dfrac{1 + \sqrt 5}{2} & 0 \\ 0 & \dfrac{1 - \sqrt 5}{2} \end{bmatrix}
	$$

	注意其中的 $\vec X_1, \vec X_2$ 是求解出的两个常向量。
\end{solve}