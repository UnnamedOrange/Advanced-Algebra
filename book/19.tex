% Licensed under the Creative Commons Attribution Share Alike 4.0 International.
% See the LICENCE file in the repository root for full licence text.

\section{正交矩阵和欧几里得空间的概念}

\subsection{正交矩阵的代数描述}

\begin{definition}{正交矩阵}
	如果\textbf{实数域}上的 $n$ 级矩阵 $A$ 满足 $A'A= I$,那么称 $A$ 是\emph{正交矩阵}。
\end{definition}

\begin{theorem}[正交矩阵的等价命题]
	设矩阵 $A$ 是实数域上的正价矩阵,则以下四个命题等价。
	\begin{enumerate}
		\item $A$ 是正交矩阵。
		\item $A'A = I$。
		\item $A$ 可逆,且 $A^{-1} = A'$。
		\item $AA' = I$。
	\end{enumerate}
\end{theorem}

正交矩阵具有以下性质。

\begin{theorem}
	$I$ 是正交矩阵。
\end{theorem}

\begin{theorem}
	若 $A$ 和 $B$ 都是 $n$ 级正交矩阵,则 $AB$ 也是正交矩阵。
\end{theorem}

\begin{proof}
	用定义验证。若 $A, B$ 都是 $n$ 级正交矩阵,则:
	$$
	(AB)(AB)' = A(BB')A' = AIA' = I
	$$

	即 $AB$ 也是正交矩阵。
\end{proof}

\begin{theorem}
	若 $A$ 是正交矩阵,则 $A^{-1}$(即 $A'$)也是正交矩阵。
\end{theorem}

\begin{theorem}
	若 $A$ 是正交矩阵,则 $|A| = 1$ 或 $-1$。
\end{theorem}

\begin{proof}
	若 $A$ 是正交矩阵,则 $|AA'| = |I|$,从而 $|A|^2 = 1$,因此 $|A| = \pm 1$。
\end{proof}

根据定义式 $A'A = I$,考察 $A'A$ 的每一个元素,我们可以得到以下重要结论。

\begin{theorem}[正交矩阵的充要条件]
	设实数域上 $n$ 级矩阵 $A$ 的行向量组为 $\vec \gamma_1, \vec \gamma_2, \ldots, \vec \gamma_n$,列向量组为 $\vec \alpha_1, \vec \alpha_2, \ldots, \vec \alpha_n$,则:
	\begin{enumerate}
		\item $A$ 为正交矩阵当且仅当 $A$ 的行向量组满足:
		$$
		\vec \gamma_i \vec \gamma'_j =
		\begin{cases}
			\begin{bmatrix} 1 \end{bmatrix}, & i = j
			\\
			\begin{bmatrix} 0 \end{bmatrix}, & i \ne j
		\end{cases}
		$$
		\item $A$ 为正交矩阵当且仅当 $A$ 的列向量组满足:
		$$
		\vec \alpha'_i \vec \alpha_j =
		\begin{cases}
			\begin{bmatrix} 1 \end{bmatrix}, & i = j
			\\
			\begin{bmatrix} 0 \end{bmatrix}, & i \ne j
		\end{cases}
		$$
	\end{enumerate}
\end{theorem}

以上定理中出现了只有一个元素的矩阵,我们可以直接将它们简记为一个数。另外,为了方便,我们提出以下记号。

\begin{definition}{Kronecker 记号}
	定义 \emph{Kronecker 记号}为:
	$$
	\delta_{ij} =
	\begin{cases}
			1, & i = j
			\\
			0, & i \ne j
	\end{cases}
	$$
\end{definition}

则正交矩阵的充要条件可简记为:
$$
\vec \gamma_i \vec \gamma'_j = \delta_{ij} \pod{1 \le i, j \le n}
$$$$
\vec \alpha'_i \vec \alpha_j = \delta_{ij} \pod{1 \le i, j \le n}
$$

(实数域上的)正交矩阵有何深刻的意义?要回答这一问题,需要研究欧几里得空间。

\subsection{欧几里得空间}

\begin{definition}{内积,标准内积}
	在 $\R^n$ 中,任给 $\vec \alpha = (a_1, \ldots, a_n)$,$\vec \beta = (b_1, \ldots, b_n)$,规定:
	$$
	(\vec \alpha, \vec \beta) = a_1 b_1 + \cdots + a_n b_n
	$$

	这个二元实值函数 $(\vec \alpha, \vec \beta)$ 称为 $\R^n$ 的一个\emph{内积},并且通常称它为\emph{标准内积}。可以把上式简写为:
	$$
	(\vec \alpha, \vec \beta) = \vec \alpha \vec \beta'
	$$

	如果 $\vec \alpha, \vec \beta$ 是列向量,那么标准内积可记为 $(\vec \alpha, \vec \beta) = \vec \alpha' \vec \beta$。
\end{definition}

根据以上定义,可以验证 $\R^n$ 的标准内积具有下列性质。

\begin{theorem}[标准内积的性质]
	\begin{enumerate}
		\item 对称性:
		$$
		(\vec \alpha, \vec \beta) = (\vec \beta, \vec \alpha)
		$$
		\item 线性性:
		$$
		(\vec \alpha + \vec \gamma, \vec \beta) = (\vec \alpha, \vec \beta) + (\vec \gamma, \vec \beta)
		$$$$
		(k \vec \alpha, \vec \beta) = k(\vec \alpha, \vec \beta)
		$$
		\item 正定性:$(\vec \alpha, \vec \alpha) \ge 0$,等号成立当且仅当 $\vec \alpha = \vec 0$。
	\end{enumerate}
\end{theorem}

所谓欧几里得空间,就是实数域上的(标准)内积空间。

\begin{definition}{欧几里得空间}
	$n$ 维向量空间 $\R^n$ 加上标准内积,就称 $\R^n$ 为\emph{欧几里得空间}。
\end{definition}

内积空间中存在长度这一重要概念。

\begin{definition}{长度}
	在欧几里得空间 $\R^n$ 中,规定向量 $\vec \alpha$ 的\emph{长度}为:
	$$
	|\vec \alpha| \triangleq \sqrt{(\vec \alpha, \vec \alpha)}
	$$
\end{definition}

在欧几里得空间中,可以定义两向量的余弦值。

\begin{definition}{两向量的余弦值}
	规定欧几里得空间中两向量 $\vec \alpha, \vec \beta$ 的余弦值为:
	$$
	\cos \langle \vec \alpha, \vec \beta \rangle \triangleq \dfrac{(\vec \alpha, \vec \beta)}{|\vec \alpha| |\vec \beta|}
	$$
\end{definition}

要使得两向量余弦值的定义式有意义,需要证明柯西不等式。

\begin{theorem}[柯西不等式]
	对于 $\R^n$ 上的向量 $\vec \alpha = (a_1, \ldots, a_n)$,$\vec \beta = (b_1, \ldots, b_n)$,总有:
	$$
	\biggl( \sum\limits_{i = 1}^n a_i^2 \biggr) \biggl( \sum\limits_{i = 1}^n b_i^2 \biggr) \ge \biggl( \sum\limits_{i = 1}^n a_i b_i \biggr)^2
	$$
\end{theorem}

\begin{proof}
	构造如下矩阵乘法:
	$$
	\begin{bmatrix}
		a_1 & a_2 & \cdots & a_n
		\\
		b_1 & b_2 & \cdots & b_n
	\end{bmatrix}
	\begin{bmatrix}
		a_1 & b_1
		\\
		a_2 & b_2
		\\
		\vdots & \vdots
		\\
		a_n & b_n
	\end{bmatrix}
	=
	\begin{bmatrix}
		\sum\limits_{i = 1}^n a_i^2 & \sum\limits_{i = 1}^n a_i b_i
		\\
		\sum\limits_{i = 1}^n a_i b_i & \sum\limits_{i = 1}^n b_i^2
	\end{bmatrix}
	$$

	右侧矩阵的行列式为:
	$$
	\biggl( \sum\limits_{i = 1}^n a_i^2 \biggr) \biggl( \sum\limits_{i = 1}^n b_i^2 \biggr) - \biggl( \sum\limits_{i = 1}^n a_i b_i \biggr)^2
	$$

	等式左侧由比内-柯西公式得其行列式为:
	$$
	\sum\limits_{1 \le i < j \le n} \begin{vmatrix} a_i & a_j \\ b_i & b_j \end{vmatrix} \begin{vmatrix} a_i & b_i \\ a_j & b_j \end{vmatrix} = (a_i b_j - a_j b_i)(a_i b_j - a_j b_i) \ge 0
	$$

	因此:
	$$
	\biggl( \sum\limits_{i = 1}^n a_i^2 \biggr) \biggl( \sum\limits_{i = 1}^n b_i^2 \biggr) - \biggl( \sum\limits_{i = 1}^n a_i b_i \biggr)^2 \ge 0
	$$
\end{proof}

\subsection{向量内积的矩阵乘法表示}

设 $\R^n$ 的一组基为 $\vec \alpha_1, \vec \alpha_2, \ldots, \vec \alpha_n$,构造矩阵 $A(a_{ij})$:
$$
a_{ij} = (\vec \alpha_i, \vec \alpha_j) =
\begin{cases}
	(\vec \alpha_i, \vec \alpha_i), & i = j
	\\
	(\vec \alpha_i, \vec \alpha_j), & i \ne j
\end{cases}
$$

则对 $\R^n$ 中的任意两个列向量 $\vec \alpha, \vec \beta$ 做内积有:
$$
\begin{aligned}
	(\vec \alpha, \vec \beta) &= \biggl( \sum\limits_{i = 1}^n x_i \vec \alpha_i, \sum\limits_{j = 1}^n y_j \vec \alpha_j \biggr)
	\\&=
	\sum\limits_{i = 1}^n \sum\limits_{j = 1}^n x_i y_j a_{ij}
	\\&=
	\sum\limits_{j = 1}^n y_j \sum\limits_{i = 1}^n x_i a_{ij}
	\\&=
	\begin{bmatrix} x_1 & x_2 & \cdots & x_n \end{bmatrix} A \begin{bmatrix} y_1 \\ y_2 \\ \vdots \\ y_n \end{bmatrix}
\end{aligned}
$$

当 $\vec \alpha_1 = \vec e_1, \ldots, \vec \alpha_n = \vec e_n$ 时,$A = I_n$,此时有 $\forall \vec \alpha, \vec \beta \in \R^n, (\vec \alpha, \vec \beta) = \vec \alpha^T \vec \beta$。可见,标准内积的“标准”在于 $A$ 的取值为单位矩阵。

之后我们将会看到,当 $A$ 取任意 $n$ 级满秩方阵,乃至任意 $n$ 级方阵时,上式会具有更普遍的含义。

\subsection{欧几里得空间的几何性质}

在定义欧几里得空间后,我们定义向量的正交。

\begin{definition}{正交的向量}
	在欧几里得空间 $\R^n$ 中,如果 $(\vec \alpha, \vec \beta) = 0$,那么称 $\vec \alpha$ 与 $\vec \beta$ 是\emph{正交}的,记作 $\vec \alpha \perp \vec \beta$。显然零向量与任何向量正交。
\end{definition}

\begin{definition}{正交向量组}
	在欧几里得空间 $\R^n$ 中,由非零向量组成的向量组如果其中每两个不同的向量都正交,那么称它们为\emph{正交向量组}。仅由一个非零向量组成的向量组也是正交向量组。
\end{definition}

\begin{definition}{单位向量,单位化}
	长度为 $1$ 的向量称为\emph{单位向量}。显然 $\vec \alpha$ 是单位向量的充分必要条件为 $(\vec \alpha, \vec \alpha) = 1$。把非零向量 $\vec \alpha$ 乘以 $\frac{1}{|\vec \alpha|}$ 称为把 $\vec \alpha$ \emph{单位化}。
\end{definition}

\begin{definition}{正交单位向量组}
	如果正交向量组的每个向量都是单位向量,那么称它为\emph{正交单位向量组}。
\end{definition}

欧几里得空间满足以下几何性质。

\begin{theorem}[三角形不等式]
	$\forall \vec \alpha, \vec \beta \in \R^n$,有 $|\vec \alpha + \vec \beta| \le |\vec \alpha| + |\vec \beta|$。
\end{theorem}

\begin{proof}
	$$
	\begin{aligned}
		|\vec \alpha + \vec \beta|^2 &= (\vec \alpha + \vec \beta, \vec \alpha + \vec \beta)
		\\&=
		|\vec \alpha|^2 + 2 (\vec \alpha, \vec \beta) + |\beta|^2
		\\&\le
		|\vec \alpha|^2 + 2|\vec \alpha||\vec \beta| + |\vec \beta|^2
		\\&=
		(|\vec \alpha| + |\vec \beta|)^2
	\end{aligned}
	$$
\end{proof}

\begin{theorem}[勾股定理]
	$\forall \vec \alpha \perp \vec \beta \pod{\vec \alpha, \vec \beta \in \R^n}$,有 $|\vec \alpha + \vec \beta|^2 = |\vec \alpha|^2 + |\vec \beta|^2$。
\end{theorem}

\begin{proof}
	$$
	|\vec \alpha + \vec \beta|^2 = (\vec \alpha + \vec \beta, \vec \alpha + \vec \beta) = |\vec \alpha|^2 + |\vec \beta|^2
	$$
\end{proof}