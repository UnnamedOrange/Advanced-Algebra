% Licensed under the Creative Commons Attribution Share Alike 4.0 International.
% See the LICENCE file in the repository root for full licence text.

\section{齐次线性方程组的解集的结构}

\begin{definition}{解向量}
	数域 $\mathbb K$ 上的 $n$ 元齐次线性方程组 $x_1 \vec \alpha_1 + x_2 \vec \alpha_2 + \cdots + x_n \vec \alpha_n = \vec 0$ 的一个解是 $\mathbb K^n$ 中的一个向量,称它为齐次线性方程组的一个\emph{解向量}。
\end{definition}

用解向量的观点来考察方程组的解集,可知 $n$ 元齐次线性方程组的解集 $W$ 是 $\mathbb K^n$ 的一个非空子集。本节的目的便是研究 $W$ 的结构。

注意到有:
\begin{enumerate}
	\item 若 $\vec \gamma, \vec \delta \in W$,则 $\vec \gamma + \vec \delta \in W$。
	\item 若 $\vec \gamma \in W, k \in \mathbb K$,则 $k \vec \gamma \in W$。
\end{enumerate}

所以 $W$ 是 $\mathbb K^n$ 的一个子空间,我们将其称为解空间。

\begin{definition}{解空间}
	齐次线性方程组的解集 $W$ 是 $\mathbb K^n$ 的一个子空间,称它为方程组的\emph{解空间}。
\end{definition}

要进一步研究解空间的结构,自然想到要研究解空间的基。为此,我们给出以下概念。

\begin{definition}{基础解系}
	齐次线性方程组有非零解时,如果它的有限多个解 $\vec \eta_1, \vec \eta_2, \ldots, \vec \eta_t$ 满足:
	\begin{enumerate}
		\item $\vec \eta_1, \vec \eta_2, \ldots, \vec \eta_t$ 线性无关。
		\item 齐次线性方程组的每一个解都可以由 $\vec \eta_1, \vec \eta_2, \ldots, \vec \eta_t$ 线性表出。
	\end{enumerate}
	则称 $\vec \eta_1, \vec \eta_2, \ldots, \vec \eta_t$ 是齐次线性方程组的一个\emph{基础解系}。
\end{definition}

基础解系的定义与基的定义相同,只是把研究对象限定为解空间了。

如果已知齐次线性方程组的一个基础解系 $\vec \eta_1, \vec \eta_2, \ldots, \vec \eta_t$,那么我们其实已经知道齐次线性方程组的解集 $W$。
$$
W = \langle \vec \eta_1, \vec \eta_2, \ldots, \vec \eta_t \rangle
$$

所以我们研究如何得到基础解系即可。利用构造法,我们可以得出以下定理。

\begin{theorem}
	数域 $\mathbb K$ 上的 $n$ 元齐次线性方程组的解空间 $W$ 的维数为 $\dim W = n - \operatorname{rank}(A)$,其中 $A$ 是方程组的系数矩阵。从而当齐次线性方程组有非零解时,它的每个基础解系所含解向量的个数都为 $n - \operatorname{rank}(A)$。
\end{theorem}

\begin{proof}[构造法]
	若 $\operatorname{rank}(A) = n$,则 $W = \{\vec 0\}$,显然成立。下面设 $\operatorname{rank}(A) = r < n$。

	对 $A$ 进行初等行变换,化成简化行阶梯形矩阵 $J$。可知 $J$ 有 $r$ 个主元,不妨设它们分别在第 $1, 2, \ldots, r$ 列,于是齐次线性方程组的一般解为:
	$$
	\begin{cases}
		x_1 = -b_{1, r + 1} x_{r + 1} - \cdots - b_{1n} x_n
		\\
		x_2 = -b_{2, r + 1} x_{r + 1} - \cdots - b_{2n} x_n
		\\
		\vdots
		\\
		x_r = -b_{r, r + 1} x_{r + 1} - \cdots - b_{rn} x_n
	\end{cases}
	$$

	让自由未知量 $x_{r + 1}, \ldots, x_n$ 分别取下述 $n - r$ 组数:
	$$
	\begin{pmatrix}
		1 \\ 0 \\ \vdots \\ 0
	\end{pmatrix},
	\begin{pmatrix}
		0 \\ 1 \\ \vdots \\ 0
	\end{pmatrix},
	\ldots,
	\begin{pmatrix}
		0 \\ 0 \\ \vdots \\ 1
	\end{pmatrix}
	$$

	则由一般解的公式可得方程组的 $n - r$ 个解:
	$$
	\vec \eta_1 =
	\begin{pmatrix}
		-b_{1, r + 1} \\ -b_{2, r + 1} \\ \vdots \\ -b_{r, r + 1} \\ 1 \\ 0 \\ \vdots \\ 0
	\end{pmatrix},
	\ldots,
	\vec \eta_{n - r} =
	\begin{pmatrix}
		-b_{1n} \\ -b_{2n} \\ \vdots \\ -b_{rn} \\ 0 \\ 0 \\ \vdots \\ 1
	\end{pmatrix}
	$$

	显然 $\vec \eta_1, \vec \eta_2, \ldots, \vec \eta_{n - r}$ 线性无关。而对于齐次线性方程组的任意一组解,都显然可以由 $\vec \eta_1, \vec \eta_2, \ldots, \vec \eta_{n - r}$ 线性表出,从而 $\vec \eta_1, \vec \eta_2, \ldots, \vec \eta_{n - r}$ 是方程组的一个基础解系,它包含的解向量的个数为 $n - \operatorname{rank}(A)$,于是 $\dim W = n - \operatorname{rank}(A)$。证毕。
\end{proof}

非齐次线性方程组的解集的结构又如何?设其对应的齐次线性方程组(常数项从 $\vec \beta$ 改为 $\vec 0$,称这个齐次线性方程组为原方程组的\emph{导出组}\index{导出组})的解空间为 $W$,事实上,非齐次线性方程组的解集是一个 $W$ 型的\emph{线性流形}\index{线性流形}。相关的内容请自行查阅,本笔记不予讨论。