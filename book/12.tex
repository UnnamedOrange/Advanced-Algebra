% Licensed under the Creative Commons Attribution Share Alike 4.0 International.
% See the LICENCE file in the repository root for full licence text.

\chapter{矩阵的运算}

\section{矩阵的加法和数量乘法}

将矩阵看作一个集合中的元素,自然想到可以为矩阵定义运算。为了书写关于矩阵的等式,我们首先定义矩阵的相等。

\begin{definition}{相等}
	设数域 $\mathbb K$ 上的两个矩阵分别为 $A$ 和 $B$。如果它们的行数相等,列数也相等,并且它们的所有元素对应相等(即第一个矩阵的 $(i, j)$ 元等于第二个矩阵的 $(i, j)$ 元),那么称 $A$ 与 $B$ \emph{相等},记作 $A = B$。
\end{definition}

类似于向量,我们可以很自然地想到定义矩阵的加法和数量乘法。

\begin{definition}{和}
	设 $A = (a_{ij})$ 和 $B = (b_{ij})$ 都是数域 $\mathbb K$ 上的 $s \times n$ 矩阵。令 $C = (a_{ij} + b_{ij})_{s \times n}$,则称 $C$ 是矩阵 $A$ 与 $B$ 的\emph{和},记作 $C = A + B$。
\end{definition}

\begin{definition}{数量乘积}
	设 $A = (a_{ij})$ 是数域 $\mathbb K$ 上的 $s \times n$ 矩阵,$k \in \mathbb K$。令 $M = (k a_{ij})_{s \times n}$,则称 $M$ 是 $k$ 与矩阵 $A$ 的\emph{数量乘积},记作 $M = k A$。
\end{definition}

类似于向量空间中加法的负元,我们为矩阵定义其负矩阵。

\begin{definition}{负矩阵}
	设 $A = (a_{ij})_{s \times n}$,则矩阵 $(-a_{ij})_{s \times n}$ 称为 $A$ 的\emph{负矩阵},记作 $-A$。
\end{definition}

容易验证,矩阵的加法与数量乘法满足类似于 $n$ 维向量的加法与数量乘法所满足的 $8$ 条运算法则。对于 $\mathbb K$ 上的 $s \times n$ 矩阵的运算,满足:
\begin{enumerate}
	\item 加法交换律。
	\item 加法结合律。
	\item 存在唯一零元(加法的单位元),称为零矩阵,记为 $\mathbf 0$。
	\item 任一矩阵存在负元(加法的逆元),即负矩阵。
	\item 数乘的单位元。
	\item 数乘的乘法结合律。
	\item 数的乘法分配律。
	\item 矩阵的乘法分配律。
\end{enumerate}

最后,利用负矩阵的概念,我们可以定义矩阵的减法。

\begin{definition}{差}
	设 $A, B$ 都是 $s \times n$ 矩阵,则矩阵减法 $A - B$ 定义为 $A + (-B)$,称为 $A$ 与 $B$ 的\emph{差}。
\end{definition}

\section{$\mathbb K^n$ 到 $\mathbb K^s$ 的线性映射}

矩阵还可以做什么运算?为了回答这一问题,需要研究 $\mathbb K^n$ 到 $\mathbb K^s$ 的线性映射。首先,我们在此明确提出映射及其相关概念。

\subsection{映射}

\begin{definition}{映射,象,原象}
	设 $S$ 和 $S'$ 是两个集合,如果存在一个对应法则 $f$,使得集合 $S$ 中的每一个元素 $a$,都有集合 $S'$ 中唯一确定的元素 $b$ 与它对应,那么称 $f$ 是集合 $S$ 到 $S'$ 的一个\emph{映射},记作:
	$$
	\begin{aligned}
		f \colon & S \to S'
		\\&
		a \mapsto b
	\end{aligned}
	$$

	其中 $b$ 称为 $a$ 在 $f$ 下的\emph{象},$a$ 称为 $b$ 在 $f$ 下的一个\emph{原象}。$a$ 在 $f$ 下的象用符号 $f(a)$ 表示,于是映射 $f$ 也可以记成 $f(a) = b \pod{a \in S}$。
\end{definition}

\begin{definition}{定义域,陪域,值域,映射的象}
	设 $f$ 是集合 $S$ 到集合 $S'$ 的一个映射,则把 $S$ 叫做映射 $f$ 的\emph{定义域},把 $S'$ 叫做 $f$ 的\emph{陪域}。$S$ 的所有元素在 $f$ 下的象组成的集合叫做 $f$ 的\emph{值域}或 $f$ 的\emph{象},记作 $f(S)$ 或 $\operatorname{Im} f$。容易看出,$f$ 的值域是 $f$ 的陪域的子集。
\end{definition}

\begin{definition}{满射}
	设 $f$ 是集合 $S$ 到集合 $S'$ 的一个映射,如果 $f(S) = S'$,那么称 $f$ 是\emph{满射}(或 $f$ 是 $S$ 到 $S'$ 上的映射)。可知,$f$ 是满射当且仅当 $f$ 的陪域中每一个元素都有至少一个原象。
\end{definition}

\begin{definition}{单射}
	如果映射 $f$ 的定义域 $S$ 中不同的元素的象也不同,那么称 $f$ 是\emph{单射}。可知,$f$ 是单射当且仅当从 $a_1, a_2 \in S$ 且 $f(a_1) = f(a_2)$ 可以推出 $a_1 = a_2$。
\end{definition}

\begin{definition}{双射,一一对应}
	如果映射 $f$ 既是单射,又是满射,那么称 $f$ 是\emph{双射}(或 $f$ 是 $S$ 到 $S'$ 的一个\emph{一一对应})。显然,$f$ 是双射当且仅当陪域中每一个元素都有唯一的一个原象。
\end{definition}

\begin{definition}{相等}
	 如果映射 $f$ 与映射 $g$ 的定义域相等,陪域相等,并且对应法则相同(即 $\forall x \in S$,有 $f(x) = g(x)$),则称映射 $f$ 与映射 $g$ \emph{相等}。
\end{definition}

下面提出一些常用的名词或概念。

\begin{definition}{变换}
	集合 $S$ 到自身的一个映射,通常称为 $S$ 上的一个\emph{变换}。
\end{definition}

\begin{definition}{函数}
	集合 $S$ 到数集(数域 $\mathbb K$ 的任一非空子集)的一个映射,通常称为 $S$ 上的一个\emph{函数}。
\end{definition}

\begin{definition}{原象集}
	陪域 $S'$ 中的元素 $b$ 在映射 $f$ 下的所有原象组成的集合称为 $b$ 在 $f$ 下的\emph{原象集},记作 $f^{-1}(b)$。
\end{definition}

\begin{definition}{恒等映射,恒等变换}
	如果映射 $f \colon S \to S$ 把 $S$ 中每一个元素对应到它自身,即 $\forall x \in S$,有 $f(x) = x$,那么称 $f$ 是\emph{恒等映射}(或 $S$ 上的\emph{恒等变换}),记作 $1_S$。
\end{definition}

\begin{definition}{乘积,合成}
	相继施行映射 $g: S \to S'$ 和 $f \colon S' \to S''$,得到 $S$ 到 $S''$ 的一个映射,称为 $f$ 与 $g$ 的\emph{乘积}(或\emph{合成}),记作 $fg$。即有 $(fg)(a) = f(g(a)) \pod{\forall a \in S}$。
\end{definition}

我们发现,在一个由映射组成的集合中,两映射之间可以进行比较和运算,还存在恒等映射的概念。所以我们接下来进一步研究这些概念与性质,并相应地提出逆映射等概念。

\begin{theorem}
	设 $h \colon S \to S$,$g \colon S' \to S''$,$f \colon S'' \to S'''$,则有 $f(gh) = (fg)h$,即映射的合成满足结合律。
\end{theorem}

需要注意,映射的合成不满足交换律。

\begin{theorem}
	对于任意一个映射 $f \colon S \to S'$,都有 $f 1_S = f = 1_{S'} f$。
\end{theorem}

\begin{definition}{可逆,逆映射}
	设 $f \colon S \to S'$,如果存在一个映射 $g \colon S' \to S$,使得 $fg = 1_{S'}$ 且 $gf = 1_S$,那么称映射 $f$ 是\emph{可逆}的,$g$ 是 $f$ 的一个\emph{逆映射}。
\end{definition}

\begin{theorem}
	如果映射 $f$ 是可逆的,那么它的逆映射是唯一的。
\end{theorem}

\begin{proof}
	设 $g$ 和 $h$ 都是 $f$ 的一个逆映射,则:
	$$
	h(fg) = h 1_{S'} = h
	$$

	由结合律又知:
	$$
	h(fg) = (hf)g = 1_S g = g
	$$

	所以 $h = g$。
\end{proof}

所以我们可以记 $f$ 的逆映射为 $f^{-1}$。

\begin{theorem}
	映射 $f \colon S \to S'$ 可逆的充分必要条件为 $f$ 是双射。
\end{theorem}

% TODO: 补充证明。

证明略。

\subsection{线性映射与矩阵乘法的定义}

\begin{definition}{线性映射}
	如果 $\mathbb K^n$ 到 $\mathbb K^s$ 的一个映射 $\sigma$ 保持加法和数量乘法,那么称 $\sigma$ 是 $\mathbb K^n$ 到 $\mathbb K^s$ 的一个\emph{线性映射}。所谓保持加法和数量乘法,是指:
	$$
	\forall \vec \alpha, \vec \beta \in \mathbb K^n, k \in \mathbb K
	$$
	都有:
	$$
	\sigma(\vec \alpha + \vec \beta) = \sigma(\vec \alpha) + \sigma(\vec \beta), \sigma(k \vec \alpha) = k \sigma (\vec \alpha)
	$$
\end{definition}

下面尝试从线性映射的概念出发定义矩阵乘法。设某一 $\mathbb K^n$ 到 $\mathbb K^s$ 的线性映射为 $\phi$。考察 $\mathbb K^n$ 的标准基 $\vec e_1, \ldots, \vec e_n$,如果我们已知:
$$
\phi(\vec e_1) = \vec \alpha_1, \ldots, \phi(\vec e_n) = \vec \alpha_n
$$
那么事实上我们可以求出 $\mathbb K^n$ 中的任一向量 $\vec x$ 在 $\phi$ 下的象。设 $\vec x = (x_1, \ldots, x_n)'$,则有:
$$
\begin{aligned}
	\phi(\vec x) &= \phi(x_1 \vec e_1 + \cdots + x_n \vec e_n)
	\\&=
	\phi(x_1 \vec e_1) + \phi(x_2 \vec e_2 + \cdots + x_n \vec e_n)
	\\&=
	x_1 \phi(\vec e_1) + \phi(x_2 \vec e_2 + \cdots + x_n \vec e_n)
	\\&=
	x_1 \vec \alpha_1 + \cdots + x_n \vec \alpha_n
\end{aligned}
$$

受 $\phi(x) = x_1 \vec \alpha_1 + \cdots + x_n \vec \alpha_n$ 的启发,我们定义矩阵的乘积如下。

\begin{definition}{矩阵的乘积}
	设 $A = (a_{ij})_{s \times n}, B = (b_{ij})_{n \times m}$,令 $C = (c_{ij})_{s \times m}$,其中:
	$$
	c_{ij} = \sum\limits_{k = 1}^n a_{ik} b_{kj} \pod{i = 1, \ldots, s; j = 1, \ldots, m}
	$$
	则矩阵 $C$ 称为\emph{矩阵 $A$ 与 $B$ 的乘积},记作 $C = AB$。
\end{definition}

如此定义后,记矩阵 $A = \begin{bmatrix} \vec \alpha_1 & \cdots & \vec \alpha_n \end{bmatrix}$($\vec \alpha_i \pod{i = 1, \ldots, n}$ 都是列向量),并把 $\vec x$ 和 $\phi(\vec x)$ 都看作 $n \times 1$ 的矩阵,则 $\phi(\vec x) = x_1 \vec \alpha_1 + \cdots + x_n \vec \alpha_n$ 便可用矩阵乘法记为 $\phi(\vec x) = A \vec x$。

\subsection{矩阵乘法与线性映射的关系}

以上对矩阵乘法的定义只是一个受启发后的产物。矩阵乘法与有限维向量空间的线性映射还有什么更深刻的关系?首先需要知道矩阵与线性映射的一一对应关系。