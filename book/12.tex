% Licensed under the Creative Commons Attribution Share Alike 4.0 International.
% See the LICENCE file in the repository root for full licence text.

\chapter{矩阵的运算}

\section{矩阵的加法和数量乘法}

将矩阵看作一个集合中的元素,自然想到可以为矩阵定义运算。为了书写关于矩阵的等式,我们首先定义矩阵的相等。

\begin{definition}{相等}
	设数域 $\mathbb K$ 上的两个矩阵分别为 $A$ 和 $B$。如果它们的行数相等,列数也相等,并且它们的所有元素对应相等(即第一个矩阵的 $(i, j)$ 元等于第二个矩阵的 $(i, j)$ 元),那么称 $A$ 与 $B$ \emph{相等},记作 $A = B$。
\end{definition}

类似于向量,我们可以很自然地想到定义矩阵的加法和数量乘法。

\begin{definition}{和}
	设 $A = (a_{ij})$ 和 $B = (b_{ij})$ 都是数域 $\mathbb K$ 上的 $s \times n$ 矩阵。令 $C = (a_{ij} + b_{ij})_{s \times n}$,则称 $C$ 是矩阵 $A$ 与 $B$ 的\emph{和},记作 $C = A + B$。
\end{definition}

\begin{definition}{数量乘积}
	设 $A = (a_{ij})$ 是数域 $\mathbb K$ 上的 $s \times n$ 矩阵,$k \in \mathbb K$。令 $M = (k a_{ij})_{s \times n}$,则称 $M$ 是 $k$ 与矩阵 $A$ 的\emph{数量乘积},记作 $M = k A$。
\end{definition}

类似于向量空间中加法的负元,我们为矩阵定义其负矩阵。

\begin{definition}{负矩阵}
	设 $A = (a_{ij})_{s \times n}$,则矩阵 $(-a_{ij})_{s \times n}$ 称为 $A$ 的\emph{负矩阵},记作 $-A$。
\end{definition}

容易验证,矩阵的加法与数量乘法满足类似于 $n$ 维向量的加法与数量乘法所满足的 $8$ 条运算法则。对于 $\mathbb K$ 上的 $s \times n$ 矩阵的运算,满足:
\begin{enumerate}
	\item 加法交换律。
	\item 加法结合律。
	\item 存在唯一零元(加法的单位元),称为零矩阵,记为 $\mathbf 0$。
	\item 任一矩阵存在负元(加法的逆元),即负矩阵。
	\item 数乘的单位元。
	\item 数乘的乘法结合律。
	\item 数的乘法分配律。
	\item 矩阵的乘法分配律。
\end{enumerate}

最后,利用负矩阵的概念,我们可以定义矩阵的减法。

\begin{definition}{差}
	设 $A, B$ 都是 $s \times n$ 矩阵,则矩阵减法 $A - B$ 定义为 $A + (-B)$,称为 $A$ 与 $B$ 的\emph{差}。
\end{definition}

\section{$\mathbb K^n$ 到 $\mathbb K^s$ 的线性映射}

矩阵还可以做什么运算?为了回答这一问题,需要研究 $\mathbb K^n$ 到 $\mathbb K^s$ 的线性映射。首先,我们在此明确提出映射及其相关概念。

\subsection{映射}

\begin{definition}{映射,象,原象}
	设 $S$ 和 $S'$ 是两个集合,如果存在一个对应法则 $f$,使得集合 $S$ 中的每一个元素 $a$,都有集合 $S'$ 中唯一确定的元素 $b$ 与它对应,那么称 $f$ 是集合 $S$ 到 $S'$ 的一个\emph{映射},记作:
	$$
	\begin{aligned}
		f \colon & S \to S'
		\\&
		a \mapsto b
	\end{aligned}
	$$

	其中 $b$ 称为 $a$ 在 $f$ 下的\emph{象},$a$ 称为 $b$ 在 $f$ 下的一个\emph{原象}。$a$ 在 $f$ 下的象用符号 $f(a)$ 表示,于是映射 $f$ 也可以记成 $f(a) = b \pod{a \in S}$。
\end{definition}

\begin{definition}{定义域,陪域,值域,映射的象}
	设 $f$ 是集合 $S$ 到集合 $S'$ 的一个映射,则把 $S$ 叫做映射 $f$ 的\emph{定义域},把 $S'$ 叫做 $f$ 的\emph{陪域}。$S$ 的所有元素在 $f$ 下的象组成的集合叫做 $f$ 的\emph{值域}或 $f$ 的\emph{象},记作 $f(S)$ 或 $\operatorname{Im} f$。容易看出,$f$ 的值域是 $f$ 的陪域的子集。
\end{definition}

\begin{definition}{满射}
	设 $f$ 是集合 $S$ 到集合 $S'$ 的一个映射,如果 $f(S) = S'$,那么称 $f$ 是\emph{满射}(或 $f$ 是 $S$ 到 $S'$ 上的映射)。可知,$f$ 是满射当且仅当 $f$ 的陪域中每一个元素都有至少一个原象。
\end{definition}

\begin{definition}{单射}
	如果映射 $f$ 的定义域 $S$ 中不同的元素的象也不同,那么称 $f$ 是\emph{单射}。可知,$f$ 是单射当且仅当从 $a_1, a_2 \in S$ 且 $f(a_1) = f(a_2)$ 可以推出 $a_1 = a_2$。
\end{definition}

\begin{definition}{双射,一一对应}
	如果映射 $f$ 既是单射,又是满射,那么称 $f$ 是\emph{双射}(或 $f$ 是 $S$ 到 $S'$ 的一个\emph{一一对应})。显然,$f$ 是双射当且仅当陪域中每一个元素都有唯一的一个原象。
\end{definition}

\begin{definition}{相等}
	 如果映射 $f$ 与映射 $g$ 的定义域相等,陪域相等,并且对应法则相同(即 $\forall x \in S$,有 $f(x) = g(x)$),则称映射 $f$ 与映射 $g$ \emph{相等}。
\end{definition}

下面提出一些常用的名词或概念。

\begin{definition}{变换}
	集合 $S$ 到自身的一个映射,通常称为 $S$ 上的一个\emph{变换}。
\end{definition}

\begin{definition}{函数}
	集合 $S$ 到数集(数域 $\mathbb K$ 的任一非空子集)的一个映射,通常称为 $S$ 上的一个\emph{函数}。
\end{definition}

\begin{definition}{原象集}
	陪域 $S'$ 中的元素 $b$ 在映射 $f$ 下的所有原象组成的集合称为 $b$ 在 $f$ 下的\emph{原象集},记作 $f^{-1}(b)$。
\end{definition}

\begin{definition}{恒等映射,恒等变换}
	如果映射 $f \colon S \to S$ 把 $S$ 中每一个元素对应到它自身,即 $\forall x \in S$,有 $f(x) = x$,那么称 $f$ 是\emph{恒等映射}(或 $S$ 上的\emph{恒等变换}),记作 $1_S$。
\end{definition}

\begin{definition}{乘积,合成}
	相继施行映射 $g: S \to S'$ 和 $f \colon S' \to S''$,得到 $S$ 到 $S''$ 的一个映射,称为 $f$ 与 $g$ 的\emph{乘积}(或\emph{合成}),记作 $fg$。即有 $(fg)(a) = f(g(a)) \pod{\forall a \in S}$。
\end{definition}

我们发现,在一个由映射组成的集合中,两映射之间可以进行比较和运算,还存在恒等映射的概念。所以我们接下来进一步研究这些概念与性质,并相应地提出逆映射等概念。