% Licensed under the Creative Commons Attribution Share Alike 4.0 International.
% See the LICENCE file in the repository root for full licence text.

\section{可逆矩阵}

矩阵乘法是否具有逆?为了研究这个问题,我们需要先定义相关概念,并给出一系列显然的定理。

\begin{definition}{可逆矩阵,非奇异矩阵}
	对于数域 $\mathbb K$ 上的矩阵 $A$,如果存在数域 $\mathbb K$ 上的矩阵 $B$,使得 $AB = BA = I$,那么称 $A$ 是\emph{可逆矩阵}(或\emph{非奇异矩阵})。
\end{definition}

根据定义,\textbf{可逆矩阵一定是方阵}。

\begin{definition}{逆矩阵}
	如果 $A$ 是可逆矩阵,那么适合 $AB = BA = I$ 的矩阵 $B$ 称为 $A$ 的\emph{逆矩阵},记作 $A^{-1}$。
\end{definition}

\begin{theorem}
	逆矩阵如果存在,那么它是唯一的。
\end{theorem}

\begin{proof}
	若 $A$ 是可逆矩阵,对于 $A$ 的任意两个逆矩阵 $B, C$,有:
	$$
	B = BI = BAC = IC = C
	$$
\end{proof}

如果 $A$ 是可逆矩阵,那么根据 $|AB| = |A| |B| = |B| |A| = |BA|$,可知 $|A| \ne 0$ 是其逆矩阵存在的必要条件,因为:
$$
|A A^{-1}| = |I_n| = 1 = |A| |A^{-1}|
$$

我们自然地想知道,$|A| \ne 0$ 是可逆的充分必要条件吗?为了研究这个问题,我们需要借助下文的初等矩阵。

\subsection{初等矩阵}

初等矩阵概念的提出源于以下定理。

\begin{theorem}
	初等行变换都是线性变换。
\end{theorem}

\begin{proof}
	只需找到初等行变换对应的矩阵,因为矩阵和线性映射有着一一对应关系。
	\begin{enumerate}
		\item 对于交换第 $i, j$ 两行(不妨设 $i < j$),该矩阵为:
		$$
		P(i, j) =
		\begin{bmatrix}
			\vec e_1 \\ \vdots \\ \vec e_j \\ \vdots \\ \vec e_i \\ \vdots \\ \vec e_n
		\end{bmatrix}
		$$

		可以验证,$P(i, j) A$ 等于将 $A$ 的第 $i, j$ 两行交换后得到的矩阵。

		\item 对于第 $i$ 行乘以 $c$,该矩阵为:
		$$
		P(i(c)) =
		\begin{bmatrix}
			\vec e_1 \\ \vdots \\ c \vec e_i \\ \vdots \\ \vec e_n
		\end{bmatrix}
		$$

		可以验证,$P(i(c)) A$ 等于将 $A$ 的第 $i$ 行乘以 $c$ 后得到的矩阵。

		\item 对于第 $i$ 行加上第 $j$ 行的 $k$ 倍,该矩阵为:
		$$
		P(i, j(c)) =
		\begin{bmatrix}
			\vec e_1 \\ \vdots \\ \vec e_i + k \vec e_j \\ \vdots \\ \vec e_n
		\end{bmatrix}
		$$

		可以验证,$P(i, j(k)) A$ 等于将 $A$ 的第 $i$ 行加上第 $j$ 行的 $k$ 倍得到的矩阵。
	\end{enumerate}
\end{proof}

以上证明过程反映了两个重要信息:
\begin{enumerate}
	\item 代入 $A = I$,则可知 $P(i, j), P(i(c)), P(i, j(k))$ 是单位矩阵经过一次初等行变换的结果。
	\item 观察 $P(i, j), P(i(c)), P(i, j(k))$ 的结构,可知它们也可以由单位矩阵经过一次初等列变换得到。
\end{enumerate}

所以,我们用以下形式给出初等矩阵的定义。

\begin{definition}{初等矩阵}
	由单位矩阵经过一次初等行(列)变换得到的矩阵称为\emph{初等矩阵}。记号如下。

	\begin{enumerate}
		\item 用 $P(i, j(k))$ 表示将第 $i$ 行加上第 $j$ 行的 $k$ 倍对应的初等矩阵。
		\item 用 $P(i, j)$ 表示将第 $i$ 行与第 $j$ 行交换对应的初等矩阵。
		\item 用 $P(i(c))$ 表示将第 $i$ 行乘上 $c$ 倍对应的初等矩阵($c \ne 0$)。
	\end{enumerate}
\end{definition}

初等矩阵与其他矩阵相乘的含义如下定理所述。

\begin{theorem}
	用初等矩阵左(右)乘 $A$,就相当于 $A$ 做了一次相应的初等行(列)变换。具体含义如下:
	\begin{enumerate}
		\item 用 $P(i, j(k))$ \textbf{左乘} $A$,相当于把 $A$ 的第 $j$ 行的 $k$ 倍加到第 $i$ 行上,其余行不变。
		\item 用 $P(i, j(k))$ \textbf{右乘} $A$,相当于把 $A$ 的第 $i$ 列的 $k$ 倍加到第 $j$ 列上,其余列不变(注意 $i$ 和 $j$ 互换了)。
		\item 用 $P(i, j)$ 左(右)乘 $A$,相当于把 $A$ 的第 $i$ 行(列)与第 $j$ 行(列)互换,其余行(列)不变。
		\item 用 $P(i(c)) \pod{c \ne 0}$ 左(右)乘 $A$,相当于用 $c$ 乘 $A$ 的第 $i$ 行(列),其余行(列)不变。
	\end{enumerate}
\end{theorem}

由此,我们将矩阵的乘法与初等变换联系起来了。利用初等变换的性质,我们可以得到以下有关初等矩阵的定理。

\begin{theorem}
	若 $|A| \ne 0$,则存在一系列初等矩阵的乘积 $P_s \cdots P_1$ 使得\footnote{$n$ 级矩阵的初等行变换不改变其行列式的非零性。}:
	$$
	P_s \cdots P_1 A = I_n
	$$
\end{theorem}

\begin{theorem}
	初等矩阵是可逆的。
\end{theorem}

\begin{proof}
	$$
	P^2(i, j) = I_n
	$$$$
	P(i(c)) P \biggl( i \biggl( \frac{1}{c} \biggr) \biggr) = I_n
	$$$$
	P(i, j(k)) P(i, j(-k)) = I_n
	$$
\end{proof}

\subsection{矩阵可逆的充分必要条件}



\subsection{可逆矩阵的性质}



\subsection{求逆矩阵的方法}

