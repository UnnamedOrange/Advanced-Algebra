% Licensed under the Creative Commons Attribution Share Alike 4.0 International.
% See the LICENCE file in the repository root for full licence text.

\chapter{行列式}

\section{行列式的定义}

\subsection{引入}

在二维直角坐标系下,如何求 $OA$、$OB$ 两条边构成的平行四边形的面积?利用向量的叉乘\footnote{在平面直角坐标系中,设 $\vec a = (a_1, b_1)$,$\vec b = (a_2, b_2)$,定义向量的叉乘为 $\vec a \times \vec b \triangleq a_1 b_2 - a_2 b_1$。$\vec a \times \vec b$ 的几何意义为 $\vec a$ 与 $\vec b$ 所围成的平行四边形的有向面积。},可以求出该平行四边形的\textbf{有向面积}。
$$
S = \overset{\longrightarrow}{OA} \times \overset{\longrightarrow}{OB}
$$

利用二维向量叉乘的性质,我们可以验证:
\begin{enumerate}
	\item 当 $\overset{\longrightarrow}{OA} = (1, 0)$,$\overset{\longrightarrow}{OB} = (0, 1)$ 时,$S = 1$。
	\item 设 $\overset{\longrightarrow}{OC} = a \cdot \overset{\longrightarrow}{OA} + b \cdot \overset{\longrightarrow}{OB}$,则 $\overset{\longrightarrow}{OC} \times \overset{\longrightarrow}{OB} = a \cdot \overset{\longrightarrow}{OA} \times \overset{\longrightarrow}{OB} + b \cdot \overset{\longrightarrow}{OB} \times \overset{\longrightarrow}{OB}$。
	\item $\overset{\longrightarrow}{OB} \times \overset{\longrightarrow}{OA} = -\overset{\longrightarrow}{OA} \times \overset{\longrightarrow}{OB}$。
\end{enumerate}

扩展到三维直角坐标系,问题变为如何求 $\overset{\longrightarrow}{OA}$、$\overset{\longrightarrow}{OB}$、$\overset{\longrightarrow}{OC}$ 三条边构成的平行六面体的有向体积。我们暂时未知“有向”为何意,但我们根据立体几何的直觉,知道该有向体积应该满足以下性质:
\begin{enumerate}
	\item 当 $\overset{\longrightarrow}{OA} = (1, 0, 0)$,$\overset{\longrightarrow}{OB} = (0, 1, 0)$,$\overset{\longrightarrow}{OC} = (0, 0, 1)$ 时,$V = 1$。
	\item 设该有向体积为 $f(\overset{\longrightarrow}{OA}, \overset{\longrightarrow}{OB}, \overset{\longrightarrow}{OC})$。则当 $\overset{\longrightarrow}{OD} = a \cdot \overset{\longrightarrow}{OA} + b \cdot \overset{\longrightarrow}{OB}$ 时,有:
	$$
	f(\overset{\longrightarrow}{OD}, \overset{\longrightarrow}{OB}, \overset{\longrightarrow}{OC}) = a \cdot f(\overset{\longrightarrow}{OA}, \overset{\longrightarrow}{OB}, \overset{\longrightarrow}{OC}) + b \cdot f(\overset{\longrightarrow}{OB}, \overset{\longrightarrow}{OB}, \overset{\longrightarrow}{OC})
	$$
\end{enumerate}

在定义了该有向体积为 $f(\overset{\longrightarrow}{OA}, \overset{\longrightarrow}{OB}, \overset{\longrightarrow}{OC})$ 后,我们规定“有向”的含义为:交换 $f$ 的任意两个参数,其函数值取反。在此基础上,我们可以发现 $f$ 的表现与二维情形下的叉乘一致。进一步地,我们希望求在 $n$ 维直角坐标系下由 $n$ 个以原点为起点的向量构成的平行 $2n$ 面体的有向体积,由此便可以引出行列式的定义。

\subsection{行列式的第一定义}

\begin{definition}{行列式}
	定义映射 $f: M_n(\mathbb F) \to \mathbb F$,其中 $\mathbb F$ 表示一个数域,$M_n(\mathbb F)$ 表示元素属于数域 $\mathbb F$ 的 $n$ 级矩阵组成的集合。若 $f$ 满足:
	\begin{enumerate}
		\item 归一化。
		$$
		f(I_n) = 1
		$$
		其中 $I_n$ 为\emph{单位矩阵},即:
		$$
		I_n =
		\begin{bmatrix}
			1 & \cdots & 0
			\\
			\vdots & 1 & \vdots
			\\
			0 & \cdots & 1
		\end{bmatrix}_{n \times n}
		$$
		\item 多线性。例如:
		$$
		\begin{aligned}&
			f\left(\begin{bmatrix}Aa_{11} + Bb_{11} & \cdots & Aa_{1n} + Bb_{1n}\\a_{22} & \cdots & a_{2n}\\\vdots & \ddots & \vdots \\a_{n1} & \cdots & a_{nn}\end{bmatrix}\right)
			\\=~&
			A \cdot f\left(\begin{bmatrix}a_{11} & \cdots & a_{1n}\\a_{22} & \cdots & a_{2n}\\\vdots & \ddots & \vdots \\a_{n1} & \cdots & a_{nn}\end{bmatrix}\right) + B \cdot f\left(\begin{bmatrix}b_{11} & \cdots & b_{1n}\\a_{22} & \cdots & a_{2n}\\\vdots & \ddots & \vdots \\a_{n1} & \cdots & a_{nn}\end{bmatrix}\right)
		\end{aligned}
		$$
		不止是第一行,对其他行也成立。
		\item 反对称。交换参数矩阵的任意两行,函数值取相反数。
	\end{enumerate}
	则称 $f(A)$ 为方阵 $A$ 的\emph{行列式},记为 $\det A$ 或 $|A|$。
\end{definition}

以上定义结合引入部分的讨论,可以给我们以下启发。设:
$$
\begin{gathered}
	\vec e_1 = (1, 0, 0, \ldots, 0)
	\\
	\vec e_2 = (0, 1, 0, \ldots, 0)
	\\
	\vdots
	\\
	\vec e_n = (0, 0, 0, \ldots, 1)
\end{gathered}
$$

则可以记单位矩阵 $I_n$ 为:
$$
I_n =
\begin{bmatrix}
	\vec e_1 \\ \vec e_2 \\ \vdots \\ \vec e_n
\end{bmatrix}
$$

即 $n$ 级方阵的每一行可以看作一个 $n$ 维向量。

\section{行列式的完全展开式}

\subsection{二阶行列式的完全展开式}

我们尝试利用行列式的第一定义求出二阶行列式的表达式。

\begin{solve}
	设 $f(A) = \det A$。
	$$
	\begin{aligned}
		f(\begin{bmatrix}a&b\\c&d\end{bmatrix}) &= f(\begin{bmatrix}a\vec e_1 + b\vec e_2\\c\vec e_1 + d\vec e_2\end{bmatrix})
		\\&= a \cdot f(\begin{bmatrix}\vec e_1\\c\vec e_1 + d\vec e_2\end{bmatrix}) + b \cdot f(\begin{bmatrix}\vec e_2\\c\vec e_1 + d\vec e_2\end{bmatrix})
		\\&= a\left(c \cdot f(\begin{bmatrix}\vec e_1\\\vec e_1\end{bmatrix}) +
		d \cdot f(\begin{bmatrix}\vec e_1\\\vec e_2\end{bmatrix})\right) +
		\\&~~~~
		b\left(c \cdot f(\begin{bmatrix}\vec e_2\\\vec e_1\end{bmatrix}) +
		d \cdot f(\begin{bmatrix}\vec e_2\\\vec e_2\end{bmatrix})\right)
	\end{aligned}
	$$

	根据行列式的反对称性质可知,若有两行相同,则行列式为 $0$。所以:
	$$
	\begin{aligned}
		\text{原式} &= ad \cdot f(\begin{bmatrix}\vec e_1\\\vec e_2\end{bmatrix}) + bc \cdot f(\begin{bmatrix}\vec e_2\\\vec e_1\end{bmatrix})
		\\&=
		ad - bc
	\end{aligned}
	$$
\end{solve}

为了得到 $n$ 阶行列式的完全展开式,我们需要引入排列的概念。

\subsection{$n$ 元排列}

\begin{definition}{$n$ 元排列}
	$n$ 个不同的自然数的一个全排列称为一个 \emph{$n$ 元排列}。
\end{definition}

下面讨论的一些性质,如果没有特别声明,考虑的是由 $1, 2, \ldots, n$ 组成的 $n$ 元排列,但对由任意 $n$ 个不同的自然数组成的 $n$ 元排列也成立。

\begin{definition}{顺序,逆序}
	对于 $n$ 元排列中的一对数,若这两个数中小的数在前,大的数在后,则称这一对数构成一个\emph{顺序};反之,称之为构成一个\emph{逆序}。
\end{definition}

\begin{definition}{逆序数}
	一个 $n$ 元排列中逆序的总数称为\emph{逆序数},记作 $\tau(a_1 a_2 \ldots a_n)$。
\end{definition}

\begin{definition}{奇排列,偶排列}
	逆序对为奇数的排列称为\emph{奇排列},逆序数为偶数的排列称为\emph{偶排列}。
\end{definition}

\begin{definition}{对换}
	把排列中的两个数 $a, b$ 互换位置,其余数不动,称这样的变换为一个\emph{对换},记作 $(a, b)$。
\end{definition}