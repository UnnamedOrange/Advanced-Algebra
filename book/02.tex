% Licensed under the Creative Commons Attribution Share Alike 4.0 International.
% See the LICENSE file in the repository root for full license text.

\chapter{行列式}

\section{行列式的定义}

\subsection{引入}

在二维直角坐标系下,如何求 $OA$、$OB$ 两条边构成的平行四边形的面积?利用向量的叉乘\footnote{在平面直角坐标系中,设 $\vec a = (a_1, b_1)$,$\vec b = (a_2, b_2)$,定义向量的叉乘为 $\vec a \times \vec b \triangleq a_1 b_2 - a_2 b_1$。$\vec a \times \vec b$ 的几何意义为 $\vec a$ 与 $\vec b$ 所围成的平行四边形的有向面积。},可以求出该平行四边形的\textbf{有向面积}。
$$
S = \overset{\longrightarrow}{OA} \times \overset{\longrightarrow}{OB}
$$

利用二维向量叉乘的性质,我们可以验证:
\begin{enumerate}
	\item 当 $\overset{\longrightarrow}{OA} = (1, 0)$,$\overset{\longrightarrow}{OB} = (0, 1)$ 时,$S = 1$。
	\item 设 $\overset{\longrightarrow}{OC} = a \cdot \overset{\longrightarrow}{OA} + b \cdot \overset{\longrightarrow}{OB}$,则 $\overset{\longrightarrow}{OC} \times \overset{\longrightarrow}{OB} = a \cdot \overset{\longrightarrow}{OA} \times \overset{\longrightarrow}{OB} + b \cdot \overset{\longrightarrow}{OB} \times \overset{\longrightarrow}{OB}$。
	\item $\overset{\longrightarrow}{OB} \times \overset{\longrightarrow}{OA} = -\overset{\longrightarrow}{OA} \times \overset{\longrightarrow}{OB}$。
\end{enumerate}

扩展到三维直角坐标系,问题变为如何求 $\overset{\longrightarrow}{OA}$、$\overset{\longrightarrow}{OB}$、$\overset{\longrightarrow}{OC}$ 三条边构成的平行六面体的有向体积。我们暂时未知“有向”为何意,但我们根据立体几何的直觉,知道该有向体积应该满足以下性质:
\begin{enumerate}
	\item 当 $\overset{\longrightarrow}{OA} = (1, 0, 0)$,$\overset{\longrightarrow}{OB} = (0, 1, 0)$,$\overset{\longrightarrow}{OC} = (0, 0, 1)$ 时,$V = 1$。
	\item 设该有向体积为 $f(\overset{\longrightarrow}{OA}, \overset{\longrightarrow}{OB}, \overset{\longrightarrow}{OC})$。则当 $\overset{\longrightarrow}{OD} = a \cdot \overset{\longrightarrow}{OA} + b \cdot \overset{\longrightarrow}{OB}$ 时,有:
	$$
	f(\overset{\longrightarrow}{OD}, \overset{\longrightarrow}{OB}, \overset{\longrightarrow}{OC}) = a \cdot f(\overset{\longrightarrow}{OA}, \overset{\longrightarrow}{OB}, \overset{\longrightarrow}{OC}) + b \cdot f(\overset{\longrightarrow}{OB}, \overset{\longrightarrow}{OB}, \overset{\longrightarrow}{OC})
	$$
\end{enumerate}

在定义了该有向体积为 $f(\overset{\longrightarrow}{OA}, \overset{\longrightarrow}{OB}, \overset{\longrightarrow}{OC})$ 后,我们规定“有向”的含义为:交换 $f$ 的任意两个参数,其函数值取反。在此基础上,我们可以发现 $f$ 的表现与二维情形下的叉乘一致。进一步地,我们希望求在 $n$ 维直角坐标系下由 $n$ 个以原点为起点的向量构成的平行 $2n$ 面体的有向体积,由此便可以引出行列式的定义。

\subsection{行列式的第一定义}

\begin{definition}{行列式}
	定义映射 $f: M_n(\mathbb F) \to \mathbb F$,其中 $\mathbb F$ 表示一个数域,$M_n(\mathbb F)$ 表示元素属于数域 $\mathbb F$ 的 $n$ 级矩阵组成的集合。若 $f$ 满足:
	\begin{enumerate}
		\item 归一化。
		$$
		f(I_n) = 1
		$$
		其中 $I_n$ 为\emph{单位矩阵},即:
		$$
		I_n =
		\begin{bmatrix}
			1 & \cdots & 0
			\\
			\vdots & 1 & \vdots
			\\
			0 & \cdots & 1
		\end{bmatrix}_{n \times n}
		$$
		\item 多线性。例如:
		$$
		\begin{aligned}&
			f\left(\begin{bmatrix}Aa_{11} + Bb_{11} & \cdots & Aa_{1n} + Bb_{1n}\\a_{22} & \cdots & a_{2n}\\\vdots & \ddots & \vdots \\a_{n1} & \cdots & a_{nn}\end{bmatrix}\right)
			\\=~&
			A \cdot f\left(\begin{bmatrix}a_{11} & \cdots & a_{1n}\\a_{22} & \cdots & a_{2n}\\\vdots & \ddots & \vdots \\a_{n1} & \cdots & a_{nn}\end{bmatrix}\right) + B \cdot f\left(\begin{bmatrix}b_{11} & \cdots & b_{1n}\\a_{22} & \cdots & a_{2n}\\\vdots & \ddots & \vdots \\a_{n1} & \cdots & a_{nn}\end{bmatrix}\right)
		\end{aligned}
		$$
		不止是第一行,对其他行也成立。
		\item 反对称。交换参数矩阵的任意两行,函数值取相反数。
	\end{enumerate}
	则称 $f(A)$ 为方阵 $A$ 的\emph{行列式},记为 $\det A$ 或 $|A|$。
\end{definition}

以上定义结合引入部分的讨论,可以给我们以下启发。设:
$$
\begin{gathered}
	\vec e_1 = (1, 0, 0, \ldots, 0)
	\\
	\vec e_2 = (0, 1, 0, \ldots, 0)
	\\
	\vdots
	\\
	\vec e_n = (0, 0, 0, \ldots, 1)
\end{gathered}
$$

则可以记单位矩阵 $I_n$ 为:
$$
I_n =
\begin{bmatrix}
	\vec e_1 \\ \vec e_2 \\ \vdots \\ \vec e_n
\end{bmatrix}
$$

即 $n$ 列矩阵的每一行可以看作一个 $n$ 维向量。

\section{行列式的完全展开式}

\subsection{二阶行列式的完全展开式}

我们尝试利用行列式的第一定义求出二阶行列式的表达式。

\begin{solve}
	设 $f(A) = \det A$。
	$$
	\begin{aligned}
		f \left( \begin{bmatrix}a&b\\c&d\end{bmatrix} \right) &= f \left( \begin{bmatrix}a\vec e_1 + b\vec e_2\\c\vec e_1 + d\vec e_2\end{bmatrix} \right)
		\\&= a \cdot f \left( \begin{bmatrix}\vec e_1\\c\vec e_1 + d\vec e_2\end{bmatrix} \right) + b \cdot f \left( \begin{bmatrix}\vec e_2\\c\vec e_1 + d\vec e_2\end{bmatrix} \right)
		\\&= a\left(c \cdot f \left( \begin{bmatrix}\vec e_1\\\vec e_1\end{bmatrix} \right) +
		d \cdot f \left( \begin{bmatrix}\vec e_1\\\vec e_2\end{bmatrix} \right) \right) +
		\\&~~~~
		b\left(c \cdot f \left( \begin{bmatrix}\vec e_2\\\vec e_1\end{bmatrix} \right) +
		d \cdot f \left( \begin{bmatrix}\vec e_2\\\vec e_2\end{bmatrix} \right) \right)
	\end{aligned}
	$$

	根据行列式的反对称性质可知,若有两行相同,则行列式为 $0$。所以:
	$$
	\begin{aligned}
		\text{原式} &= ad \cdot f \left( \begin{bmatrix}\vec e_1\\\vec e_2\end{bmatrix} \right) + bc \cdot f \left( \begin{bmatrix}\vec e_2\\\vec e_1\end{bmatrix} \right)
		\\&=
		ad - bc
	\end{aligned}
	$$
\end{solve}

为了得到 $n$ 阶行列式的完全展开式,我们需要引入排列的概念。

\subsection{$n$ 元排列}

\begin{definition}{$n$ 元排列}
	$n$ 个不同的自然数的一个全排列称为一个 \emph{$n$ 元排列}。
\end{definition}

\begin{definition}{顺序,逆序}
	对于 $n$ 元排列中的一对数,若这两个数中小的数在前,大的数在后,则称这一对数构成一个\emph{顺序};反之,称之为构成一个\emph{逆序}。
\end{definition}

\begin{definition}{逆序数}
	一个 $n$ 元排列中逆序的总数称为\emph{逆序数},记作 $\tau(a_1 a_2 \ldots a_n)$。
\end{definition}

\begin{definition}{奇排列,偶排列}
	逆序数为奇数的排列称为\emph{奇排列},逆序数为偶数的排列称为\emph{偶排列}。
\end{definition}

\begin{definition}{对换}
	把排列中的两个数 $a, b$ 互换位置,其余数不动,称这样的变换为一个\emph{对换},记作 $(a, b)$。
\end{definition}

下面讨论的一些性质,如果没有特别声明,考虑的是由 $1, 2, \ldots, n$ 组成的 $n$ 元排列,但对由任意 $n$ 个不同的自然数组成的 $n$ 元排列也成立。

\begin{theorem}
	对换改变 $n$ 元排列的奇偶性。
\end{theorem}

\begin{proof}
	当对换的两个数相邻时,交换后,显然逆序数加一或者减一,导致 $n$ 元排列的奇偶性改变。

	当对换的两个数不相邻时,考虑以下交换方法:
	$$
	\begin{aligned}
		& i~\underset{\text{$t$ 个数}}{\underbrace{x~\cdots~x}}~j & \pod{\text{相邻交换 $t$ 次}}
		\\
		\longrightarrow \quad & i~j~\underset{\text{$t$ 个数}}{\underbrace{x~\cdots~x}} & \pod{\text{相邻交换 $1$ 次}}
		\\
		\longrightarrow \quad & j~i~\underset{\text{$t$ 个数}}{\underbrace{x~\cdots~x}} & \pod{\text{相邻交换 $t$ 次}}
		\\
		\longrightarrow \quad & j~\underset{\text{$t$ 个数}}{\underbrace{x~\cdots~x}}~i
	\end{aligned}
	$$

	共交换 $2t + 1$ 次,即交换了相邻的两个数奇数次。可知 $n$ 元排列的奇偶性也发生了改变。
\end{proof}

\begin{theorem}
	任一 $n$ 元排列与排列 $123 \ldots n$ 可以经过一系列对换互变,并且所做对换的次数与这个 $n$ 元排列有相同的奇偶性。
\end{theorem}

\begin{proof}
	构造互变方法。从 $1$ 到 $n$ 枚举 $i$,对于第 $i$ 个位置,若这个位置是 $i$,则跳过,若这个位置不是 $i$,则将这个位置的数与 $i$ 进行一次对换。最终任一 $n$ 元排列都能通过该方法变为 $123 \ldots n$。而要实现互变,只需要将之前所作的对换按相反顺序进行一次即可。

	考虑 $123 \ldots n$,这个排列是偶排列。因此对于任一排列,若该排列是奇排列,则需要作奇数次对换才能变为偶排列,而 $123 \ldots n$ 是偶排列,因此需要作的对换的次数与该排列具有相同的奇偶性。同理,若该排列是偶排列,则作偶数次对换才能仍然为偶排列,而 $123 \ldots n$ 是偶排列,因此需要作的对换的次数与该排列具有相同的奇偶性。
\end{proof}

\subsection{$n$ 阶行列式的完全展开式}

\begin{theorem}[行列式的完全展开式]
	$n$ 阶行列式:
	$$
	\begin{vmatrix}
		a_{11} & a_{12} & \cdots & a_{1n}
		\\
		a_{21} & a_{22} & \cdots & a_{2n}
		\\
		\vdots & \vdots &  & \vdots
		\\
		a_{n1} & a_{n2} & \cdots & a_{nn}
		\end{vmatrix}
	$$

	是 $n!$ 项的代数和:
	$$
	\begin{vmatrix}
		a_{11} & a_{12} & \cdots & a_{1n}
		\\
		a_{21} & a_{22} & \cdots & a_{2n}
		\\
		\vdots & \vdots &  & \vdots
		\\
		a_{n1} & a_{n2} & \cdots & a_{nn}
		\end{vmatrix} =
		\sum_{j_1 j_2 \ldots j_n} (-1)^{\tau(j_1 j_2 \ldots j_n)} a_{1j_1} a_{2j_2} \cdots a_{nj_n}
	$$
	其中 $j_1 j_2 \ldots j_n$ 表示枚举 $123 \ldots n$ 的所有排列。

	上式称为 $n$ 阶行列式的\emph{完全展开式}。
\end{theorem}

\begin{proof}
	由于矩阵的每一行可以看作一个向量,所以设:
	$$
	\det A =
	\begin{vmatrix}
		a_{11} \cdot \vec e_1 + \sum\limits_{i = 2}^n a_{1i} \cdot \vec e_i
		\\
		\sum\limits_{i = 1}^n a_{2i} \cdot \vec e_i
		\\
		\vdots
		\\
		\sum\limits_{i = 1}^n a_{ni} \cdot \vec e_i
	\end{vmatrix}
	$$

	利用行列式的多线性,上式等于:
	$$
	a_{11} \begin{vmatrix} \vec e_1 \\ \sum\limits_{i = 1}^n a_{2i} \cdot \vec e_i \\ \vdots \\ \sum\limits_{i = 1}^n a_{ni} \cdot \vec e_i \end{vmatrix} + \begin{vmatrix} \sum\limits_{i = 2}^n a_{1i} \cdot \vec e_i \\ \sum\limits_{i = 1}^n a_{2i} \cdot \vec e_i \\ \vdots \\ \sum\limits_{i = 1}^n a_{ni} \cdot \vec e_i \end{vmatrix}
	$$

	进一步把第一行完全展开,上式等于:
	$$
	\sum\limits_{i_1 = 1}^n a_{1 i_1} \begin{vmatrix} \vec e_{i_1} \\ \sum\limits_{i = 1}^n a_{2i} \cdot \vec e_i \\ \vdots \\ \sum\limits_{i = 1}^n a_{ni} \cdot \vec e_i \end{vmatrix}
	$$

	进一步展开第二行,上式等于:
	$$
	\sum\limits_{i_1 = 1}^n a_{1 i_1} \Biggl( \sum\limits_{i_2 = 1}^n a_{2 i_2} \begin{vmatrix} \vec e_{i_1} \\ \vec e_{i_2} \\ \sum\limits_{i = 1}^n a_{3i} \cdot \vec e_i \\ \vdots \\ \sum\limits_{i = 1}^n a_{ni} \cdot \vec e_i \end{vmatrix} \Biggr)
	$$

	进一步展开剩余行,得:
	$$
	\sum\limits_{i_1 = 1}^n \sum\limits_{i_2 = 1}^n \cdots \sum\limits_{i_n = 1}^n a_{1 i_1} a_{2 i_2} \cdots a_{n i_n} \begin{vmatrix} \vec e_{i_1} \\ \vec e_{i_2} \\ \vdots \\ \vec e_{i_n} \end{vmatrix}
	$$

	当 $\begin{vmatrix} \vec e_{i_1} \\ \vec e_{i_2} \\ \vdots \\ \vec e_{i_n} \end{vmatrix}$ 中有两行相等时,该项为 $0$,否则,由奇偶排列与对换次数的关系,可得原式等于:
	$$
	\sum\limits_{(i_1, i_2, \ldots, i_n) \in S_n} (-1)^{\tau(i_1 i_2 \cdots i_n)} a_{1 i_1} a_{2 i_2} \cdots a_{n i_n}
	$$
	其中 $S_n$ 是由 $1, 2, \ldots, n$ 组成的所有 $n$ 元排列的集合。
\end{proof}