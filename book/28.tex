% Licensed under the Creative Commons Attribution Share Alike 4.0 International.
% See the LICENCE file in the repository root for full licence text.

\chapter{多项式环}

\section{一元多项式环的基本概念}

\subsection{一元多项式的基本概念}

一元多项式的研究动力是,我们需要求解一元多项式 $f(x) = 0$ 的根。因为需要对一元多项式 $f(x)$ 进行因式分解,所以需要研究一元多项式组成的集合的结构。

\begin{definition}{一元多项式,不定元,系数,$i$ 次项,零次项,常数项}
	形如 $a_n x^n + a_{n - 1} x^{n - 1} + \cdots + a_1 x + a_0$ 的表达式称为数域 $\mathbb K$ 上的\emph{一元多项式}。其中 $x$ 是一个符号(它不属于 $\mathbb K$),称之为\emph{不定元};$n$ 是非负整数;$a_i \in \mathbb K \pod{i = 0, 1, \ldots, n}$ 称为\emph{系数};$a_i x^i$ 称为 \emph{$i$ 次项}($i = 1, 2, \ldots, n$);$a_0$ 称为\emph{零次项}或\emph{常数项}。
\end{definition}

两个一元多项式相等规定为它们含有完全相同的项(除去系数为 $0$ 的项外,系数为 $0$ 的项允许任意删去和添加)。

\begin{definition}{零多项式}
	系数全为 $0$ 的多项式称为\emph{零多项式},记作 $0$。
\end{definition}

\begin{definition}{首项,次数}
	设 $f(x) = a_n x^n + \cdots + a_1 x + a_0$,如果 $a_n \ne 0$,那么称 $a_n x^n$ 是 $f(x)$ 的\emph{首项},称 $n$ 是 $f(x)$ 的\emph{次数},记作 $\deg f(x)$ 或 $\deg f$。
\end{definition}

零多项式的次数被定义为 $-\infty < 0$。

\bigskip

数域 $\mathbb K$ 上所有一元多项式组成的集合记作 $\mathbb K[x]$。在 $\mathbb K[x]$ 中可以定义加法和乘法运算。设 $f(x) = \sum\limits_{i = 0}^n a_i x^i$,$g(x) = \sum\limits_{i = 0}^m b_i x^i$,不妨设 $m \le n$,定义:
$$
f(x) + g(x) = \sum_{i = 0}^n (a_i + b_i) x^i
$$$$
f(x) g(x) = \sum_{s = 0}^{n + m} \biggl( \sum_{i + j = s} a_i b_j \biggr) x^s
$$

\subsection{环的基本概念}

\begin{definition}{代数运算}
	设某个集合为 $S$。称 $S \times S$ 到 $S$ 的一个映射为集合 $S$ 上的一个\emph{代数运算}。
\end{definition}

\begin{definition}{环,加法,乘法,零元素}
	设 $R$ 是一个非空集合,如果它有两个代数运算,一个叫做\emph{加法},记作 $a + b$,另一个叫做\emph{乘法},记作 $ab$,并且这两个运算满足下列六条运算法则($\forall a, b, c \in R$):
	\begin{itemize}
		\item 加法结合律,即 $(a + b) + c = a + (b + c)$
		\item 加法交换律,即 $a + b = b + a$
		\item 在 $R$ 中有元素 $0$,使得 $a + 0 = a$,称 $0$ 是 $R$ 的\emph{零元素}。
		\item 对于 $a$,在 $R$ 中有元素 $d$,使得 $a + d = 0$,称 $d$ 是 $a$ 的\emph{负元素},记作 $-a$。
		\item 乘法结合律,即 $(ab)c = a(bc)$
		\item 乘法对于加法的左、右分配律,即:
		$$
		a(b + c) = ab + ac
		$$$$
		(b + c) a = ba + ca
		$$
	\end{itemize}
	那么称 $R$ 是一个\emph{环}。
\end{definition}

之所以要定义环,是因为我们想要抽象出像 $\Z, \mathbb K[x], M_n[\mathbb K]$ 这样的集合所具有的共同性质。下面首先给出一些有关环的简单结论。

\begin{proposition}
	环 $R$ 中的零元素是唯一的。
\end{proposition}

\begin{proof}[反证法]
	设 $\forall a \in R$,有 $a + 0_1 = a \pod{0_1 \in R}$,又有 $a + 0_2 = a \pod{0_2 \in R}$,且 $0_1 \ne 0_2$,则分别代入 $0_2$ 和 $0_1$ 有:
	$$
	0_1 + 0_2 = 0_1 = 0_2 = 0_2 + 0_1
	$$

	矛盾,说明零元素是唯一的。
\end{proof}

\begin{theorem}
	环 $R$ 中元素 $a$ 的负元素是唯一的,且 $-(-a) = a$。
\end{theorem}

\begin{proof}[反证法]
	设 $\forall a \in R$,有 $a + (-a_1) = 0$,又有 $a + (-a_2) = 0$,且 $-a_1 \ne -a_2$,则等式 $a + (-a_1) = a + (-a_2)$ 两侧同时加上 $-a_1$ 有:
	$$
	0 + (-a_1) = 0 + (-a_2)
	$$

	矛盾,说明负元素是唯一的。

	\bigskip

	因为 $a$ 的负元是 $-a$,所以 $a + (-a) = 0$,这也意味着 $-a$ 的负元是 $a$,即证得 $-(-a) = a$。
\end{proof}

容易发现,$\Z$、$\mathbb K[x]$、$M_n[\mathbb K]$ 都是环,它们分别被称为\emph{\idx{整数环}}、\emph{数域 $\mathbb K$ 上\idx{一元多项式环}}、\emph{数域 $\mathbb K$ 上 $n$ 级\idx{全矩阵环}}。任意一个数域 $\mathbb K$ 也是环。

\bigskip

在环的定义的基础上,存在一些特殊的环。

\begin{definition}{交换环}
	若环 $R$ 中的乘法还满足交换律,则称 $R$ 为\emph{交换环}。
\end{definition}

\begin{definition}{单位元,有单位元的环}
	若环 $R$ 中有一个元素 $e$ 满足 $\forall a \in R, ea = ae = a$,则称 $e$ 是 $R$ 的\emph{单位元},此时称 $R$ 是\emph{有单位元的环}。
\end{definition}

\begin{proposition}
	在有单位元的环 $R$ 中,单位元是唯一的,通常把单位元记成 $1$。
\end{proposition}

\subsection{子环与扩环}

