% Licensed under the Creative Commons Attribution Share Alike 4.0 International.
% See the LICENSE file in the repository root for full license text.

\chapter{多项式环}

\section{一元多项式环的基本概念}

\subsection{一元多项式的基本概念}

一元多项式的研究动力是,我们需要求解一元多项式 $f(x) = 0$ 的根。因为需要对一元多项式 $f(x)$ 进行因式分解,所以需要研究一元多项式组成的集合的结构。

\begin{definition}{一元多项式,不定元,系数,$i$ 次项,零次项,常数项}
	形如 $a_n x^n + a_{n - 1} x^{n - 1} + \cdots + a_1 x + a_0$ 的表达式称为数域 $\mathbb K$ 上的\emph{一元多项式}。其中 $x$ 是一个符号(它不属于 $\mathbb K$),称之为\emph{不定元};$n$ 是非负整数;$a_i \in \mathbb K \pod{i = 0, 1, \ldots, n}$ 称为\emph{系数};$a_i x^i$ 称为 \emph{$i$ 次项}($i = 1, 2, \ldots, n$);$a_0$ 称为\emph{零次项}或\emph{常数项}。
\end{definition}

两个一元多项式相等规定为它们含有完全相同的项(除去系数为 $0$ 的项外,系数为 $0$ 的项允许任意删去和添加)。

\begin{definition}{零多项式}
	系数全为 $0$ 的多项式称为\emph{零多项式},记作 $0$。
\end{definition}

\begin{definition}{首项,次数}
	设 $f(x) = a_n x^n + \cdots + a_1 x + a_0$,如果 $a_n \ne 0$,那么称 $a_n x^n$ 是 $f(x)$ 的\emph{首项},称 $n$ 是 $f(x)$ 的\emph{次数},记作 $\deg f(x)$ 或 $\deg f$。
\end{definition}

零多项式的次数被定义为 $-\infty < 0$。

\bigskip

数域 $\mathbb K$ 上所有一元多项式组成的集合记作 $\mathbb K[x]$。在 $\mathbb K[x]$ 中可以定义加法和乘法运算。设 $f(x) = \sum\limits_{i = 0}^n a_i x^i$,$g(x) = \sum\limits_{i = 0}^m b_i x^i$,不妨设 $m \le n$,定义:
$$
f(x) + g(x) = \sum\limits_{i = 0}^n (a_i + b_i) x^i
$$$$
f(x) g(x) = \sum\limits_{s = 0}^{n + m} \biggl( \sum\limits_{i + j = s} a_i b_j \biggr) x^s
$$

\subsection{一元多项式的基本性质}

\begin{proposition}
	对于 $f(x), g(x) \in \mathbb K[x]$,有:
	$$
	\deg(f(x) \pm g(x)) \le \max \{ \deg f(x), \deg g(x) \}
	$$
\end{proposition}

\begin{proposition}
	对于 $f(x), g(x) \in \mathbb K[x]$,有:
	$$
	\deg(f(x) g(x)) = \deg f(x) + \deg g(x)
	$$
\end{proposition}

一元多项式满足消去律。

\begin{proposition}
	对于 $f(x), g(x), h(x) \in \mathbb K[x] \pod{f(x) \ne 0}$,若 $f(x) g(x) = f(x) h(x)$,则 $g(x) = h(x)$。
\end{proposition}

\begin{proof}
	移项得 $f(x)(g(x) - h(x)) = 0$,于是 $\deg f(x) + \deg(g(x) - h(x)) = - \infty$,由 $\deg f(x) \ge 0$,可知 $\deg(g(x) - h(x)) = -\infty$,所以 $g(x) = h(x)$。
\end{proof}

\subsection{环的基本概念}

\begin{definition}{代数运算}
	设某个集合为 $S$。称 $S \times S$ 到 $S$ 的一个映射为集合 $S$ 上的一个\emph{代数运算}。
\end{definition}

\begin{definition}{环,加法,乘法,零元素}
	设 $R$ 是一个非空集合,如果它有两个代数运算,一个叫做\emph{加法},记作 $a + b$,另一个叫做\emph{乘法},记作 $ab$,并且这两个运算满足下列六条运算法则($\forall a, b, c \in R$):
	\begin{itemize}
		\item 加法结合律,即 $(a + b) + c = a + (b + c)$
		\item 加法交换律,即 $a + b = b + a$
		\item 在 $R$ 中有元素 $0$,使得 $a + 0 = a$,称 $0$ 是 $R$ 的\emph{零元素}。
		\item 对于 $a$,在 $R$ 中有元素 $d$,使得 $a + d = 0$,称 $d$ 是 $a$ 的\emph{负元素},记作 $-a$。
		\item 乘法结合律,即 $(ab)c = a(bc)$
		\item 乘法对于加法的左、右分配律,即:
		$$
		a(b + c) = ab + ac
		$$$$
		(b + c) a = ba + ca
		$$
	\end{itemize}
	那么称 $R$ 是一个\emph{环}。
\end{definition}

之所以要定义环,是因为我们想要抽象出像 $\Z, \mathbb K[x], M_n(\mathbb K)$ 这样的集合所具有的共同性质。下面首先给出一些有关环的简单结论。

\begin{proposition}
	环 $R$ 中的零元素是唯一的。
\end{proposition}

\begin{proof}[反证法]
	设 $\forall a \in R$,有 $a + 0_1 = a \pod{0_1 \in R}$,又有 $a + 0_2 = a \pod{0_2 \in R}$,且 $0_1 \ne 0_2$,则分别代入 $0_2$ 和 $0_1$ 有:
	$$
	0_1 + 0_2 = 0_1 = 0_2 = 0_2 + 0_1
	$$

	矛盾,说明零元素是唯一的。
\end{proof}

\begin{theorem}
	环 $R$ 中元素 $a$ 的负元素是唯一的,且 $-(-a) = a$。
\end{theorem}

\begin{proof}[反证法]
	设 $\forall a \in R$,有 $a + (-a_1) = 0$,又有 $a + (-a_2) = 0$,且 $-a_1 \ne -a_2$,则等式 $a + (-a_1) = a + (-a_2)$ 两侧同时加上 $-a_1$ 有:
	$$
	0 + (-a_1) = 0 + (-a_2)
	$$

	矛盾,说明负元素是唯一的。

	\bigskip

	因为 $a$ 的负元是 $-a$,所以 $a + (-a) = 0$,这也意味着 $-a$ 的负元是 $a$,即证得 $-(-a) = a$。
\end{proof}

容易发现,$\Z$、$\mathbb K[x]$、$M_n(\mathbb K)$ 都是环,它们分别被称为\emph{\idx{整数环}}、\emph{数域 $\mathbb K$ 上\idx{一元多项式环}}、\emph{数域 $\mathbb K$ 上 $n$ 级\idx{全矩阵环}}。任意一个数域 $\mathbb K$ 也是环。

\bigskip

在环的定义的基础上,存在一些特殊的环。

\begin{definition}{交换环}
	若环 $R$ 中的乘法还满足交换律,则称 $R$ 为\emph{交换环}。
\end{definition}

\begin{definition}{单位元,有单位元的环}
	若环 $R$ 中有一个元素 $e$ 满足 $\forall a \in R, ea = ae = a$,则称 $e$ 是 $R$ 的\emph{单位元},此时称 $R$ 是\emph{有单位元的环}。
\end{definition}

\begin{proposition}
	在有单位元的环 $R$ 中,单位元是唯一的,通常把单位元记成 $1$。
\end{proposition}

\subsection{子环与扩环}

\begin{definition}{子环}
	如果环 $R$ 的一个非空子集 $R_1$ 对于 $R$ 的加法和乘法也成为一个环,那么称 $R_1$ 是 $R$ 的一个\emph{子环}。
\end{definition}

\begin{proposition}
	环 $R$ 的一个非空子集 $R_1$ 为一个子环的充分必要条件是 $R_1$ 对于 $R$ 的\textbf{减法}和乘法都封闭,即 $\forall a, b \in R_1, a - b \in R_1$ 且 $ab \in R_1$。
\end{proposition}

\begin{proof}
	必要性显然,只证充分性。由已知条件得 $c - c \in R_1$,即 $0 \in R_1$,所以可得 $-b \in R_1$,进一步得 $a + b = a - (-b) \in R_1$。

	于是可以把 $R$ 的加法和乘法看成是 $R_1$ 的加法和乘法,易得 $R_1$ 是 $R$ 的子环。
\end{proof}

从以上证明过程中可以看出,要求对减法封闭是为了保证集合中存在零元素。

\begin{definition}{扩环}
	设 $R$ 是有单位元 $1'$ 的交换环,如果 $R$ 有一个子环 $R_1$ 满足下列条件:

	\begin{enumerate}
		\item $1' \in R_1$
		\item 数域 $\mathbb K$ 到 $R_1$ 有一个双射 $\tau$,且 $\tau$ 保持加法与乘法运算。即,存在一个双射 $\tau \colon \mathbb K \to R_1$ 满足 $\tau(a + b) = \tau(a) + \tau(b)$ 且 $\tau(ab) = \tau(a) \tau(b)$。注意等号左边的加号(乘号)是 $\mathbb K$ 中的加号(乘号),等号右边的加号(乘号)是 $R_1$ 中的乘号。
	\end{enumerate}

	那么 $R$ 可看成是 $\mathbb K$ 的一个\emph{扩环}。
\end{definition}

形象地说,若环 $R$ 的一个子环 $R_1$ 可以被挖去,然后用数域 $\mathbb K$ 取而代之,则称 $R$ 是 $\mathbb K$ 的一个扩环。引入扩环的概念是为了证明一些一元多项式的性质。

\begin{example}
	\begin{enumerate}
		\item $\mathbb K[x]$ 是 $\mathbb K$ 的扩环,双射 $\tau$ 是:
		$$
		\begin{aligned}
			\tau \colon & \mathbb K \to \set{t \colon t \in \mathbb K[x], \deg t \le 0}
			\\&
			k \mapsto k
		\end{aligned}
		$$

		\item 给定 $A \in M_n(\mathbb K)$,形如下述的表达式被称为数域 $\mathbb K$ 上矩阵 $A$ 的多项式:
		$$
		a_m A^m + a_{m - 1} A^{m - 1} + \cdots + a_1 A + a_0 I
		$$
		其中 $m \in \N, a_i \in \mathbb K, i = 0, 1, \ldots, m$。注意以上表达式表示的是一个矩阵,而非一个含不定元的多项式。

		把数域 $\mathbb K$ 上矩阵 $A$ 的所有多项式组成的集合记作 $\mathbb K[A]$,可知 $\mathbb K[A]$ 是 $M_n(\mathbb K)$ 的一个子环。同时,$\mathbb K[A]$ 是一个有单位元的交换环,是 $\mathbb K$ 的扩环。
	\end{enumerate}
\end{example}

若 $R$ 是 $\mathbb K$ 的扩环,则子环 $R_1$ 的结构与 $\mathbb K$ 有一定相似性。

\begin{proposition}
	如果有单位元 $1'$ 的交换环 $R$ 可看作是数域 $\mathbb K$ 的扩环,那么 $\mathbb K$ 到子环 $R_1$ 的双射 $\tau$ 满足 $\tau(1) = 1'$。
\end{proposition}

\begin{proof}
	任取 $b \in R_1$,由于 $\tau$ 是满射,因而存在 $k \in \mathbb K$,使得 $\tau(k) = b$。于是:\
	$$
	\tau(1) b = \tau(1) \tau(k) = \tau(1k) = \tau(k) = b
	$$

	从而 $\tau(1)$ 是交换环 $R_1$ 的单位元。由于单位元在环中是唯一的,所以 $\tau(1) = 1'$。
\end{proof}

\subsection{代入}

\begin{definition}{代入}
	设 $R$ 是数域 $\mathbb K$ 的扩环,对应双射记为 $\tau$,定义映射:\
	$$
	\begin{aligned}
		\sigma_t \colon & \mathbb K[x] \to R
		\\&
		f(x) = \sum\limits_{i = 0}^n a_i x^i \mapsto \sum\limits_{i = 0}^n \tau(a_i) t^i \triangleq f(t)
	\end{aligned}
	$$

	映射 $\sigma_t$ 被称为 $x$ 用 $t$ \emph{代入}。
\end{definition}

研究代入的动力是,如果我们已知一个公式,例如:
$$
(x + a)^2 = x^2 + 2ax + a^2
$$

我们希望将 $x$ 替换为一个常数,例如:
$$
(100 + 1)^2 = 100^2 + 2 \times 1 \times 100 + 1^2
$$

而等式仍然成立。下面的定理说明了代入的正确性。

\begin{theorem}
	设 $R$ 是数域 $\mathbb K$ 的扩环。如果在 $\mathbb K[x]$ 中,有 $f(x) + g(x) = h(x)$,$f(x)g(x) = p(x)$,那么在 $R$ 中有 $f(t) + g(t) = h(t)$,$f(t) g(t) = p(t)$,而且还有 $\sigma_t(x) = t$。
\end{theorem}

由以上定理可知,由于 $\mathbb K[x], \mathbb K[A]$ 都是 $\mathbb K$ 的扩环,因而不定元 $x$ 可以用 $\mathbb K[x]$ 中任一多项式代入,也可以用矩阵 $A$ 的任一多项式代入。更一般地,可以用 $\mathbb K$ 的任一扩环中的任一元素代入。

\subsection{例题}

\begin{exercise}
	设数域 $\mathbb K$ 上的 $n$ 级矩阵 $A$ 为:
	$$
	A =
	\begin{bmatrix}
		k & c & 0 & \cdots & 0
		\\
		0 & k & c & \cdots & 0
		\\
		\vdots & \vdots & \vdots & & \vdots
		\\
		0 & 0 & 0 & \cdots & c
		\\
		0 & 0 & 0 & \cdots & k
	\end{bmatrix}
	$$
	其中 $k, c \in \mathbb K \backslash \set{0}$。

	说明 $A$ 可逆,并且求 $A^{-1}$。
\end{exercise}

\begin{solve}
	记 $A = k I_n + c H$。注意到,$H$ 是一个幂零矩阵,满足 $H^n = 0$。

	在 $\mathbb K[x]$ 中直接计算可得:
	$$
	(1 - x)(1 + x + \cdots+ x^{n - 1}) = 1 - x^n
	$$

	由于 $\mathbb K[H]$ 可看成是 $K$ 的一个扩环,所以 $x$ 可以用 $-\dfrac{c}{k} H$ 代入,整理可得:
	$$
	(kI + cH) \biggl( \dfrac{1}{k} I_n - \dfrac{c}{k^2} H + \cdots + (-1)^{n - 1} \dfrac{c^{n - 1}}{k^n} H^{n - 1} \biggr) = I_n
	$$

	则 $A^{-1}$ 为:
	$$
	A^{-1} = \dfrac{1}{k} I_n - \dfrac{c}{k^2} H + \cdots + (-1)^{n - 1} \dfrac{c^{n - 1}}{k^n} H^{n - 1}
	$$
\end{solve}

\begin{exercise}
	设数域 $\mathbb K$ 上 $n$ 级矩阵 $A$ 的特征多项式为:
	$$
	|\lambda I - A| = (\lambda - \lambda_1)^{l_1} \cdots (\lambda - \lambda_s)^{l_s}
	$$

	求证:对于 $k \in \mathbb K \backslash \set{0}$,矩阵 $kA$ 的特征多项式为 $|\lambda I - kA| = (\lambda - k \lambda_1)^{l_1} \cdots (\lambda - k \lambda_s)^{l_s}$。(不妨假设 $\lambda_i \in \mathbb K$)
\end{exercise}

\begin{proof}
	计算 $|\lambda I - A|$:
	$$
	\begin{aligned}
		|\lambda I - A| &=
		\begin{vmatrix}
			\lambda - a_{11} & -a_{12} & \cdots & -a_{1n}
			\\
			-a_{21} & \lambda - a_{22} & \cdots & -a_{2n}
			\\
			\vdots & \vdots && \vdots
			\\
			-a_{n1} & -a_{n2} & \cdots & \lambda - a_{nn}
		\end{vmatrix}
		\\&=
		\sum\limits_{j_1 j_2 \cdots j_n} (-1)^{\tau(j_1 j_2 \cdots j_n)} (\lambda \delta_{1 j_1} - a_{1 j_1}) \cdots (\lambda \delta_{n j_n} - a_{n j_n})
		\\&=
		(\lambda - \lambda_1)^{l_1} \cdots (\lambda - \lambda_s)^{l_s}
	\end{aligned}
	$$
	其中 $\delta_{ij}$ 是 Kronecker 记号。

	注意到:
	$$
	\sum\limits_{j_1 j_2 \cdots j_n} (-1)^{\tau(j_1 j_2 \cdots j_n)} (\lambda \delta_{1 j_1} - a_{1 j_1}) \cdots (\lambda \delta_{n j_n} - a_{n j_n}) \in \mathbb K[\lambda]
	$$$$
	(\lambda - \lambda_1)^{l_1} \cdots (\lambda - \lambda_s)^{l_s} \in \mathbb K[\lambda]
	$$

	又因为 $\dfrac{\lambda}{k} \in \mathbb K[\lambda]$,而 $\mathbb K[\lambda]$ 是 $\mathbb K$ 的扩环,所以可以把不定元 $\lambda$ 用 $\dfrac{\lambda}{k}$ 代入以上等式,得:
	$$
	\begin{aligned}&
		\sum\limits_{j_1 j_2 \cdots j_n} (-1)^{\tau(j_1 j_2 \cdots j_n)} \biggl( \dfrac{\lambda}{k} \delta_{1 j_1} - a_{1 j_1} \biggr) \cdots \biggl( \dfrac{\lambda}{k} \delta_{n j_n} - a_{n j_n} \biggr)
		\\=~&
		\biggl( \dfrac{\lambda}{k} - \lambda_1 \biggr)^{l_1} \cdots \biggl( \dfrac{\lambda}{k} - \lambda_s \biggr)^{l_s}
	\end{aligned}
	$$

	根据行列式的定义,$\qty|\dfrac{\lambda}{k} I - A|$ 等于:
	$$
	\begin{aligned}&
		\qty|\dfrac{\lambda}{k} I - A|
		\\=~&
		\sum\limits_{j_1 j_2 \cdots j_n} (-1)^{\tau(j_1 j_2 \cdots j_n)} \biggl( \dfrac{\lambda}{k} \delta_{1 j_1} - a_{1 j_1} \biggr) \cdots \biggl( \dfrac{\lambda}{k} \delta_{n j_n} - a_{n j_n} \biggr)
	\end{aligned}
	$$

	所以:
	$$
	\qty|\dfrac{\lambda}{k} I - A| = \biggl( \dfrac{\lambda}{k} - \lambda_1 \biggr)^{l_1} \cdots \biggl( \dfrac{\lambda}{k} - \lambda_s \biggr)^{l_s}
	$$

	等式两边同时乘以 $k^n$,即可得:
	$$
	|\lambda I - kA| = (\lambda - k \lambda_1)^{l_1} \cdots (\lambda - k \lambda_s)^{l_s}
	$$
\end{proof}

需要注意的是,我们不能直接说 $|\lambda I - A| \in \mathbb K[\lambda]$,需要写出行列式的完全展开式后才能如此断言,因为 $|\lambda I - A|$ 本身只是一个记号。而应用代入时,必须代入到一个关于多项式的等式中,否则无法应用。

\begin{exercise}
	设数域 $\mathbb K$ 上 $n$ 级矩阵 $A$ 的特征多项式为:
	$$
	|\lambda I - A| = (\lambda - \lambda_1)^{l_1} \cdots (\lambda - \lambda_s)^{l_s}
	$$

	求证:$A^2$ 的特征多项式为 $|\lambda I - A^2| = (\lambda - \lambda_1^2)^{l_1} \cdots (\lambda - \lambda_s^2)^{l_s}$。(不妨假设 $\lambda_i \in \mathbb K$)
\end{exercise}

\begin{proof}
	由上例,已知:
	$$
	|\lambda I - kA| = (\lambda - k \lambda_1)^{l_1} \cdots (\lambda - k \lambda_s)^{l_s}
	$$

	取 $k = -1$,然后与原式相乘得:
	$$
	|\lambda^2 I - A^2| = (\lambda^2 - \lambda_1^2)^{l_1} \cdots (\lambda^2 - \lambda_s^2)^{l_s}
	$$

	同上例,将以上等式中的 $|\lambda^2 I - A^2|$ 展开,得到一个关于 $\lambda^2$ 的多项式,则以上等式被化为一个 $\mathbb K[\lambda^2]$ 上的等式。由于 $\mathbb K[\lambda]$ 是 $\mathbb K$ 的一个扩环,所以可以把等式中的 $\lambda^2$ 用 $\lambda$ 代入,最后根据行列式的完全展开式得:
	$$
	|\lambda I - A^2| = (\lambda - \lambda_1^2)^{l_1} \cdots (\lambda - \lambda_s^2)^{l_s}
	$$
\end{proof}

\begin{exercise}
	设数域 $\mathbb K$ 上 $n$ 级矩阵 $A$ 的特征多项式为:
	$$
	|\lambda I - A| = (\lambda - \lambda_1)^{l_1} \cdots (\lambda - \lambda_s)^{l_s}
	$$

	求证:对于任一正整数 $m$,$A^m$ 的特征多项式为 $|\lambda I - A^m| = (\lambda - \lambda_1^m)^{l_1} \cdots (\lambda - \lambda_s^m)^{l_s}$。(不妨假设 $\lambda_i \in \mathbb K$)
\end{exercise}

\begin{proof}
	设 $\xi = \mathrm e^{\mathrm i \frac{2\pi}{m}}$,则 $1, \xi, \ldots, \xi^{m - 1}$ 是所有的 $m$ 次单位根,根据因式分解可得:
	$$
	(\lambda I - A)(\lambda I - \xi A)(\lambda I - \xi^2 A) \cdots (\lambda I - \xi^{m - 1} A) = \lambda^m I - A^m
	$$$$
	(\lambda - \lambda_i)(\lambda - \xi \lambda_i)(\lambda - \xi^2 \lambda_i) \cdots (\lambda - \xi^{m - 1} \lambda_i) = \lambda^m - \lambda_i^m
	$$

	同上例,相乘可得:
	$$
	|\lambda^m I - A^m| = (\lambda^m - \lambda_1^m) \cdots (\lambda^m - \lambda_s^m)
	$$

	同上例,代入可得:
	$$
	|\lambda I - A^m| = (\lambda - \lambda_1^m) \cdots (\lambda - \lambda_s^m)
	$$
\end{proof}