% Licensed under the Creative Commons Attribution Share Alike 4.0 International.
% See the LICENSE file in the repository root for full license text.

\section{商集构成的环,域上的一元多项式环}

前面我们已经研究了数域 $\mathbb K$ 上的一元多项式环 $\mathbb K[x]$,但我们时常需要面对更一般的情况。例如,前面研究了整系数多项式组成的集合 $\mathbb Z[x]$,已经说明,它构成一个环,所以它是整环 $\Z$ 上的一元多项式环。

本节中,我们将使用商集构造出一些常见的环,并说明它们是域,然后通过研究域上的一元多项式环来得出一些实用结论。

\subsection{商集和域的概念}

商集与等价关系有关。如果集合 $S$ 中的两个元素存在某种等价关系,那么它们属于同一个等价类。我们想要研究不同等价类之间的性质,就需要先弄清楚到底有哪些等价类。商集就是所有这些等价类组成的集合。

\begin{definition}{商集}
	设 $\sim$ 是非空集合 $A$ 的一个等价关系,若把以 $A$ 关于 $\sim$ 的全部等价类作为元素组成一个新的集合 $B$,则称集合 $B$ 为 $A$ 关于 $\sim$ 的\emph{商集},记作 $B = A / {\sim}$。
\end{definition}

\bigskip

前面我们定义了环。环有加法、乘法,由负元还可以推导出减法,但环上不存在除法。域弥补了环的这一缺陷。

\begin{definition}{可逆的,逆元}
	设 $R$ 是有单位元的环,若 $a \in R$ 且 $a \ne 0$,且存在 $b \in R$ 使得 $ab = ba = 1$,则称 $a$ 是\emph{可逆的},称 $b$ 是 $a$ 的\emph{逆元}。
\end{definition}

\begin{proposition}
	若环 $R$ 中的元素 $a$ 存在逆元,则这个逆元是唯一的。
\end{proposition}

\begin{proof}
	若 $a$ 存在两个不同的逆元 $b, b'$,则有:
	$$
	ab = ba = ab' = ba' = 1
	$$

	在 $ab = ab'$ 的两侧同时左乘以 $b$,得:
	$$
	(ba) b = (ba) b' \Longrightarrow b = b'
	$$

	这说明 $a$ 的逆元若存在,则是唯一的。
\end{proof}

\begin{definition}{域}
	设 $R$ 是有单位元的无非平凡零因子的交换环(即 $R$ 是整环),若 $R$ 中的任一非零元素都可逆,则称 $R$ 是\emph{域}。
\end{definition}

\begin{proposition}
	如果已知 $R$ 是有单位元的交换环,且 $R$ 中的任一非零元素都可逆,则 $R$ 一定是域。
\end{proposition}

\begin{proof}
	假设 $R$ 存在非平凡的零因子 $a$(不妨假设 $a$ 是左零因子),则存在非零元素 $b$ 使得:
	$$
	a b = 0
	$$

	由于 $a$ 可逆,所以存在非零元素 $a^{-1}$ 使得 $a^{-1} a = a a^{-1} = 1$,于是等式两侧左乘 $a^{-1}$ 得:
	$$
	b = (a^{-1} a) b = 0
	$$

	这与 $b \ne 0$ 矛盾,故不存在非平凡的零因子,从而根据已知条件可以推出 $R$ 是域。
\end{proof}

\begin{definition}{有限域,无限域}
	只含有限多个元素的域被称为\emph{有限域},否则称为\emph{无限域}。
\end{definition}

\subsection{模 $p$($p$ 是素数)剩余类域与模 $m$ 剩余类环}

\begin{definition}{同余关系,剩余类}
	在整数集 $\Z$ 上规定一个二元关系 $\sim$ 为:设 $a, b \in \Z$,若 $a$ 与 $b$ 被 $m$ 除所得余数相同,即 $m \mid (a - b)$,则有 $a \sim b$。容易发现 $\sim$ 是一个等价关系,称这个关系为\emph{模 $m$ 同余关系},记作:
	$$
	a \equiv b \pmod{m}
	$$

	在模 $m$ 同余关系下的等价类称为\emph{模 $m$ 剩余类},它们被记作:
	$$
	\overline i = \set{a \in \Z \colon a \equiv i \pmod{m}} = \set{mk + i \colon k \in \Z} \pod{i = 0, 1, \ldots, m - 1}
	$$

	并且规定 $\overline i$ 中的任一元素都可以作为代表,例如,$\overline{i + m}$ 也可用于表示 $\overline i$。
\end{definition}

\begin{proposition}
	若 $a \equiv b \pmod{m}, c \equiv d \pmod{m}$,则:
	$$
	a + c \equiv b + d \pmod{m}
	$$$$
	ac \equiv bd \pmod{m}
	$$
\end{proposition}

\begin{proof}
	已知 $m \mid (a - b), m \mid (c - d)$,则显然有:
	$$
	m \mid (a - b) + (c - d) \Longrightarrow m \mid (a + c) - (b + d)
	$$

	即证得了 $a + c \equiv b + d \pmod{m}$。

	由于有:
	$$
	ac - bd = ac - bc + bc - bd = (a - b) c + b (c - d)
	$$

	所以 $bc \equiv bd \pmod{m}$。
\end{proof}

\begin{definition}{同余关系的商集}
	由模 $m$ 剩余类组成的集合称为\emph{模 $m$ 同余关系的商集},记作 $\Z_m$ 或 $\Z / (7)$。
\end{definition}

在 $\Z_m$ 中可以规定加法和乘法:
\begin{enumerate}
	\item $\overline i + \overline j \triangleq \overline{i + j}$
	\item $\overline i \cdot \overline j \triangleq \overline {ij}$
\end{enumerate}

下面首先证明以上定义的合理性,即定义式的成立与代表的选取无关。

\begin{proof}
	设 $\overline i = \overline a$,$\overline j = \overline b$,则:
	$$
	i \equiv a \pmod{m}, \quad j \equiv b \pmod{m}
	$$

	于是有:
	\begin{enumerate}
		\item $i + j \equiv a + b \pmod{m}$
		\item $ij \equiv ab \pod{m}$
	\end{enumerate}
\end{proof}

显然,$\overline 0$ 是 $\Z_m$ 的零元素,$\overline i$ 有负元素 $\overline{-i}$,$\overline 1$ 是 $\Z_m$ 的单位元,所以 $\Z_m$ 是一个有单位元的环。进一步验证得,\textbf{$\Z_m$ 是一个有单位元的交换环},称它为\emph{模 $m$ \idx{剩余类环}}。

当且仅当 $m$ 是素数时(不妨认为 $\abs{m} \ge 2$,一般把素数 $m$ 记作 $p$),$\Z_p$ 中的每个非零元素都存在逆元,从而 \textbf{$\Z_p$ 构成一个域},称它为\emph{模 $p$ \idx{剩余类域}},它是一个有限域。

\begin{proof}
	必要性。设 $\Z_p$ 中的任一非零元为 $\overline a \pod{0 < a < p}$。由于 $p$ 是素数,且 $p \nmid a$,所以\footnote{若 $p$ 是素数,则对任意整数 $a$,都有 $p \mid a$ 或 $(p, a) = 1$。} $(p, a) = 1$。从而存在 $u, v \in \Z$,使得:
	$$
	ua + vp = 1
	$$

	于是:
	$$
	\overline 1 = \overline{ua + vp} = \overline u \cdot \overline a + \overline v \cdot \overline p = \overline u \cdot \overline a
	$$

	因此 $\overline a$ 可逆,从而 $\Z_p$ 是一个域。

	\bigskip

	充分性,证明逆否命题。若 $p$ 是合数,记它为 $m$,则 $m = m_1 m_2 \pod{0 < m_i < m, i = 1, 2}$,于是:
	$$
	\overline{m_1} \cdot \overline{m_2} = \overline{m_1 m_2} = \overline m = \overline 0
	$$

	从而 $Z_m$ 有非平凡的零因子 $\overline{m_1}$,因此 $\Z_p$ 不是一个域。
\end{proof}

\subsection{$\Q$ 与 $\Z_p$ 上的不可约多项式之间的关系}

\subsubsection{域上的一元多项式环}

我们可以完全仿照 $\mathbb K[x]$ 的概念定义出域上的一元多项式环 $\mathbb F[x]$。问题是,哪些 $\mathbb K[x]$ 中的结论对于 $\mathbb F[x]$ 同样适用?

事实上,只要 $\mathbb K[x]$ 的结论在证明中没有用到这个域含有无穷多个元素这一条件,并且注意识别域中的零元素,那么这些结论对于任一域 $\mathbb F$ 上的一元多项式环 $\mathbb F[x]$ 仍然成立。下面给出两个反例:
\begin{enumerate}
	\item 在有限域 $\mathbb F$ 上的一元多项式环 $\mathbb F[x]$ 中,如果两个多项式不相等,那么它们诱导的多项式函数可能相等。这与在数域 $\mathbb K$ 中的情况不同。

	\begin{example}
		在 $\Z_3[x]$ 中,设:
		$$
		f(x) = x^3 + \overline 2 x^2 + \overline 2, \quad g(x) = \overline 2 x^2 + x + \overline 2
		$$

		由于:
		$$
		f(\overline 0) = \overline 2, \quad f(\overline 1) = \overline 2, \quad f(\overline 2) = \overline 0
		$$$$
		g(\overline 0) = \overline 2, \quad g(\overline 1) = \overline 2, \quad g(\overline 2) = \overline 0
		$$

		所以 $f(x)$ 和 $g(x)$ 诱导的多项式函数是相等的,但 $f(x) \ne g(x)$。
	\end{example}

	\item 在数域 $\mathbb K$ 上的一元多项式环 $\mathbb K[x]$ 中,如果不可约多项式 $p(x)$ 是 $f(x)$ 的一个 $k \pod{k \ge 1}$ 重因式,那么 $p(x)$ 是 $f'(x)$ 的 $k - 1$ 重因式。而对于某些域 $\mathbb F$ 来说,可以证明:在 $\mathbb F[x]$ 中,若 $(f(x), f'(x)) = 1$,则 $f(x)$ 没有重因式;若 $f(x)$ 没有重因式,则 $(f(x), f'(x)) = 1$ 或者 $f(x)$ 有一个单因式 $p(x)$,使得 $p'(x) = 0$。
\end{enumerate}

\subsubsection{判断 $\Q$ 上的多项式不可约的又一充分条件}

前面已经说明:
$$
\begin{aligned}&
	\text{有理系数多项式在 $\Q$ 上不可约}
	\\\Longleftrightarrow~&
	\text{与其相伴的本原多项式在 $\Z$ 上不可分解(往上时乘以任一有理数)}
	\\\Longleftrightarrow~&
	\text{与该本原多项式在 $\Q[x]$ 中相伴的任一整系数多项式在 $\Q$ 上不可约}
	\\\Longleftarrow~&
	\text{Eisenstein 判别法}
\end{aligned}
$$

下面的定理给出了判断整系数多项式在 $\Q$ 上不可约的又一充分条件。

\begin{proposition}
	设 $f(x) = a_n x^n + a_{n - 1} x^{n - 1} + \cdots a_1 x + a_0$ 是一个整系数多项式,$p$ 是一个素数,$p \nmid a_n$。把 $f(x)$ 的各项系数模 $p$ 变成 $\Z_p$ 的元素,得到 $\Z_p$ 上的一个多项式,记作 $\tilde f(x)$,即:
	$$
	\tilde f(x) = \overline{a_n} x^n + \overline{a_{n - 1}} x^{n - 1} + \cdots + \overline{a_1} x + \overline{a_0}
	$$

	如果 $\tilde f(x)$ 在 $\Z_p$ 上不可约,那么 $f(x)$ 在 $\Q$ 上不可约。
\end{proposition}

\begin{proof}
	假设 $f(x)$ 在 $\Q$ 上可约,那么存在次数较低的两个整系数多项式 $f_1(x), f_2(x)$,使得 $f(x) = f_1(x) + f_2(x)$。设:
	$$
	f_1(x) = \sum\limits_{i = 0}^m b_i x^i, \quad f_2(x) = \sum\limits_{j = 0}^{n - m} c_j x^j
	$$

	显然有:
	$$
	\tilde f(x) = \tilde f_1(x) \tilde f_2(x)
	$$

	由于 $p \nmid a_n$,$a_n = b_m c_{n - m}$,所以 $p \nmid b_m$ 且 $p \nmid c_{n - m}$,从而 $\deg \tilde f_1(x) = \deg f_1(x)$,$\deg \tilde f_2(x) = \deg f_2(x)$。又因为 $\deg \tilde f(x) = \deg f(x)$,所以:
	$$
	\deg \tilde f_i(x) = \deg f_i(x) < \deg f(x) = \deg \tilde f(x) \pod{i = 1, 2}
	$$

	这意味着 $\tilde f(x)$ 在 $\Z_p$ 上可约,矛盾。
\end{proof}

这意味着我们还可以用以下方法来说明有理系数多项式在 $\Q$ 上不可约:
$$
\begin{aligned}&
	\text{有理系数多项式在 $\Q$ 上不可约}
	\\\Longleftrightarrow~&
	\text{与其相伴的本原多项式在 $\Z$ 上不可分解(往上时乘以任一有理数)}
	\\\Longleftrightarrow~&
	\text{与该本原多项式在 $\Q[x]$ 中相伴的任一整系数多项式在 $\Q$ 上不可约}
	\\\Longleftarrow~&
	\text{与其对应的 $\Z_p[x]$ 中的多项式在 $\Z_p$ 上不可约}
\end{aligned}
$$

但需要注意,以上两条路线在最后一个箭头处都是单向的。如果 $\tilde f(x)$ 在 $\Z_p$ 上可约,那么 $f(x)$ 在 $\Q$ 上可能不可约,也可能可约。

\subsection{模 $f(x)$ 剩余类域}

