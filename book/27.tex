% Licensed under the Creative Commons Attribution Share Alike 4.0 International.
% See the LICENCE file in the repository root for full licence text.

\section{正定二次型}

\subsection{正定二次型与正定矩阵}

对于欧式几何而言,在 $\R^n$ 上定义内积,要求内积函数满足对称性、双线性、正定性。我们已经学习了标准内积:
$$
(\vec \alpha, \vec \beta) = a_1 b_1 + \cdots + a_n b_n
$$
自然要问,是否存在其他内积函数?

可以发现,内积函数被定义为对称的二重线性函数。所以要研究上面的问题,只需要研究对称的二重线性函数的正定性,即只需要研究二次型的取值情况。

\begin{definition}{正定的实二次型,正定}
	设 $n$ 元实二次型 $\vec X^T A \vec X$。如果对于 $\R^n$ 中任意非零列向量 $\vec \alpha$,都有 $\vec \alpha^T A \vec \alpha > 0$,则称 $\vec X^T A \vec X$ 为\emph{正定的实二次型},称 $\vec X^T A \vec X$ 是\emph{正定}的。
\end{definition}

联想到实二次型的标准形和规范形,显然有以下结论成立。

\begin{theorem}
	$n$ 元实二次型 $\vec X^T A \vec X$ 是正定的,当且仅当它的正惯性指数等于 $n$。
\end{theorem}

\begin{proof}
	必要性。设 $\vec X^T A \vec X$ 是正定的,作非退化线性替换 $\vec X = C \vec Y$,化成规范形:
	$$
	y_1^2 + \cdots + y_p^2 - y_{p + 1}^2 - \cdots - y_r^2
	$$

	如果 $p < n$,那么 $y_n^2$ 的系数为 $0$ 或 $-1$,取 $\vec \beta = (0, \ldots, 0, 1)'$,令 $\vec \alpha = C \vec  \beta$,则 $\vec \alpha^T A \vec \alpha = 0$ 或 $-1$,矛盾,因此 $p = n$。

	\bigskip

	充分性。设 $\vec X^T A \vec X$ 的正惯性指数等于 $n$,则可以作非退化线性替换 $\vec X = C \vec Y$,化成规范形:
	$$
	y_1^2 + \cdots + y_n^2
	$$

	任取 $\vec \alpha \in \R^n$ 且 $\vec \alpha \ne \vec 0$,令 $\vec \beta = C^{-1} \vec \alpha = (b_1, \ldots, b_n)'$,则 $\vec \beta \ne \vec 0$,从而得出 $\vec \alpha^T A \vec \alpha = b_1^2 + \cdots + b_n^2 > 0$,因此 $\vec X^T A \vec X$ 是正定的。
\end{proof}

\begin{theorem}
	$n$ 元实二次型 $\vec X^T A \vec X$ 是正定的 $\Longleftrightarrow$ 它的规范形为 $y_1^2 + y_2^2 + \cdots + y_n^2$ $\Longleftrightarrow$ 它的标准形中 $n$ 个系数全大于 $0$。
\end{theorem}

\begin{definition}{正定的实对称矩阵,正定,正定矩阵}
	设实对称矩阵 $A$。如果实二次型 $\vec X^T A \vec X$ 是正定的,即对于 $\R^n$ 中任意非零列向量 $\vec \alpha$,有 $\vec \alpha^T A \vec \alpha > 0$,则称 $A$ 是\emph{正定的实对称矩阵},称 $A$ 是\emph{正定}的。正定的实对称矩阵简称为\emph{正定矩阵}。
\end{definition}

\subsection{正定矩阵的等价命题}

联想到矩阵的合同、合同标准形、合同规范形等概念,可以提出以下有关正定矩阵的一系列等价命题。

设 $A$ 是 $n$ 级\textbf{实对称矩阵},则下列命题等价:

\begin{enumerate}
	\item 正定矩阵的定义:$\forall \vec X \in \R^n \backslash \{\vec 0\}, \vec X^T A \vec X > 0$
	\item $A \simeq I_n$
	\item $A$ 的正惯性指数为 $n$。
	\item 存在一个正交矩阵 $P$ 使得 $P^T A P = \operatorname{diag}\{ \lambda_1, \ldots, \lambda_n \}$,其中 $\lambda_i > 0$,且分别为 $A$ 的所有特征值。
	\item 有 $n$ 级实可逆矩阵 $C$ 使得 $A = C^T C$。
	\item $A$ 的各阶顺序主子式($n$ 级矩阵有 $n$ 个顺序主子式)全大于 $0$。
	\item $A$ 的各阶主子式大于零。
\end{enumerate}

1, 2, 3 的等价是显然的,1, 2, 3, 4, 5 的等价是容易证明的。

\begin{proof}[1, 2, 3 $\Longrightarrow$ 4]
	由于 $A \simeq I_n$ 且 $A \simeq \operatorname{diag} \{ \lambda_1, \cdots, \lambda_n \}$,所以 $I_n \simeq \operatorname{diag} \{ \lambda_1, \cdots, \lambda_n \}$。于是 $\lambda_i > 0$。
\end{proof}

\begin{proof}[4 $\Longrightarrow$ 2]
	由于 $A \simeq \operatorname{diag} \{ \lambda_1, \cdots, \lambda_n \}$ 且 $I_n \simeq \operatorname{diag} \{ \lambda_1,  \cdots, \lambda_n \}$,所以 $A \simeq I_n$。
\end{proof}

\begin{proof}[1, 2, 3, 4 $\Longleftrightarrow$ 5]
	由 $A \simeq I_n$,可知存在 $n$ 级实可逆矩阵 $C$ 使得 $A = C^T I_n C$,则 $A = C^T C$。反过来,可知 $A$ 合同于 $I_n$。
\end{proof}

较为麻烦的是对 6 和 7 的证明。注意下面证明过程中用到的技巧。

\begin{proof}[1, 2, 3, 4, 5 $\Longrightarrow$ 6]

\end{proof}

