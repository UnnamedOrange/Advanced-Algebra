% Licensed under the Creative Commons Attribution Share Alike 4.0 International.
% See the LICENCE file in the repository root for full licence text.

\section{线性映射的核与象空间}

我们继续研究线性映射。首先提出两个非常重要的概念。

\begin{definition}{核,零空间}
	设 $\sigma$ 是 $\mathbb K^n$ 到 $\mathbb K^s$ 的一个映射,称 $\mathbb K^n$ 的一个子集:
	$$
	\{ \vec \alpha \in \mathbb K^n \colon \sigma (\vec \alpha) = \vec 0 \}
	$$
	为映射 $\sigma$ 的\emph{核},记作 $\operatorname{Ker} \sigma$。如果 $\sigma$ 是 $\mathbb K^n$ 到 $\mathbb K^s$ 的一个线性映射,那么 $\operatorname{Ker} \sigma$ 是 $\mathbb K^n$ 的一个子空间,此时也称 $\operatorname{Ker} \sigma$ 为 $\sigma$ 的\emph{零空间}。
\end{definition}

可知,线性映射的核等于齐次线性方程组 $A \vec x = \vec 0$ 的解空间。

\begin{definition}{象空间}
	称线性映射 $\phi$ 的象(值域)为它的\emph{象空间},记为 $\operatorname{Im} \phi$。
\end{definition}

可知,线性映射的象空间即为其对应矩阵的列向量张成的子空间。

关于核与象空间,我们不加证明地给出以下定理。

\begin{theorem}
	$$
	\dim \operatorname{Ker} \phi = n - \operatorname{rank}(A)
	$$$$
	\dim \operatorname{Im} \phi = \operatorname{rank}(A)
	$$

	其中 $A$ 是线性映射 $\phi$ 对应的矩阵,$n$ 是 $A$ 的列数。
\end{theorem}

\begin{theorem}
	$$
	\dim \operatorname{Ker} \phi + \dim \operatorname{Im} \phi = \dim \mathbb K^n
	$$

	其中 $n$ 是线性映射 $\phi$ 对应的矩阵的列数。
\end{theorem}

\begin{theorem}
	设 $\phi \colon \mathbb K^m \to \mathbb K^n$ 是一个线性映射,则 $\phi$ 是单射的充分必要条件是 $\operatorname{Ker} \phi = \{ \vec 0 \}$;$\phi$ 是满射的充分必要条件是 $\operatorname{Im} \phi = \mathbb K^n$。
\end{theorem}

\begin{theorem}
	设 $\phi \colon \mathbb K^n \to \mathbb K^n$ 是一个线性映射,则 $\phi$ 是单射的充分必要条件是 $\phi$ 是满射。
\end{theorem}

核与象空间的概念是深刻的。其运用举例暂时略去。

% TODO: 核与象空间的应用举例。