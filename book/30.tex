% Licensed under the Creative Commons Attribution Share Alike 4.0 International.
% See the LICENSE file in the repository root for full license text.

\section{最大公因式}

\subsection{公因式的概念}

\begin{definition}{公因式}
	在 $\mathbb K[x]$ 中,若 $c(x) \mid f(x)$ 且 $c(x) \mid g(x)$,则称 $c(x)$ 是 $f(x)$ 与 $g(x)$ 的\emph{公因式}。
\end{definition}

\begin{definition}{最大公因式}
	$\mathbb K[x]$ 中多项式 $f(x)$ 与 $g(x)$ 的一个公因式 $d(x)$ 如果满足下述条件:对于 $f(x)$ 与 $g(x)$ 的任一公因式 $c(x)$,都有 $c(x) \mid d(x)$,那么称 $d(x)$ 是 $f(x)$ 与 $g(x)$ 的一个\emph{最大公因式}。
\end{definition}

根据以上定义可知,$f(x)$ 是 $f(x)$ 与 $0$ 的一个最大公因式。特别地,$0$ 是 $0$ 与 $0$ 的最大公因式。对于不全为 $0$ 的两个多项式,它们的最大公因式是次数最高的公因式。注意不能直接把最大公因式定义为次数最高的公因式,否则无法确定 $0$ 与 $0$ 的最大公因式,因为任一多项式都是 $0$ 与 $0$ 的公因式。

容易看出,如果 $f(x)$ 与 $g(x)$ 的最大公因式存在,那么 $f(x)$ 与 $g(x)$ 的任意两个最大公因式 $d_1(x)$ 与 $d_2(x)$ 是相伴的。如果 $f(x)$ 与 $g(x)$ 不全为 $0$,那么它们的最大公因式不是 $0$,于是用 $(f(x), g(x))$ 或者 $\gcd(f(x), g(x))$ 表示首项系数为 $1$ 的最大公因式,简称为 $f(x)$ 和 $g(x)$ 的\emph{\idx{首一最大公因式}}。

\subsection{求最大公因式}

首先给出一个引理。

\begin{proposition}
	在 $\mathbb K[x]$ 中,如果等式 $f(x) = h(x) g(x) + r(x)$ 成立,那么有:
	$$
	\set{\text{$f(x)$ 与 $g(x)$ 的最大公因式}} = \set{\text{$g(x)$ 与 $r(x)$ 的最大公因式}}
	$$
\end{proposition}

\begin{proof}
	设 $c(x)$ 是 $f(x)$ 与 $g(x)$ 的公因式,则 $c(x) \mid f(x) - h(x) g(x)$,从而 $c(x) \mid r(x)$。于是 $c(x)$ 是 $g(x)$ 与 $r(x)$ 的公因式。

	设 $u(x)$ 是 $g(x)$ 与 $r(x)$ 的公因式,则 $u(x) \mid h(x) g(x) + r(x)$,从而 $u(x) \mid f(x)$。于是 $u(x)$ 是 $f(x)$ 与 $g(x)$ 的公因式。

	根据集合相等的常用证明方法,易知它们的公因式集合相等,从而它们的最大公因式集合相等。
\end{proof}

由于 $\mathbb K[x]$ 中有除法算式,所以由以上引理,可以使用\emph{\idx{辗转相除法}}求最大公因式。由辗转相除法,我们可以得到以下定理。

\begin{proposition}
	对于 $\mathbb K[x]$ 中任意两个多项式 $f(x)$ 与 $g(x)$,一定存在它们的一个最大公因式 $d(x)$,并且存在 $u(x), v(x) \in \mathbb K[x]$,使得:
	$$
	d(x) = u(x) f(x) + v(x) g(x)
	$$
\end{proposition}

\begin{proof}
	如果 $g(x) = 0$,那么 $f(x)$ 就是 $f(x)$ 与 $g(x)$ 的一个最大公因式,并且 $f(x) = 1 \cdot f(x) + 1 \cdot 0$。下面设 $f(x) \ne 0$ 且 $g(x) \ne 0$。设 $n = \min \{ \deg f(x), \deg g(x) \}$,对 $n$ 作数学归纳法。当 $n = 0$ 时,显然有 $f(x)$ 是它们的最大公因式,并且 $f(x) = 1 \cdot f(x) + 0 \cdot g(x)$。假设 $n \le k$ 时正确,下面证明 $n = k + 1$ 时正确。不妨设 $\deg f(x) \ge \deg g(x)$,作带余除法,有:
	$$
	f(x) = q(x) g(x) + r(x)
	$$

	由引理可知,$(f(x), g(x)) = (g(x), r(x))$,又有 $\min\{\deg g(x), \deg r(x)\} < k + 1$,于是由归纳假设,存在性得证。由归纳假设,有:
	$$
	\begin{aligned}
		(g(x), r(x)) &= g(x) u_1(x) + r(x) v_1(x)
		\\&=
		g(x) u_1(x) + (f(x) - q(x) g(x)) v_1(x)
		\\&=
		v_1(x)f(x) + (u_1(x) - q(x) v_1(x)) g(x)
		\\&=
		(f(x), g(x))
	\end{aligned}
	$$

	故命题对 $n = k + 1$ 成立。
\end{proof}

\subsection{互素和有限个多项式的最大公因式}

\begin{definition}{互素}
	设 $f(x), g(x) \in \mathbb K[x]$,如果 $(f(x), g(x)) = 1$,那么称 $f(x)$ 与 $g(x)$ \emph{互素}。
\end{definition}

\begin{theorem}[互素的充分必要条件]
	$\mathbb K[x]$ 中两个多项式 $f(x)$ 与 $g(x)$ 互素的充分必要条件是:存在 $u(x), v(x) \in \mathbb K[x]$,使得 $u(x) f(x) + v(x) g(x) = 1$。
\end{theorem}

\begin{proof}
	必要性显然,下面证充分性。对于 $f(x)$ 和 $g(x)$ 的任意公因式 $c(x)$,可以由该式推出 $c(x) \mid 1$,从而 $c(x)$ 是 $\mathbb K$ 中的非零数。因此 $f(x)$ 与 $g(x)$ 互素。
\end{proof}

在互素定义的基础上,可以得到以下性质。

\begin{proposition}
	设 $f(x), g(x) \in \mathbb K[x]$,数域 $\mathbb F \supseteq \mathbb K$,则 $f(x)$ 与 $g(x)$ 在 $\mathbb K[x]$ 中的首一最大公因式等于它们在 $\mathbb F[x]$ 中的首一最大公因式。即 $f(x)$ 与 $g(x)$ 的首一最大公因式不随数域的扩大而改变。
\end{proposition}

\begin{proof}
	由于 $\mathbb K \subseteq \mathbb F$,所以 $f(x)$ 与 $g(x)$ 在 $\mathbb K[x]$ 中作辗转相除法可看成它们在 $\mathbb F[x]$ 中作辗转相除法,从而它们的首一最大公因式相等。
\end{proof}

\begin{proposition}
	在 $\mathbb K[x]$ 中,如果 $f(x) \mid g(x) h(x)$,且 $(f(x), g(x)) = 1$,那么 $f(x) \mid h(x)$。
\end{proposition}

\begin{proof}
	由于 $(f(x), g(x)) = 1$,因而存在 $u(x), v(x) \in \mathbb K[x]$,使得 $u(x) f(x) + v(x) g(x) = 1$,从而:
	$$
	u(x) f(x) h(x) + v(x) g(x) h(x) = h(x)
	$$

	由于 $f(x) \mid g(x) h(x)$,所以 $f(x)$ 整除左式,进而 $f(x) \mid h(x)$。
\end{proof}

\begin{proposition}
	在 $\mathbb K[x]$ 中,如果 $f(x) \mid h(x)$,$g(x) \mid h(x)$,且 $(f(x), g(x)) = 1$,那么 $f(x) g(x) \mid h(x)$。
\end{proposition}

\begin{proof}
	由于 $f(x) \mid h(x)$,所以存在 $p(x) \in \mathbb K[x]$,使得 $h(x) = p(x) f(x)$,从而 $g(x) \mid p(x) f(x)$。由于 $(g(x), f(x)) = 1$,所以 $g(x) \mid p(x)$。从而存在 $q(x) \in \mathbb K[x]$,使得 $p(x) = q(x) g(x)$。所以 $h(x) = q(x) g(x) f(x)$,因此 $f(x) g(x) \mid h(x)$。
\end{proof}

\begin{proposition}
	在 $\mathbb K[x]$ 中,如果 $(f(x), h(x))= 1, (g(x), h(x)) = 1$,那么 $(f(x) g(x), h(x)) = 1$。
\end{proposition}

\begin{proof}
	已知存在 $u(x), v(x), w(x), p(x) \in \mathbb K[x]$,使得:
	$$
	u(x) f(x) + v(x) h(x) = 1
	$$$$
	w(x) g(x) + p(x) h(x) = 1
	$$

	两式相乘后整理得:
	$$
	\begin{aligned}&
		\bigl( u(x) w(x) \bigr) f(x) g(x)
		\\+~&
		\bigl( u(x) f(x) p(x) + v(x) w(x) g(x) + v(x) h(x) p(x) \bigr) h(x) = 1
	\end{aligned}
	$$

	从而\footnote{互素的充分必要条件:$\mathbb K[x]$ 中两个多项式 $f(x)$ 与 $g(x)$ 互素的充分必要条件是:存在 $u(x), v(x) \in \mathbb K[x]$,使得 $u(x) f(x) + v(x) g(x) = 1$。} $(f(x) g(x), h(x)) = 1$。
\end{proof}

不断运用该性质,可以推广得到下面这个性质。

\begin{proposition}
	在 $\mathbb K[x]$ 中,若 $\forall i \in \set{1, \ldots, n}, j \in \set{1, \ldots, m}, (f_i(x), g_j(x)) = 1$,则:
	$$
	\biggl( \prod\limits_{i = 1}^n f_i(x), \prod\limits_{j = 1}^n g_j(x) \biggr) = 1
	$$
\end{proposition}

\begin{proposition}
	在 $\mathbb K[x]$ 中,若 $f(x) \ne 0, g(x) \ne 0, d(x) = (f(x), g(x))$,则:
	$$
	\biggl( \dfrac{f(x)}{d(x)}, \dfrac{g(x)}{d(x)} \biggr) = 1
	$$
\end{proposition}

\begin{proof}
	设 $f(x) = d(x) f_1(x)$,$g(x) = d(x) g_1(x)$,则有:
	$$
	d(x) = u(x) d(x) f_1(x) + v(x) d(x) g_1(x)
	$$

	消去即得 $u(x) f_1(x) + v(x) g_1(x) = 1$,则 $(f_1(x), g_1(x)) = 1$。
\end{proof}

当多项式多于两个时,我们也可以为它们定义最大公因式、互素。

\begin{definition}{最大公因式}
	设 $f_1(x), \ldots, f_s(x) \in \mathbb K[x]$,若 $d(x) \mid f_i(x) \pod{i = 1, \ldots, s}$,且 $\forall h(x) \mid f_i(x) \pod{i = 1, \ldots, s}$,都有 $h(x) \mid d(x)$,则称 $d(x)$ 为 $f_1, \ldots, f_s$ 的\emph{最大公因式}。
\end{definition}

记 $(f_1(x), \ldots, f_s(x))$ 为 $f_1(x), \ldots, f_s(x)$ 的\emph{\idx{首一最大公因式}}。

\begin{proposition}
	在 $\mathbb K[x]$ 中,$(f_1, \ldots, f_s) = \bigl( (f_1, \ldots, f_{s - 1}), f_s \bigr)$。
\end{proposition}

\begin{proof}[只证明 $s = 3$ 的情况,其他情况可将其改为数学归纳法证明]
	设 $d(x) = \bigl( (f_1, f_2), f_3 \bigr)$,$d_1(x) = (f_1, f_2)$,则要证 $d(x) = (f_1, f_2, f_3)$。

	可知 $d(x) \mid f_1(x)$ 且 $d(x) \mid f_2(x)$,因此 $d(x)$ 是 $f_1, f_2, f_3$ 的公因式,有 $d(x) \mid (f_1, f_2, f_3)$。任取 $f_1, f_2, f_3$ 的一个公因式 $c(x)$,则 $c(x) \mid d_1(x)$ 且 $c(x) \mid f_3(x)$,于是 $c(x)$ 是 $d_1(x)$ 和 $f_3(x)$ 的公因式,有 $c(x) \mid d(x)$,从而 $d(x)$ 是 $f_1, f_2, f_3$ 的一个最大公因式。
\end{proof}

于是可得以下定理。

\begin{proposition}
	在 $\mathbb K[x]$ 中存在多项式 $u_i(x) \pod{i = 1, \ldots, s}$,使得:
	$$
	\sum\limits_{i = 1}^s u_i(x) f_i(x) = (f_1, \ldots, f_s)
	$$
\end{proposition}

\begin{definition}{互素}
	$\mathbb K[x]$ 中,$s$ 个多项式 $f_1(x), f_2(x), \ldots, f_s(x)$ 如果满足:
	$$
	\bigl( f_1(x), f_2(x), \ldots, f_s(x) \bigr) = 1
	$$

	那么称 $f_1(x), \ldots, f_s(x)$ \emph{互素}。
\end{definition}

多个多项式互素的充分必要条件与两个多项式的类似。

\begin{proposition}[互素的充分必要条件]
	在 $\mathbb K[x]$ 中,$f_1(x), \ldots, f_s(x)$ 互素的充分必要条件是:存在 $u_1(x), u_2(x), \ldots, u_s(x) \in \mathbb K[x]$,使得:
	$$
	\sum\limits_{i = 1}^s u_i(x) f_i(x) = 1
	$$
\end{proposition}

需要注意,大于两个的多项式互素时,它们不一定两两互素。

\subsection{在整数环 $\Z$ 上讨论最大公因数和互素}

在整数环 $\Z$ 中,也可以类似地讨论最大公因数和互素的概念,以及相关的性质。下面不加证明地给出这些概念和性质。

\begin{definition}{最大公因数}
	设整数 $a$ 与 $b$ 的一个公因数为 $d$。如果 $a$ 与 $b$ 的任一公因数 $c$ 都能整除 $d$,那么称 $d$ 是 $a$ 与 $b$ 的一个\emph{最大公因数},\textbf{其绝对值}记作 $(a, b)$ 或 $\gcd(a, b)$。
\end{definition}

若 $d_1, d_2$ 都是整数 $a$ 与 $b$ 的最大公因数,则 $d_1 = \pm d_2$。若 $a$ 与 $b$ 不全为 $0$,则 $a$ 与 $b$ 的最大公因数恰有两个,它们互为相反数。

\begin{proposition}
	任给两个整数 $a$ 与 $b$,都存在它们的一个最大公因数 $d$,并且存在整数 $u, v$,使得:
	$$
	ua + vb = d
	$$
\end{proposition}

\begin{definition}{互素}
	设 $a, b \in \Z$,如果 $(a, b) = 1$,那么称 $a$ 与 $b$ \emph{互素}。
\end{definition}

\begin{proposition}[互素的充分必要条件]
	两个整数 $a$ 与 $b$ 互素当且仅当存在 $u, v \in \Z$,使得:
	$$
	ua + vb = 1
	$$
\end{proposition}

\begin{proposition}
	在 $\Z$ 中:
	\begin{enumerate}
		\item 若 $a \mid bc$ 且 $(a, b) = 1$,则 $a \mid c$。
		\item 若 $a \mid c$,$b \mid c$ 且 $(a, b) = 1$,则 $ab \mid c$。
		\item 若 $(a, c) = 1, (b, c) = 1$,则 $(ab, c) = 1$。
		\item 若 $(a_i, c) = 1 \pod{i = 1, 2, \ldots, s}$,则 $(a_1 a_2 \cdots a_i, c) = 1$。
	\end{enumerate}
\end{proposition}

\begin{definition}{最大公因数}
	在 $\Z$ 中,设 $a_1, a_2, \ldots, a_s$ 的一个公因数为 $d$。如果 $a_1, a_2, \ldots, a_s$ 的任一公因数 $c$ 都能整除 $d$,那么称 $d$ 是 $a_1, a_2, \ldots, a_s$ 的一个最大公因数。\textbf{其绝对值}记为 $(a_1, a_2, \ldots, a_s)$ 或 $\gcd(a_1, a_2, \ldots, a_s)$。
\end{definition}

同样,不全为 $0$ 的整数 $a_1, a_2, \ldots, a_s$ 的最大公因数恰有两个,它们互为相反数。

\begin{proposition}
	对于不全为 $0$ 的整数 $a_1, a_2, \ldots, a_s$,有:
	$$
	(a_1, a_2, \ldots, a_s) = \bigl( (a_1, a_2, \ldots, a_{s - 1}), a_s \bigr)
	$$
\end{proposition}

\begin{proposition}
	对于不全为 $0$ 的整数 $a_1, a_2, \ldots, a_s$,存在 $u_1, u_2, \ldots, u_s \in \Z$,使得:
	$$
	u_1 a_1 + u_2 a_2 + \cdots + u_s a_s = (a_1, a_2, \ldots, a_s)
	$$
\end{proposition}

\begin{definition}{互素}
	对于 $s$ 个整数 $a_1, a_2, \ldots, a_s$,如果 $(a_1, a_2, \ldots, a_s) = 1$,那么称 $a_1, a_2, \ldots, a_s$ 互素。
\end{definition}

\begin{proposition}[互素的充分必要条件]
	$a_1, a_2, \ldots, a_s$ 互素当且仅当存在 $u_1, u_2, \ldots, u_s \in \Z$,使得:
	$$
	u_1 a_1 + u_2 a_2 + \cdots + u_s a_s = 1
	$$
\end{proposition}