% Licensed under the Creative Commons Attribution Share Alike 4.0 International.
% See the LICENSE file in the repository root for full license text.

\section{各数域上的不可约多项式}

\subsection{多项式的根的概念}

研究各数域的不可约多项式的动力是,我们已经知道了次数大于 $1$ 的多项式不可约的必要条件是它没有一次因式,但我们不知道它的充分条件。为此,我们需要提出多项式的根的概念。

\begin{theorem}[余数定理]
	在 $\mathbb K[x]$ 中,$f(x)$ 除以 $x - a$ 的余式是 $f(a)$。
\end{theorem}

\begin{proof}
	作带余除法,得:
	$$
	f(x) = h(x) (x - a) + r(x) \pod{\deg r(x) < \deg (x - a)}
	$$

	可知 $\deg r(x) \le 0$,设 $r(x) = r \pod{r \in \mathbb K}$。用 $\mathbb K$ 中的元素 $a$ 代入上面带余除法的等式,即可得:
	$$
	f(a) = r = r(x)
	$$
\end{proof}

\begin{proposition}
	在 $\mathbb K[x]$ 中,$x - a \mid f(x) \Longleftrightarrow f(a) = 0$。
\end{proposition}

由以上命题,我们可以引出多项式的根的概念。

\begin{definition}{根,复根,实根,有理根}
	设 $\mathbb K$ 是数域,环 $R$ 可看成是 $\mathbb K$ 的一个扩环。对于 $f(x) \in \mathbb K[x]$,如果有 $c \in R$,使得 $f(c) = 0$,那么称 $c$ 是 $f(x)$ 在 $R$ 中的一个\emph{根}。

	$f(x)$ 在复数域、实数域、有理数域中的根分别被称为\emph{复根}、\emph{实根}、\emph{有理根}。
\end{definition}

利用多项式的根可以如下描述以上命题。

\begin{proposition}[裴蜀定理]
	在 $\mathbb K[x]$ 中,$x - a$ 是 $f(x)$ 的一次因式当且仅当 $a$ 是 $f(x)$ 在 $\mathbb K$ 中的一个根。
\end{proposition}

\begin{definition}{重根,单根}
	如果 $x - a$ 是 $f(x)$ 的 $k$ 重因式($k \ge 0$),则称 $a$ 是 $f(x)$ 的 \emph{$k$ 重根}。当 $ k \ge 2$ 时,$a$ 被称为\emph{重根}。当 $k = 1$ 时,$a$ 被称为\emph{单根}。当 $k = 0$ 时,$a$ 不是根。
\end{definition}

\subsection{复数域上的不可约多项式}

\begin{theorem}[代数基本定理]
	每一个次数大于 $0$ 的复系数多项式至少有一个复根。
\end{theorem}

代数基本定理的证明比较复杂,此处省略。由代数基本定理可知,每一个次数大于 $1$ 的复系数多项式都是可约的。

\begin{proposition}
	复数域上的不可约多项式只有一次多项式。
\end{proposition}

\begin{theorem}[复系数多项式唯一因式分解定理]
	每一个次数大于 $0$ 的复系数多项式在复数域上都可以唯一地分解成一次因式的乘积。次数大于 $0$ 的复系数多项式 $f(x)$ 的标准分解式记为:
	$$
	f(x) = a (x - c_1)^[l_1] (x - c_2)^{l_2} \cdots (x - c_s)^{l_s}
	$$
\end{theorem}

\begin{proposition}
	每一个 $n \pod{n \ge 0}$ 次复系数多项式恰有 $n$ 个复根(重根按重数计算)。
\end{proposition}

我们特别关注复数域上的多项式 $x^n - 1$ 的根。设它的根为 $\xi$,则有:
$$
\xi = \mathrm e^{\mathrm i \frac{2 \pi}{n}} \pod{i = 0, 1, \ldots, n - 1}
$$

称 $\xi$ 是一个\emph{\idx{本原 $n$ 次单位根}}。

\subsection{多项式插值}

由唯一因式分解定理,不难得到以下命题。

\begin{proposition}
	$\mathbb K[x]$ 中 $n \pod{n \ge 0}$ 次多项式 $f(x)$ 在 $\mathbb K$ 中至多有 $n$ 个根(重根按重数计算)。
\end{proposition}

利用该命题,可以证得多项式插值所依赖的基本命题。

\begin{proposition}
	在 $\mathbb K[x]$ 中,设 $f(x)$ 和 $g(x)$ 的次数都不超过 $n$。如果 $\mathbb K$ 中有 $n + 1$ 个不同的数 $c_1, c_2, \ldots, c_{n + 1}$,使得:
	$$
	f(c_i) = g(c_i) \pod{i = 1, 2, \ldots, n + 1}
	$$

	那么 $f(x) = g(x)$。
\end{proposition}

\begin{proof}
	设 $h(x) = f(x) - g(x)$,则 $\deg h \le \max\{ \deg f, \deg g \} \le n$。由于 $h(x)$ 在 $\mathbb K$ 中至少有 $n + 1$ 个不同的根,而 $\deg h(x) \le n$,所以 $h(x) = 0$,从而 $f(x) = g(x)$。
\end{proof}

由此可知,如果变量 $y$ 与变量 $x$ 之间满足函数关系,并且我们知道当 $x$ 取 $n + 1$ 个不同的值 $c_0, c_1, \ldots, c_n$ 时 $y$ 的对应值为 $d_0, d_1, \ldots, d_n$,那么我们可以找一个次数不超过 $n$ 的多项式 $f(x)$,满足 $f(c_i) = d_i$。这时把这个\textbf{多项式函数} $y = f(x)$ 称为原来函数的\emph{\idx{插值函数}},或\emph{\idx{插值多项式}}。

\subsubsection{多项式函数}

前文提到了多项式函数。注意,我们在定义多项式时,提到 $x$ 是一个不定元,$x \not \in \mathbb K$,这说明\textbf{$\mathbb K[x]$ 中的元素本身并不是一个映射}。但我们总是可以把 $\mathbb K$ 中的元素代入多项式,为此我们特别提出多项式函数的概念。

\begin{definition}{多项式函数, 一元多项式函数}
	任意给定 $f(x) \in \mathbb K[x]$,可以得到 $\mathbb K$ 到自身的一个映射:
	$$
	\begin{aligned}
		f \colon & \mathbb K \to \mathbb K
		\\&
		a \mapsto f(a)
	\end{aligned}
	$$

	其中 $f(a)$ 表示把 $\mathbb K$ 中的元素 $a$ 代入多项式 $f(x)$。则映射 $f$ 被称为由多项式 $f(x)$ 诱导的\emph{多项式函数},也称为 \emph{$\mathbb K$ 上的一元多项式函数}。$\mathbb K[x]$ 中的多项式诱导得出的多项式函数组成的集合记为 $\mathbb K_\mathrm{pol}$。
\end{definition}

可以在 $\mathbb K_\mathrm{pol}$ 上定义朴素的加法和乘法,则 $\mathbb K_\mathrm{pol}$ 构成一个环。称之为 $\mathbb K$ 上的\emph{\idx{一元多项式函数环}}。

一元多项式环和一元多项式函数环有什么关系?我们首先证明以下命题。

\begin{proposition}
	如果数域 $\mathbb K$ 上的两个多项式 $f(x)$ 和 $g(x)$ 不相等,那么它们诱导的多项式函数 $f$ 与 $g$ 也不相等。
\end{proposition}

\begin{proof}
	设 $f(x) \ne g(x)$。假如 $f = g$,则 $\forall a \in \mathbb K$,有 $f(a) = g(a)$。由于 $\mathbb K$ 是数域,它有无穷多个元素,于是\footnote{在 $\mathbb K[x]$ 中,设 $f(x)$ 和 $g(x)$ 的次数都不超过 $n$。如果 $\mathbb K$ 中有 $n + 1$ 个不同的数 $c_1, c_2, \ldots, c_{n + 1}$,使得 $f(c_i) = g(c_i) \pod{i = 1, 2, \ldots, n + 1}$,那么 $f(x) = g(x)$。} $f(x) = g(x)$,矛盾。因此 $f \ne g$。
\end{proof}

显然,一元多项式环和一元多项式函数环之间存在一个双射。但事实上它们之间有更强的联系。

\begin{definition}{同构,同构映射}
	设 $R$ 和 $R'$ 是两个环,如果存在从 $R$ 到 $R'$ 的一个双射 $\sigma$,它保持加法和乘法运算,即 $\forall a, b \in \R$,有:
	$$
	\sigma(a + b) = \sigma(a) + \sigma(b)
	$$$$
	\sigma(ab) = \sigma(a) \sigma(b)
	$$

	那么称 $\sigma$ 是环 $R$ 到 $\R'$ 的一个\emph{同构映射},此时称环 $R$ 与 $R'$ 是\emph{同构}的,记作 $R \cong R'$。
\end{definition}

所以有:
$$
\mathbb K[x] \cong \mathbb K_\mathrm{pol}
$$

这意味着可以把数域 $\mathbb K$ 上的一元多项式 $f(x)$(一个表达式) 与数域 $\mathbb K$ 上的一元多项式函数 $f$(一个映射)等同起来。

\subsubsection{拉格朗日插值公式}

我们首先证明插值多项式的存在性。

\begin{theorem}
	设 $c_0, c_1, \ldots, c_n$ 是数域 $\mathbb K$ 中 $n + 1$ 个不同的数 $d_0, d_1, \ldots, d_n \in \mathbb K$,则 $\mathbb K[x]$ 中存在唯一的一个次数不超过 $n$ 的多项式 $f(x)$,使得:
	$$
	f(c_i) = d_i \pod{i = 0, 1, 2, \ldots, n}
	$$
\end{theorem}

\begin{proof}
	根据前文的定理,如果存在这样的多项式,那它一定是唯一的。下面证明存在性。

	构造函数 $f_i$,使得:
	$$
	f_i(c_j) = 0 \pod{j \ne i}
	$$$$
	f_i(c_i) = d_i
	$$

	于是令:
	$$
	f_i(x) = a_i (x - c_0) \cdots (x - c_{i - 1}) (x - c_{i + 1}) \cdots (x - c_n)
	$$

	则 $f_i(x)$ 是 $n$ 次多项式。为了满足 $f_i(c_i) = d_i$,$a_i$ 应等于:
	$$
	a_i = \dfrac{d_i}{(c_i - c_0) \cdots (c_i - c_{i - 1}) (c_i - c_{i + 1}) \cdots (c_i - c_n)}
	$$

	即有:
	$$
	f_i(x) = d_i \dfrac{(x - c_0) \cdots (x - c_{i - 1}) (x - c_{i + 1}) \cdots (x - c_n)}{(c_i - c_0) \cdots (c_i - c_{i - 1}) (c_i - c_{i + 1}) \cdots (c_i - c_n)}
	$$

	令 $f(x) = \sum\limits_{i = 0}^n f_i(x)$,则 $\deg f(x) \le n$,且满足 $f(c_j) = d_j$。
\end{proof}

以上证明过程得到的公式:
$$
f(x) = \sum\limits_{i = 0}^n d_i \dfrac{(x - c_0) \cdots (x - c_{i - 1}) (x - c_{i + 1}) \cdots (x - c_n)}{(c_i - c_0) \cdots (c_i - c_{i - 1}) (c_i - c_{i + 1}) \cdots (c_i - c_n)}
$$
被称为\emph{\idx{拉格朗日插值公式}}。

\subsubsection{其他插值方法}

设:
$$
\begin{aligned}
	f(x) &= u_0 + u_1 (x - c_0) + u_2 (x - c_0) (x - c_1) + \cdots +
	\\&~~~~
	u_n (x - c_0) (x - c_1) \cdots (x - c_{n - 1})
\end{aligned}
$$

依次用 $c_0, c_1, \ldots, c_n$ 代入 $x$,即可求解出 $u_0, u_1, \ldots, u_n$。以上公式被称为\emph{\idx{牛顿插值公式}}。

\bigskip

还可以使用待定系数法进行插值。由于系数矩阵是一个范德蒙德行列式,所以方程组一定有唯一解。

\subsection{实数域上的不可约多项式}

\subsubsection{实系数多项式唯一因式分解定理}

我们开门见山地给出实系数多项式唯一分解定理,以便于和复系数多项式唯一分解定理进行对比。

\begin{theorem}[实系数多项式唯一分解定理]
	每一个次数大于 $0$ 的实系数多项式 $f(x)$ 在实数域上都可以唯一地分解成一次因式与判别式小于 $0$ 的二次因式的乘积。即:
	$$
	f(x) = a (x - c_1)^r_1 \cdots (x - c_s)^{r_s} (x^2 + p_1 x + q_1)^{k_1} \cdots (x^2 + p_t x + q_t)^{k_t}
	$$

	其中 $a$ 是 $f(x)$ 的首项系数,$c_1, \ldots, c_s$ 是两两不等的实数,$(p_i, q_i) \pod{i = 1, 2, \ldots, t}$ 是不同的实数对,且满足 $p_i^2 - 4 q_i < 0$,$r_1, \ldots, r_s, k_1, \ldots, k_t$ 都是非负整数。
\end{theorem}

证明实系数多项式唯一因式分解定理的关键是以下命题。

\begin{proposition}
	设 $f(x)$ 是实系数多项式,如果 $c$ 是 $f(x)$ 的一个复根,那么 $\overline c$ 也是 $f(x)$ 的一个复根。
\end{proposition}

\begin{proof}
	设 $f(x) = a_n x^n + a_{n - 1} x^{n - 1} + \cdots + a_1 x + a_0 \pod{a_i \in \R, i = 0, 1, \ldots, n}$。由于 $c$ 是 $f(x)$ 的复根,所以有:
	$$
	a_n c^n + a_{n - 1} c^{n - 1} + \cdots + a_1 c + a_0 = 0
	$$

	等式两侧取共轭复数,得:
	$$
	a_n \overline c^n + a_{n - 1} \overline c^{n - 1} + \cdots + a_1 \overline c + a_0 = 0
	$$

	即 $f(\overline c) = 0$。
\end{proof}

上面的命题说明,实系数多项式的虚根共轭成对出现,且它们的重数相同。下面我们证明实系数多项式唯一分解定理。

\begin{proof}[数学归纳法]
	设 $f(x) \in \R[x] \backslash \set{0}$,$\deg f(x) = n$,对 $n$ 进行归纳。

	当 $n = 1$ 时,$f(x) = a(x - c) \pod{a, c \in \R}$,显然成立。

	下面假设当 $\deg f(x) < n$ 时假设成立。设实系数多项式 $f(x)$ 有一复根 $c$,则 $f(c) = 0$。
	\begin{enumerate}
		\item 如果 $c \in \R$,则 $f(x) = (x - c) f_1(x)$。对 $f_1(x)$ 应用归纳假设,则在 $\deg f(x) = n$ 的前提下原命题对于该情况也成立。

		\item 如果 $c \not \in \R, c \in \C$,则可知:
		$$
		(x - c) \mid f(x), \quad (x - \overline c) \mid f(x)
		$$

		又因为 $(x - c, x - \overline c) = 1$,所以\footnote{在 $\mathbb K[x]$ 中,如果 $f(x) \mid h(x)$,$g(x) \mid h(x)$,且 $(f(x), g(x)) = 1$,那么 $f(x) g(x) \mid h(x)$。} $\bigl( (x - c) (x - \overline c) \bigr) \mid f(x)$,于是可以设:
		$$
		f(x) = (x^2 - (c + \overline c) x + c \cdot \overline c) \cdot f_2(x)
		$$

		对 $f_2(x)$ 应用归纳假设,则在 $\deg f(x) = n$ 的前提下原命题对于该情况也成立。
	\end{enumerate}
\end{proof}

由实系数多项式唯一因式分解定理,我们易得以下结论。

\begin{proposition}
	实系数的奇次多项式至少有一个实根。
\end{proposition}

\subsubsection{实系数多项式根的范围}

\begin{proposition}
	设 $f(x) = a_n x^n + a_{n - 1} x^{n - 1} + \cdots + a_1 x + a_0$ 是一个复系数多项式,其次数 $n \ge 1$。令:
	$$
	M = \max \{ \abs{a_{n - 1}}, \abs{a_{n - 2}}, \ldots, \abs{a_0} \}
	$$

	则当 $\abs{z} \ge 1 + \dfrac{M}{\abs{a_n}}$ 时,有:
	$$
	\abs{f(z)} \ge \abs{a_n z^n} - \abs{a_{n - 1} z^{n - 1} + \cdots + a_1 z + a_0} > 0
	$$
\end{proposition}

\begin{proof}
	首先有:
	$$
	\begin{aligned}
		\abs{f(z)} &= \abs{a_n z^n + a_{n - 1} z^{n - 1} + \cdots + a_1 z + a_0}
		\\&\ge
		\abs{a_n z^n} - \abs{a_{n - 1} z^{n - 1} + \cdots + a_1 z + a_0}
	\end{aligned}
	$$

	令 $M = \max \{ \abs{a_{n - 1}}, \ldots, \abs{a_1}, \abs{a_0} \}$,则:
	$$
	\begin{aligned}
		\abs{a_{n - 1} z^{n - 1} + \cdots + a_1 z + a_0} &\le \abs{a_{n - 1}} \abs{z}^{n - 1} + \cdots + \abs{a_1} \abs{z} + \abs{a_0}
		\\&\le
		M \bigl( \abs{z}^{n - 1} + \cdots + \abs{z} + 1 \bigr)
		\\&=
		M \dfrac{\abs{z}^n - 1}{\abs{z} - 1}
	\end{aligned}
	$$

	当 $\abs{z} > 1$ 时,可得:
	$$
	M \dfrac{\abs{z}^n - 1}{\abs{z} - 1} < M \dfrac{\abs{z}^n}{\abs{z} - 1}
	$$

	于是此时,$\abs{f(z)}$ 满足:
	$$
	\abs{f(z)} > \abs{a_n z^n} - M \dfrac{\abs{z}^n}{\abs{z} - 1}
	$$

	解关于 $\abs{z}$ 的不等式:
	$$
	\abs{a_n z^n} - M \dfrac{\abs{z}^n}{\abs{z} - 1} \ge 0
	$$
	得:
	$$
	\abs{z} \ge 1 + \dfrac{M}{\abs{a_n}}
	$$

	从而,当 $\abs{z} \ge 1 + \dfrac{M}{\abs{a_n}}$ 时,有原命题成立。
\end{proof}

以上定理表明,$f(x)$ 的复根全部在以原点为圆心,以 $1 + \dfrac{M}{\abs{a_n}}$ 为半径的圆内。把这一结论用到实系数多项式上,便可得到以下推论。

\begin{theorem}
	设 $f(x) = a_n x^n + a_{n - 1} x^{n - 1} + \cdots + a_1 x + a_0$ 是一个实系数多项式,其次数 $n \ge 1$。令:
	$$
	M = \max \{ \abs{a_{n - 1}}, \abs{a_{n - 2}}, \ldots, \abs{a_1}, \abs{a_0} \}
	$$

	则 $f(x)$ 的实根全部在区间 $\biggl( -1 - \dfrac{M}{\abs{a_n}}, 1 + \dfrac{M}{\abs{a_n}} \biggr)$ 内。
\end{theorem}

\begin{example*}
	设 $f(x) = x^3 - x + 1$,则 $M = 1$,$1 + \dfrac{M}{\abs{a_n}} = 2$,因此 $f(x)$ 的实根全都在区间 $(-2, 2)$ 内。
\end{example*}

需要注意,以上定理只能得到 $f(x)$ 有实根时实根的范围,并没有说明 $f(x)$ 一定有实根。

\subsubsection{实系数多项式实根的数量:Sturm 定理}

\begin{definition}{变号,变号数}
	设 $c_1, c_2, \ldots, c_m$ 是一个非零实数的有限序列。如果 $c_i c_{i + 1} < 0$,那么我们说在第 $i + 1$ 项有一个\emph{变号}。这个序列中变号的总数称为它的\emph{变号数}。一个有限的实数序列的变号数定义为去掉这个序列中的 $0$ 以后得到的序列的变号数。
\end{definition}

\begin{example*}
	序列 $-2, 0, 1, 0, 0, 3, -4, 5$ 的变号数是 3。
\end{example*}

\begin{theorem}[Sturm 定理]
	设 $f(x)$ 是一个次数大于 $0$ 的实系数多项式。令 $f_0(x) = f(x)$,$f_1(x) = f'(x)$,不断做下述略微修改的辗转相除法:
	$$
	\begin{aligned}
		f_0(x) &= q_1(x) f_1(x) \mathbin{\textcolor{red}{-}} f_2(x) \pod{\deg f_2(x) < \deg f_1(x)}
		\\
		f_1(x) &= q_2(x) f_2(x) \mathbin{\textcolor{red}{-}} f_3(x) \pod{\deg f_3(x) < \deg f_2(x)}
		\\
		\vdots
		\\
		f_{s - 1}(x) &= q_s(x) f_s(x)
	\end{aligned}
	$$

	由此得到一个多项式序列 $f_0, f_1, f_2, \ldots, f_s$,称该序列是 $f(x)$ 的\emph{\idx{标准序列}}。设 $V_c$ 表示序列 $f_0(c), f_1(c), \ldots, f_s(c)$ 的变号数。若区间 $[a, b]$ 使得 $f(a) \ne 0, f(b) \ne 0$,则 $f(x)$ 在区间 $(a, b)$ 内的不同实根的个数等于 $V_a - V_b$。
\end{theorem}

Sturm 定理的证明比较复杂,此处略去。我们下面只看它的应用举例。

\begin{example}
	$f(x) = x^3 - x + 1$,前面已经说明它的实根全部在区间 $(-2, 2)$ 内,$f(-2) f(2) \ne 0$。由 Sturm 定理,其标准序列为:
	$$
	f_0(x) = x^3 - x + 1, \quad f_1(x) = 3x^2 - 1, \quad f_2(x) = \dfrac{2}{3} x - 1, \quad f_3 = -\dfrac{23}{4}
	$$

	分别求解 $V_a, V_b$:
	\begin{table}[H]
		\centering
		\begin{tabular}{c|cccc}\toprule
			& $f_0$ & $f_1$ & $f_2$ & $f_3$
			\\\hline
			$-2$ & $-$ & $+$ & $-$ & $-$
			\\
			$2$ & $+$ & $+$ & $+$ & $-$
			\\\bottomrule
		\end{tabular}
	\end{table}

	得 $V(-2) = 2, V(2) = 1$。所以 $f(x)$ 在 $[-2, 2]$ 里的实根个数为 $1$。
\end{example}

利用 Sturm 定理,我们可以得到一类特殊的三次多项式的根的判别公式。

\begin{proposition}
	设 $f(x) = x^3 + px + q \pod{p, q \in \R}$。令 $\Delta = \dfrac{q^2}{4} + \dfrac{p^3}{27}$,则 $\Delta$ 与 $f(x)$ 的根的关系为:
	\begin{table}[H]
		\centering
		\begin{tabular}{c|c}\toprule
			$\Delta$ & $f(x)$
			\\\hline
			$> 0$ & 有一个实根,有一对复根。
			\\
			$= 0$ & 有三个实根,但有一个根是 2 重根或 3 重根。
			\\
			$< 0$ & 有三个不同实根。
			\\\bottomrule
		\end{tabular}
	\end{table}
\end{proposition}

\begin{proof}
	$f(x)$ 的标准序列为:
	$$
	\begin{aligned}
		&f_0(x) = x^3 + px + q, \quad f_1(x) = 3x^2 + p,
		\\
		&f_2(x) = -2px - q, \quad f_3(x) = -(4p^3 + 27 q^2)
	\end{aligned}
	$$

	\begin{enumerate}
		\item 若 $\Delta < 0$,则 $p < 0$。当 $x \to -\infty$ 时,有:
		$$
		f_0(-\infty) < 0, \quad f_1(-\infty) > 0, \quad f_2(-\infty) < 0, \quad f_3(-\infty) > 0
		$$

		当 $x \to +\infty$ 时,标准序列的取值都大于 $0$,变号数等于 $0$。由 Sturm 定理,此时 $f(x)$ 有三个不同的实根。

		\item 若 $\Delta = 0$,$q = 0$,则 $f(x)$ 有一个 3 重根 $0$。若 $\Delta = 0$,$q \ne 0$,则 $p < 0$,同理,可得 $V(-\infty) = 2, V(+\infty) = 0$。由 Sturm 定理,此时 $f(x)$ 有两个不同的实根,则必然有一个 2 重根。

		\item 若 $\Delta > 0$,则根据 $p$ 的取值进行分类讨论:
		\begin{enumerate}
			\item 若 $p < 0$,则 $V(-\infty) = 2, V(+\infty) = 1$。
			\item 若 $p = 0$,则无论 $q$ 是否大于 $0$,都有 $V(-\infty) = 2, V(+\infty) = 1$。
			\item 若 $p > 0$,则 $V(-\infty) = 2, V(+\infty) = 1$。
		\end{enumerate}

		此时 $f(x)$ 有一个实根,则另两个根必然是共轭复根。
	\end{enumerate}
\end{proof}

% TODO: 补充书上例 10。

\subsection{有理数域上的不可约多项式}

\subsubsection{整系数多项式与本原多项式}

我们知道,有理数总是可以表示成 $p / q \pod{p, q \in \Z}$ 的形式,所以可以很自然地想到把一个有理数域上的多项式化为一个整系数多项式乘一个有理数。为此我们首先形式化地定义整系数多项式组成的集合。

\begin{definition}{整系数多项式环}
	记 $\Z[x]$ 表示所有整系数多项式组成的集合。可以证明 $\Z[x]$ 是一个环,且是一个有单位元的无零因子的交换环。
\end{definition}

注意,$\Z$ 不是一个数域,因此前面给出的有关 $\mathbb K[x]$ 的定义和性质对 $\Z[x]$ 不一定适用。我们强调 $\Z[x]$ 中可分解的概念。

\begin{definition}{可分解,不可分解的整系数多项式}
	设 $f(x) \in \Z[x]$。若存在 $g(x), h(x) \in \mathbb Z[x]$,使得 $f(x) = g(x) h(x)$ 且 $\deg g(x) < \deg f(x), \deg h(x) < \deg f(x)$,则称 $f(x)$ 在 $\Z[x]$ 中\emph{可分解},否则称 $f(x)$ 是\emph{不可分解的整系数多项式}。
\end{definition}

特别注意,$\Z[x]$ 中可分解的概念不是可约的概念。作为扩展,我们仍然形式化地给出可约的概念。

\begin{definition}{不可约多项式,可约的}
	设 $f(x) \in \Z[x]$。如果 $f(x)$ 在 $\Z[x]$ 中的因式只有 $\pm 1$(即 $\Z[x]$ 的可逆元)和 $\pm f(x)$(即 $f(x)$ 的相伴元),那么称 $f(x)$ 是 $\Z$ 上的\emph{不可约多项式},否则称它是在 $\Z$ 上\emph{可约的}。
\end{definition}

事实上,一般而言,在整环\footnote{一个有单位元的无零因子的交换环被称为\emph{\idx{整环}}。} $R$ 中,如果一个元素 $a$($a \ne 0$,且 $a$ 不是可逆元)的因子只有可逆元和 $a$ 的相伴元,那么称 $a$ 是不可约的,否则称 $a$ 是可约的。

\bigskip

\begin{definition}{容度}
	设 $g(x) = b_m x^m + \cdots + b_1 x + b_0 \in \Z[x]$,称:
	$$
	\gcd(b_0, b_1, \ldots, b_m)
	$$
	为 $g(x)$ 的\emph{容度}。
\end{definition}

\begin{definition}{本原多项式}
	非零的容度为 $1$ 的整系数多项式被称为\emph{本原多项式}。
\end{definition}

\begin{proposition}
	与非零的有理系数多项式 $f(x)$ 相伴的本原多项式在相差一个正负号下是唯一的。
\end{proposition}

\begin{proof}
	充分性显然,下面证必要性。设 $f(x) = r g(x) = s h(x)$,其中 $g(x), h(x)$ 都是本原多项式,$r, s \in \Q^*$,则 $g(x) = \dfrac{s}{r} h(x)$。设 $\dfrac{s}{r} = \dfrac{q}{p}$,其中 $p, q \in \Z$,且 $(p, q) = 1$,则有:
	$$
	p g(x) = q h(x)
	$$

	设 $g(x) = \sum\limits_{i = 0}^ n b_i x^i$,$h(x) = \sum\limits_{i = 0}^n c_i x^i$,则:
	$$
	p \sum\limits_{i = 0}^n b_i x^i = q \sum\limits_{i = 0}^n c_i x^i
	$$

	从而 $p b_i = q c_i \pod{i = 0, 1, \ldots, n}$,于是:
	$$
	q \mid p b_i \pod{i = 0, 1, \ldots, n}
	$$

	由于 $(q, p) = 1$,所以 $q \mid b_i \pod{i = 0, 1, \ldots, n}$。由于 $g(x)$ 是本原多项式,所以 $q	 = \pm 1$。同理可证 $p = \pm 1$。于是 $g(x) = \pm h(x)$。
\end{proof}

我们立即可得以下命题。

\begin{proposition}
	两个本原多项式 $g(x)$ 与 $h(x)$ 在 $\Q[x]$ 中相伴当且仅当 $g(x) = \pm h(x)$。
\end{proposition}

要继续研究本原多项式在 $\Q[x]$ 中是否可约的问题,我们需要用到下面的高斯引理。

\begin{proposition}[高斯引理]
	两个本原多项式的乘积还是本原多项式。
\end{proposition}

\begin{proof}
	设 $f(x) = \sum\limits_{i = 0}^n a_i x^i$,$g(x) = \sum\limits_{j = 0}^m b_j x^j$ 是两个本原多项式,则:
	$$
	f(x) g(x) = \sum\limits_{k = 0}^{m + n} \biggl( \sum\limits_{i + j = k} a_i b_j \biggr) x^k \triangleq \sum\limits_{k = 0}^{m + n} c_k x^k
	$$

	假设存在素数 $p$,使得 $p \mid c_s \pod{s = 0, 1, \ldots, m + n}$。由于 $f(x)$ 是本原多项式,所以 $p$ 不能同时整除 $f(x)$ 的每一项的系数,于是存在 $r \pod{0 \le r \le n}$ 满足:
	$$
	p \mid a_0, p \mid a_1, \ldots, p \mid a_{r - 1}, p \nmid a_r
	$$

	同理,存在 $s \pod{0 \le s \le m}$ 满足:
	$$
	p \mid b_0, p \mid b_1, \ldots, p \mid b_{s - 1}, p \nmid b_s
	$$

	于是对于 $f(x) g(x)$ 的 $r + s$ 次项的系数,满足 $p \mid a_r b_s$。由于 $p$ 是素数,所以 $p \mid a_r$ 或 $p \mid b_s$,矛盾。所以 $f(x) g(x)$ 也是本原多项式。
\end{proof}

\subsubsection{有理数域多项式可约的充分必要条件}

\begin{proposition}
	一个次数大于 $0$ 的本原多项式 $g(x)$ 在 $\Q$ 上可约的充分必要条件是,$g(x)$ 能分解成两个次数较低的本原多项式的乘积。
\end{proposition}

\begin{proof}
	充分性显然,下面证必要性。设本原多项式 $g(x)$ 在 $\Q$ 上可约,则存在 $g_i(x) \in \Q[x] \pod{i = 1, 2}$,使得:
	$$
	g(x) = g_1(x) g_2(x) \pod{\deg g_i(x) < \deg g(x), i = 1, 2}
	$$

	设 $g_i(x) = r_i h_i(x)$,其中 $r_i \in \Q^*$,$h_i(x)$ 是本原多项式,$i = 1, 2$。则:
	$$
	g(x) = r_1 r_2 h_1(x) h_2(x)
	$$

	由于 $h_1(x) h_2(x)$ 也是本原多项式,所以 $r_1 r_2 = \pm 1$,从而 $g(x) = \pm h_1(x) h_2(x)$。
\end{proof}

可见,次数大于 $0$ 的整系数多项式 $f(x)$ 在 $\Q[x]$ 上可约的充分必要条件是与它相伴的整系数多项式在 $\Z[x]$ 中可分解。

\subsubsection{整系数多项式的有理根、整系数多项式在 $\Q$ 上不可约的判别方法}

根据前文所述的命题,要继续研究有理数系数多项式是否可约的问题,就需要研究整系数多项式在 $\Z[x]$ 中是否可分解的问题,也就相当于要研究整系数多项式在 $\Q[x]$ 中是否可约的问题。由于 $f(x)$ 在 $\Q$ 中有根当且仅当 $f(x)$ 有一次因式,而 $f(x)$ 有一次因式就意味着 $f(x)$ 可约,所以我们首先给出一个整系数多项式在 $\Q$ 中有根的必要条件。

\begin{proposition}
	设 $f(x) = \sum\limits_{i = 0}^n a_i x^i$ 是一个次数 $n$ 大于 $0$ 的整系数多项式,如果 $\dfrac{q}{p}$ 是 $f(x)$ 的一个有理根,其中 $p, q$ 是互素的整数,那么 $p \mid a_n, q \mid a_0$。
\end{proposition}

\begin{proof}
	设 $f(x) = r f_1(x)$,其中 $r \in \Z^*$,$f_1(s)$ 是本原多项式。设 $\dfrac{q}{p}$ 是 $f(x)$ 的一个根,则 $0 = f \biggl( \dfrac{q}{p} \biggr) = r f_1 \biggl( \dfrac{q}{p} \biggr)$。从而 $\dfrac{q}{p}$ 也是 $f_1(x)$ 的一个根。于是在 $\Q[x]$ 中,有 $\biggl( x - \dfrac{q}{p} \biggr) \mid f_1(x)$。因此 $(px - q) \mid f_1(x)$。

	由于 $(p, q) = 1$,因而 $px - q$ 是本原多项式,所以可设:
	$$
	f_1(x) = (px - q) g(x)
	$$
	其中 $g(x) = \sum\limits_{i = 0}^{n - 1} b_i x^i$ 是本原多项式\footnote{易证一个本原多项式可以唯一分解成若干个不可约的本原多项式的积。}。有:
	$$
	f(x) = r (px - q) g(x)
	$$

	分别比较上式两侧的首项系数与常数项,有:
	$$
	a_n = r p b_{n - 1}, \quad a_0 = - r q b_0
	$$

	因此 $p \mid a_n, q \mid a_0$。
\end{proof}

\begin{example}
	求 $f(x) = x^3 - 6 x^2 + 15 x - 14$ 的有理根。

	因为有 $a_3 = 1, a_0 = -14$,所以有理根一定是整数,且必然属于 $\set{\pm 1, \pm 2, \pm 7, \pm 14}$。验证知只有 $2$ 是 $f(x)$ 的一个有理根。
\end{example}

利用以上方法,可以判断一个二次或三次整系数多项式是否在 $\Q$ 上不可约,\textbf{二次或三次}整系数多项式在 $\Q$ 上不可约当且仅当它没有有理根。但是对于四次及以上的整系数多项式 $f(x)$,如果它没有有理根,只能说明 $f(x)$ 没有一次因式,不能说明 $f(x)$ 在 $\Q$ 上不可约。下面我们探寻整系数多项式在 $\Q$ 中不可约的充分条件。

\begin{theorem}[Eisenstein 判别法]
	设 $f(x) = \sum\limits_{i = 0}^n a_i x^n$ 是一个次数大于 $n$ 的整系数多项式。如果存在一个素数 $p$,使得:
	\begin{enumerate}
		\item $p \mid a_i \pod{i = 0, 1, \ldots, n - 1}$
		\item $p \nmid a_n$
		\item $p^2 \nmid a_0$
	\end{enumerate}

	那么 $f(x)$ 在 $\Q$ 上不可约。
\end{theorem}

