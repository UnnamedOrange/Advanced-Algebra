% Licensed under the Creative Commons Attribution Share Alike 4.0 International.
% See the LICENCE file in the repository root for full licence text.

\section{分块矩阵}

前文中提出了矩阵乘法的更多理解方式,其中一种为 “$C$ 的每一列是 $A$ 的各列向量以 $B$ 的对应列为权的加权和”:
$$
C = \begin{bmatrix} A \vec \beta_1 & A \vec \beta_2 & \cdots & A \vec \beta_n \end{bmatrix}
$$

我们希望用与之类似的方法研究矩阵,使得问题更容易被理解。为此,我们提出以下概念。

\begin{definition}{子矩阵}
	由矩阵 $A$ 的若干行、若干列的交叉位置元素按原来顺序排成的矩阵称为 $A$ 的一个\emph{子矩阵}。
\end{definition}

\begin{definition}{矩阵的分块,分块矩阵}
	把一个矩阵 $A$ 的行分成若干组,列也分成若干组,从而 $A$ 被分成若干个子矩阵。再把 $A$ 看成是由这些子矩阵组成的,这种方法称为\emph{矩阵的分块}。称像 $A$ 这样的由子矩阵组成的矩阵为\emph{分块矩阵}。
\end{definition}