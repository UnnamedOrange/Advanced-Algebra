% Licensed under the Creative Commons Attribution Share Alike 4.0 International.
% See the LICENCE file in the repository root for full licence text.

\section{特殊矩阵}

\subsubsection{对角矩阵}

\begin{definition}{对角矩阵}
	主对角线以外的元素全为 $0$ 的方阵称为\emph{对角矩阵},简记为 $\operatorname{diag} \{d_1, \ldots, d_n\}$。
\end{definition}

\subsubsection{基本矩阵}

\begin{definition}{基本矩阵}
	只有一个元素是 $1$,其余元素全为 $0$ 的矩阵称为\emph{基本矩阵}。$(i, j)$ 元为 $1$ 的基本矩阵记作 $E_{ij}$。
\end{definition}

可以用基本矩阵表示任一矩阵 $A_{m \times n} = (a_{ij})$。
$$
A_{m \times n} = \sum\limits_{i = 1}^{m} \sum\limits_{j = 1}^n a_{ij} E_{ij}
$$

有时可在计算矩阵乘法时按以上等式进行展开,这样做的好处是可以使用以下定理。

\begin{theorem}
	若基本矩阵 $E_{ij}$ 可以与基本矩阵 $E_{kl}$ 相乘,则:
	$$
	E_{ij} E_{kl} =
	\begin{cases}
		E_{il}, & \pod{k = j}
		\\
		0, & \pod{k \ne j}
	\end{cases}
	$$
\end{theorem}

\begin{theorem}
	设矩阵 $A_{m \times n} = (a_{ij})$,$E_{ij}, E_{kl}$ 均为基本矩阵。若 $E_{ij} A E_{kl}$ 是合法的矩阵乘法,则:
	$$
	E_{ij} A E_{kl} = a_{jk} E_{il}
	$$
\end{theorem}

\subsubsection{三角矩阵}

\begin{definition}{上三角矩阵,下三角矩阵}
	主对角线下(上)方的元素全为 $0$ 的方阵称为\emph{上(下)三角(形)矩阵}。
\end{definition}

三角矩阵是一类非常重要的矩阵,下面仅先给出一个有关三角矩阵做乘法的性质。

\begin{theorem}
	两个 $n$ 级上(下)三角矩阵 $A$ 与 $B$ 的乘积仍为上(下)三角矩阵,并且 $AB$ 的主对角元等于 $A$ 与 $B$ 相应主对角元的乘积。
\end{theorem}

\begin{proof}[证明上三角矩阵的情形]
	设 $A = (a_{ij}), B = (b_{ij})$ 都是 $n$ 级上三角矩阵,则:
	$$
	(AB)(i; j) = \sum\limits_{k = 1}^n a_{ik} b_{kj} = \sum\limits_{k = 1}^j a_{ik}b_{kj} + \sum\limits_{k = j + 1}^n a_{ik} b_{kj}
	$$

	当 $i > j$ 时,对于满足 $1 \le k \le j$ 的 $k$,由于 $k \le j < i$,因此 $a_{ik} = 0$;对于满足 $j < k \le n$ 的 $k$,$b_{kj} = 0$。从而 $(AB)(i; j) = 0 \pod{i > j}$,于是 $AB$ 是上三角矩阵。

	计算 $(AB)(i; i)$ 易得 $a_{ii} b_{jj}$。
\end{proof}

