% Licensed under the Creative Commons Attribution Share Alike 4.0 International.
% See the LICENSE file in the repository root for full license text.

\section{特殊矩阵}

\subsubsection{对角矩阵}

\begin{definition}{对角矩阵}
	主对角线以外的元素全为 $0$ 的方阵称为\emph{对角矩阵},简记为 $\operatorname{diag} \{d_1, \ldots, d_n\}$。
\end{definition}

\subsubsection{基本矩阵}

\begin{definition}{基本矩阵}
	只有一个元素是 $1$,其余元素全为 $0$ 的矩阵称为\emph{基本矩阵}。$(i, j)$ 元为 $1$ 的基本矩阵记作 $E_{ij}$。
\end{definition}

可以用基本矩阵表示任一矩阵 $A_{m \times n} = (a_{ij})$。
$$
A_{m \times n} = \sum\limits_{i = 1}^{m} \sum\limits_{j = 1}^n a_{ij} E_{ij}
$$

有时可在计算矩阵乘法时按以上等式进行展开,这样做的好处是可以使用以下定理。

\begin{theorem}
	若基本矩阵 $E_{ij}$ 可以与基本矩阵 $E_{kl}$ 相乘,则:
	$$
	E_{ij} E_{kl} =
	\begin{cases}
		E_{il}, & \pod{k = j}
		\\
		0, & \pod{k \ne j}
	\end{cases}
	$$
\end{theorem}

\begin{theorem}
	设矩阵 $A_{m \times n} = (a_{ij})$,$E_{ij}, E_{kl}$ 均为基本矩阵。若 $E_{ij} A E_{kl}$ 是合法的矩阵乘法,则:
	$$
	E_{ij} A E_{kl} = a_{jk} E_{il}
	$$
\end{theorem}

\subsubsection{三角矩阵}

\begin{definition}{上三角矩阵,下三角矩阵}
	主对角线下(上)方的元素全为 $0$ 的方阵称为\emph{上(下)三角(形)矩阵}。
\end{definition}

三角矩阵是一类非常重要的矩阵,下面仅先给出一个有关三角矩阵做乘法的性质。

\begin{theorem}
	两个 $n$ 级上(下)三角矩阵 $A$ 与 $B$ 的乘积仍为上(下)三角矩阵,并且 $AB$ 的主对角元等于 $A$ 与 $B$ 相应主对角元的乘积。
\end{theorem}

\begin{proof}[证明上三角矩阵的情形]
	设 $A = (a_{ij}), B = (b_{ij})$ 都是 $n$ 级上三角矩阵,则:
	$$
	(AB)(i; j) = \sum\limits_{k = 1}^n a_{ik} b_{kj} = \sum\limits_{k = 1}^j a_{ik}b_{kj} + \sum\limits_{k = j + 1}^n a_{ik} b_{kj}
	$$

	当 $i > j$ 时,对于满足 $1 \le k \le j$ 的 $k$,由于 $k \le j < i$,所以 $a_{ik} = 0$;对于满足 $j < k \le n$ 的 $k$,$b_{kj} = 0$。从而 $(AB)(i; j) = 0 \pod{i > j}$,于是 $AB$ 是上三角矩阵。

	计算 $(AB)(i; i)$ 易得 $a_{ii} b_{jj}$。
\end{proof}

\begin{proof}[使用另一种方法证明下三角矩形的情形]
	设 $A = (a_{ij}), B = (b_{ij})$ 都是 $n$ 级下三角矩阵,则:
	$$
	\begin{aligned}
		AB &= \biggl( \sum_{i = 1}^n \sum_{j = 1}^i a_{ij} E_{ij} \biggr) \biggl( \sum_{k = 1}^n \sum_{l = 1}^k b_{kl} E_{kl} \biggr)
		\\&=
		\sum_{i = 1}^n \sum_{j = 1}^i \sum_{l = 1}^j a_{ij} b_{jl} E_{il}
		\\&=
		\sum_{i = 1}^n \sum_{l = 1}^i \sum_{j = l}^i a_{ij} b_{jl} E_{il}
	\end{aligned}
	$$
	可以看出,求和项中 $E_{il}$ 满足 $i \ge l$,于是 $AB$ 是下三角矩阵。

	$AB$ 的 $(i, i)$ 元等于 $E_{ii}$ 的系数,易得 $(AB)(i; i) = a_{ii} b_{ii}$。
\end{proof}

\subsubsection{对称矩阵}

\begin{definition}{对称矩阵}
	如果矩阵 $A$ 满足 $A' = A$,那么称 $A$ 是\emph{对称矩阵}。
\end{definition}

以下定理显然成立。

\begin{theorem}
	设 $A, B$ 都是数域 $K$ 上的 $n$ 级对称矩阵,则 $A + B, kA \pod{k \in \mathbb K}$ 都是对称矩阵。
\end{theorem}

以下定理将对称矩阵与矩阵乘法联系了起来。

\begin{theorem}
	设 $A$ 与 $B$ 都是 $n$ 级对称矩阵,则 $AB$ 为对称矩阵的充分必要条件是 $A$ 与 $B$ 可交换。
\end{theorem}

\begin{proof}
	因为 $A$ 与 $B$ 都是对称矩阵,所以 $(AB)' = B'A' = BA$,于是:$AB$ 为对称矩阵 $\Longleftrightarrow$ $(AB)' = AB$ $\Longleftrightarrow$ $BA =  AB$。
\end{proof}

\subsubsection{斜对称矩阵}

\begin{definition}{斜对称矩阵}
	一个矩阵 $A$ 如果满足 $A' = -A$,那么称 $A$ 为\emph{斜对称矩阵}。
\end{definition}

\begin{theorem}
	数域 $\mathbb K$ 上奇数级斜对称矩阵的行列式等于 $0$。
\end{theorem}

\begin{proof}
	设 $A$ 是 $n$ 级斜对称矩阵,$n$ 是奇数,则 $A' = -A$,从而 $|A'| = |-A|$,于是 $|A| = |A'| = (-1)^n |A| = -|A|$。因此 $|A| = 0$。
\end{proof}

\begin{theorem}
	数域 $\mathbb K$ 上的 $n$ 级矩阵一定可以分解成一个对称矩阵和一个斜对称矩阵的和。
\end{theorem}

\begin{proof}
	设 $B$ 是一个 $n$ 级矩阵,则:
	$$
	B = \dfrac{B + B'}{2} + \dfrac{B - B'}{2}
	$$

	其中加号左侧是一个对称矩阵,加号右侧是一个斜对称矩阵。证毕。
\end{proof}