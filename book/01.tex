% Licensed under the Creative Commons Attribution Share Alike 4.0 International.
% See the LICENSE file in the repository root for full license text.

\chapter{线性方程组}

\section{线性方程组与矩阵}

我们首先给出线性方程组的相关定义。

\begin{definition}{线性方程组,系数,常数项}
	若一个方程组的每个方程中左侧都是未知量 $x_1, \ldots, x_n$ 的一次齐次式,右侧是常数,则称这样的方程组为\emph{线性方程组}。每个未知量前面的数称为\emph{系数},右端的项称为\emph{常数项}。
\end{definition}

\begin{definition}{$n$ 元线性方程组}
	含 $n$ 个未知量的线性方程组称为 $n$ 元线性方程组。
\end{definition}

\begin{definition}{一个解,解集}
	对于一个 $n$ 元线性方程组,如果用 $c_1, \ldots, c_n$ 分别代入 $x_1, \ldots, x_n$ 后,每个方程都变成恒等式,那么称 $n$ 元有序数组 $(c_1, c_2, \ldots, c_n)$ 是该线性方程组的\emph{一个解}。该方程组的所有解组成的集合称为这个方程组的\emph{解集}。
\end{definition}

\begin{definition}{线性方程组的初等变换}
	称以下三种变换为\emph{线性方程组的初等变换}:
	\begin{enumerate}
		\item 把一个方程的倍数加到另一个方程上。例如,将第一个方程的 $-3$ 倍加到第二个方程上,记为 $\circled{2} +\circled{1} \cdot (-3)$。
		\item 互换两个方程的位置。例如,互换第一个方程和第二个方程的位置,记为 $(\circled{1}, \circled{2})$。
		\item 用一个非零数乘某一个方程。例如,将第一个方程乘以 $\dfrac{1}{3}$,记为 $\circled{1} \cdot \dfrac{1}{3}$。
	\end{enumerate}
\end{definition}

\begin{definition}{同解}
	两个线性方程组的解集相等,称这两个方程组\emph{同解}。线性方程组经过线性方程组的初等变换得到的方程组与原方程组同解。
\end{definition}

为了方便求解线性方程组,我们给出矩阵的相关概念。

\begin{definition}{增广矩阵,系数矩阵}
	对于一个线性方程组,只写出它的系数和常数项,并且把它们按照原来的次序排成一张表,这张表称为线性方程组的\emph{增广矩阵},而只列出系数的表称为方程组的\emph{系数矩阵}。
\end{definition}

\begin{definition}{矩阵,一个元素,$(i \comma j)$ 元}
	由 $s \times m$ 个数排成 $s$ 行、$m$ 列的一张表称为一个 \emph{$s \times m$ 矩阵},其中的每一个数称为这个矩阵的\emph{一个元素},第 $i$ 行第 $j$ 列交叉位置的元素称为矩阵的 \emph{$(i, j)$ 元}。
\end{definition}

一个 $s \times m$ 矩阵可以记作 $A_{s \times m}$。它的 $(i, j)$ 元记作 $A(i; j)$。如果矩阵 $A$ 的 $(i, j)$ 元是 $a_{ij}$,那么可以记作 $A = (a_{ij})$。

下面定义一些特殊矩阵。

\begin{definition}{零矩阵}
	元素全为 $0$ 的矩阵称为\emph{零矩阵},简记为 $0$。$s$ 行 $m$ 列的零矩阵可以记作 $0_{s \times m}$。
\end{definition}

\begin{definition}{方阵,$m$ 级矩阵}
	如果一个矩阵 $A$ 的行数与列数相等,则称它为\emph{方阵}。$m$ 行 $m$ 列的方阵也称为 \emph{$m$ 级矩阵}。
\end{definition}

将矩阵作用于线性方程组上,我们可以给出以下概念。

\begin{definition}{矩阵的初等行变换}
	称以下三种变换为\emph{矩阵的初等行变换}:
	\begin{enumerate}
		\item 把一行的倍数加到另一行上。
		\item 互换两行的位置。
		\item 用一个非零数乘某一行。
	\end{enumerate}
\end{definition}

\begin{definition}{零行,非零行}
	元素全为 $0$ 的行称为\emph{零行};元素不全为 $0$ 的行称为\emph{非零行}。
\end{definition}

\begin{definition}{主元}
	非零行从左边数起第一个不为 $0$ 的元素称为\emph{主元}。
\end{definition}

\begin{definition}{阶梯形矩阵}
	符合以下特点的矩阵称为\emph{阶梯形矩阵}:
	\begin{enumerate}
		\item 如果有零行的话,零行在下方。
		\item 非零行的主元的列指标随着行指标的递增而严格增大。
	\end{enumerate}
\end{definition}

\begin{definition}{简化行阶梯形矩阵}
	符合以下特点的矩阵称为\emph{简化行阶梯形矩阵}:
	\begin{enumerate}
		\item 是阶梯形矩阵。
		\item 每个非零行的主元都是 $1$。
		\item 每个主元所在的列的其余元素都是 $0$。
	\end{enumerate}
\end{definition}

\begin{theorem}
	任意一个矩阵都可以经过一系列初等行变换化成阶梯形矩阵。
\end{theorem}

按行进行数学归纳法即可证明,证明过程略。

\begin{theorem}
	任意一个矩阵都可以经过一系列初等行变换化成简化行阶梯形矩阵。
\end{theorem}

在得到阶梯形矩阵后,不难通过进一步的矩阵初等行变换得到简化行阶梯形矩阵,具体证明过程略。

\section{线性方程组解的情况及其判别准则}

\begin{definition}{一般解,主变量,自由未知量}
	假设某方程组的一个解可以写成 $(c_2 + 2, c_2, -1)$,其中 $c_2$ 表示 $x_2$ 取任意一个数。可见原方程组有无穷多个解,用下述表达式来表示这无穷多个解:$\begin{cases} x_1 = x_2 + 2 \\ x_3 = -1 \end{cases}$,称之为原线性方程组的\emph{一般解}。

	在简化行阶梯形矩阵中,以主元为系数的未知量称为\emph{主变量},如上例中的 $x_1, x_3$。其余未知量称为\emph{自由未知量}。
\end{definition}

由于总是可以把一个矩阵通过初等行变换化为阶梯形矩阵,所以我们只需研究与原方程组同解的阶梯形方程组。

\begin{theorem}
	系数和常数项为有理数(或实数,或复数)的 $n$ 元线性方程组的解的情况有且只有三种可能:无解、有唯一解、有无穷多解。判别方式如下。

	设阶梯形方程组有 $n$ 个未知量,它的增广矩阵 $J$ 有 $r$ 个非零行,它的简化行阶梯形矩阵为 $J_1$。
	\begin{enumerate}
		\item 阶梯形方程组中出现形如 $0 = d \pod{d \ne 0}$ 的方程,则阶梯形方程组无解。
		\item 阶梯形方程组不出现形如 $0 = d \pod{d \ne 0}$ 的方程。由于阶梯形矩阵非零行主元的列指标随行指标的增加严格递增,而 $J$ 的最后一个非零行的主元不能位于第 $n + 1$ 列,因此 $r \le n$。
		\begin{enumerate}
			\item 若 $r = n$,通过观察 $J_1$,可以得出原阶梯形方程组必有唯一解 $(c_1, \ldots, c_{n - 1}, c_n)$,其中 $c_i$ 是 $J_1$ 的第 $i$ 个常数项。
			\item 若 $r < n$,将自由未知量的项移到等号右边,等式左边保留主变量,并且省略 $0 = 0$ 这样的方程。发现,对自由未知量取任意一组值,都能求出主变量的值从而得到方程组的一个解。由于有理数集具有稠密性,所以方程组有无穷多个解。
		\end{enumerate}
	\end{enumerate}
\end{theorem}

\subsection{高斯-约当算法}

高斯约当算法用于求解线性方程组,其思路如下。首先将方程组用增广矩阵的形式表示,然后通过初等行变换将其转换为阶梯形矩阵,并通过上面的判别方式对解的情况进行判定。如果有解,继续通过初等行变换将其转换为简化行阶梯形矩阵,即得方程的一般解或唯一解。

其中的具体判断与操作方法略。

\subsection{齐次线性方程组}

这是一类特殊的方程组,因此我们要对其进行额外讨论。

\begin{definition}{齐次线性方程组}
	常数项全为 $0$ 的线性方程组称为\emph{齐次线性方程组}。
\end{definition}

\begin{definition}{零解,非零解}
	$(0, 0, \cdots, 0)$ 是齐次线性方程组的一个解,称为\emph{零解};其余的解(如果有)称为\emph{非零解}。由线性方程组解的判定方法,如果一个齐次线性方程组有非零解,那么它有无穷多个解。
\end{definition}

\begin{theorem}
	$n$ 元齐次线性方程组有非零解的充分必要条件是:它的系数矩阵经过初等行变换化成的阶梯形矩阵中,非零行的数目 $r < n$。
\end{theorem}

\begin{theorem}
	如果$n$ 元齐次线性方程组方程的数目 $s$ 小于未知量的数目 $n$,那么它一定有非零解。
\end{theorem}

\section{数域}

\begin{definition}{数域}
	复数集的一个子集 $K$ 如果满足:
	\begin{enumerate}
		\item $0, 1 \in K$;
		\item $a, b \in K \Longrightarrow a \pm b, ab \in K$;
		\item $a, b \in K \pod{b \ne 0} \Longrightarrow \dfrac{a}{b} \in K$。
	\end{enumerate}
	其中的加法和乘法是我们常见的加法和乘法,则称 $K$ 是一个\emph{数域}。
\end{definition}

数域定义中的三个条件可以减弱为:
\begin{enumerate}
	\item $1 \in K$;
	\item $a, b \in K \Longrightarrow a - b \in K$;
	\item $a, b \in K \pod{b \ne 0} \Longrightarrow \dfrac{a}{b} \in K$。
\end{enumerate}

原因是可以借以上三个条件推导出原来的三个条件。
\begin{enumerate}
	\item $1 \in K \Longrightarrow 1 - 1 \in K \Longrightarrow 0 \in K$;
	\item $0, b \in K \Longrightarrow 0 - b \in K \Longrightarrow -b \in K$;
	\item $a, (-b) \in K \Longrightarrow a - (-b) \in K \Longrightarrow a + b \in K$;
	\item $1, b \pod{b \ne 0} \in K \Longrightarrow \dfrac{1}{b} \in K \Longrightarrow \dfrac{a}{\frac{1}{b}} \in K \Longrightarrow ab \in K$。
\end{enumerate}

根据定义,最大的数域是 $\C$。$\Q, \R, \C$ 均为数域,但 $\Z$ 不是数域,因为它对除法不封闭。

\begin{theorem}
	有无穷多个数域 $\mathbb F$ 满足 $\Q \subset \mathbb F \subset \R$。
\end{theorem}

\begin{proof}[构造法]
	构造 $\Q(\sqrt p) = \{ a + b \sqrt p : a, b \in \Q \} \pod{p ~\text{是素数}}$。

	显然 $\Q \subset \Q(\sqrt p) \subset \R$。下面证明 $\Q(\sqrt p)$ 是一个数域:

	\begin{enumerate}
		\item 当 $a = 1$,$b = 0$ 时,可知 $1 \in \Q(\sqrt p)$;
		\item 设 $c = a_1 + b_1 \sqrt p$,$d = a_2 + b_2 \sqrt p$($a_1, a_2, b_1, b_2 \in \Q$),显然 $c, d \in \Q(\sqrt p)$。设 $e = c - d = (a_1 - a_2) + (b_1 - b_2) \sqrt p$,显然 $e \in \Q(\sqrt p)$。
		\item 设 $c = a_1 + b_1 \sqrt p$,$d = a_2 + b_2 \sqrt p$($a_1, a_2, b_1, b_2 \in \Q$,$d \ne 0$)。$\frac{c}{d} = \frac{(a_1 + b_1 \sqrt p)(a_2 - b_2 \sqrt p)}{(a_2 + b_2 \sqrt p)(a_2 - b_2 \sqrt p)} = \frac{(a_1a_2 - pb_1b_2) + (b_1 a_2 - a_1 b_2)\sqrt p}{a_2^2 - pb_2^2}$,显然 $\frac{c}{d} \pod{d \ne 0} \in \Q(\sqrt p)$。
	\end{enumerate}

	由数域的定义,可知 $\Q(\sqrt p)$ 是一个数域。证毕。
\end{proof}

\begin{theorem}
	不存在数域 $\mathbb F$ 满足 $\R \subset \mathbb F \subset \C$。
\end{theorem}

\begin{proof}[反证法]
	假设原命题成立,可知:存在一个 $c = a + b \mathrm i \pod{a, b \in \R, b \ne 0}$,使得 $c \in \mathbb F$。由 $\mathbb F$ 是数域,可知 $\frac{c - a}{b} \in \mathbb F$,即 $\mathrm i \in \mathbb F$。由 $\mathrm i \in \mathbb F$,可知 $d + e \mathrm i \in \mathbb F \pod{d, e \in \R}$,即 $\C \subseteq \mathbb F$,与 $\mathbb F \subset \C$ 矛盾。证毕。
\end{proof}