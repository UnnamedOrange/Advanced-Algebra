% Licensed under the Creative Commons Attribution Share Alike 4.0 International.
% See the LICENCE file in the repository root for full licence text.

\section{克拉默法则}

当某线性方程组有唯一解时,我们希望得出用方程组的系数和常数项表示的解。

设某 $n$ 个方程的 $n$ 元线性方程组的增广矩阵为:
$$
\left[\begin{array}{c|c}
	\begin{matrix}
		a_{11} & \cdots & a_{1n}
		\\
		a_{21} & \cdots & a_{2n}
		\\
		\vdots & & \vdots
		\\
		a_{n1} & \cdots & a_{nn}
	\end{matrix}
	&
	\begin{matrix}
		b_1 \\ b_2 \\ \vdots \\ b_n
	\end{matrix}
\end{array}\right]
$$

我们用一个列向量的等式表示原方程组。这里尚未明确定义列向量的概念,但由于该方法非常重要,因此提前于此提出,之后会详细介绍该方法涉及的定义与知识。
$$
x_1 \cdot \begin{pmatrix} a_{11} \\ a_{21} \\ \vdots \\ a_{n1} \end{pmatrix} +
x_2 \cdot \begin{pmatrix} a_{12} \\ a_{22} \\ \vdots \\ a_{n2} \end{pmatrix} +
\cdots +
x_n \cdot \begin{pmatrix} a_{1n} \\ a_{2n} \\ \vdots \\ a_{nn} \end{pmatrix} =
\begin{pmatrix} b_1 \\ b_2 \\ \vdots \\ b_n \end{pmatrix}
$$

设:
$$
\begin{matrix}
	\vec{\beta_i} = \begin{pmatrix} a_{1i} \\ a_{2i} \\ \vdots \\ a_{ni} \end{pmatrix},
	&
	\vec{\beta} = \begin{pmatrix} b_1 \\ b_2 \\ \vdots \\ b_n \end{pmatrix}
\end{matrix}
$$

则原方程组可记为:
$$
x_1 \vec \beta_1 + x_2 \vec \beta_2 + \cdots + x_n \vec \beta_n = \vec \beta
$$

移项可得:
$$
x_1 \vec \beta_1 + x_2 \vec \beta_2 + \cdots + x_n \vec \beta_n - \vec \beta =
\begin{pmatrix} 0 \\ 0 \\ \vdots \\ 0 \end{pmatrix}
\triangleq \vec 0
$$

希望得到 $x_i$ 的表达式,于是将上式记为:
$$
x_1 \vec \beta_1 + \cdots + (x_i \vec \beta_i - \vec \beta) + \cdots + x_n \vec \beta_n = \vec 0
$$

设 $n \times n$ 的矩阵 $P$ 为:
$$
P =
\begin{bmatrix} \vec \beta_1 & \cdots & \vec \beta_{i - 1} & x_i \vec \beta_i - \vec \beta & \vec \beta_{i + 1} & \cdots & \vec \beta_n \end{bmatrix}
$$

则由行列式的多线性,$\det P$ 为:
$$
\det P = x_i \cdot
\begin{vmatrix} \vec \beta_1 & \cdots & \vec \beta_n \end{vmatrix}
-
\begin{vmatrix} \vec \beta_1 & \cdots & \vec \beta_{i - 1} & \vec \beta & \vec \beta_{i + 1} & \cdots & \vec \beta_n \end{vmatrix}
$$

由于 $-(x_i \vec \beta_i - \vec \beta)$ 是其他各列向量的和,因此 $\det P = 0$。当 $\begin{vmatrix} \vec \beta_1 & \cdots & \vec \beta_n \end{vmatrix} \ne 0$ 时,可得:
$$
x_i = \dfrac{\begin{vmatrix} \vec \beta_1 & \cdots & \vec \beta_{i - 1} &\vec \beta & \vec \beta_{i + 1} & \cdots & \vec \beta_n \end{vmatrix}}{\begin{vmatrix} \vec \beta_1 & \cdots & \vec \beta_n \end{vmatrix}}
$$

总结以上结论,我们得到以下定理。

\begin{theorem}[克拉默法则]
	设 $A$ 是某 $n$ 个方程的 $n$ 元线性方程组的系数矩阵,$B_i$ 是把系数矩阵 $A$ 的第 $i$ 列换成常数项得到的矩阵。当 $|A| \ne 0$ 时,原方程组的唯一解是:
	$$
	\left( \dfrac{|B_i|}{|A|}, \dfrac{|B_2|}{|A|}, \ldots, \dfrac{|B_n|}{|A|} \right)'
	$$
\end{theorem}

在知道了克拉默法则的结论后,我们也可以用以下方法证明该结论。

\begin{proof}
	把 $x_j = \dfrac{|B_j|}{|A|} \pod{j = 1, 2, \ldots, n}$ 代入第 $i$ 个方程的左侧,得:
	$$
	\begin{aligned}&
		a_{i1} \dfrac{|B_1|}{|A|} + a_{i2} \dfrac{|B_2|}{|A|} + \cdots + a_{in} \dfrac{|B_n|}{|A|}
		\\=~&
		\sum\limits_{j = 1}^n a_{ij} \dfrac{|B_j|}{|A|}
		\\=~&
		\dfrac{1}{|A|} \sum\limits_{j = 1}^n a_{ij} \biggl( \sum\limits_{k = 1}^n b_k A_{kj} \biggr) \pod{\text{按第 $j$ 列展开}}
		\\=~&
		\dfrac{1}{|A|} \sum\limits_{k = 1}^n b_k \biggl( \sum\limits_{j = 1}^n a_{ij} A_{kj} \biggr)
		\\=~&
		b_i \pod{\text{某行元素与某行代数余子式乘积之和}}
	\end{aligned}
	$$

	即说明 $\left( \dfrac{|B_i|}{|A|}, \dfrac{|B_2|}{|A|}, \ldots, \dfrac{|B_n|}{|A|} \right)'$ 是原线性方程组的一组解。
\end{proof}

\section{行列式按 $k$ 行(列)展开}

\begin{definition}{$k$ 阶子式}
	$n$ 级矩阵 $A$ 中任意取定 $k$ 行、$k$ 列($1 \le k < n$),位于这些行和列的交叉处的 $k^2$ 个元素按原来的排法组成的 $k$ 级矩阵的行列式称为 $A$ 的一个 \emph{$k$ 阶子式}。取定 $A$ 的第 $i_1, i_2, \ldots, i_k$ 行($i_1 < i_2 < \cdots < i_k$),第 $j_1, j_2, \ldots, j_k$ 列($j_1 < j_2 < \cdots < j_k$),所得到的 $k$ 阶子式记作:
	$$
	A \begin{pmatrix} i_1, i_2, \ldots, i_k \\ j_1, j_2, \ldots, j_k \end{pmatrix}
	$$
\end{definition}

\begin{definition}{余子式}
	划去一个 $k$ 阶子式所在的行和列,剩下的元素按原来的排法组成的 $(n - k)$ 级矩阵的行列式称为该子式的\emph{余子式}。

	令:
	$$
	\{ i'_1, i'_2, \ldots, i'_{n - k} \} = \{ 1, 2, \ldots, n \} \backslash \{ i_1, i_2, \ldots, i_k \}
	$$$$
	\{ j'_1, j'_2, \ldots, j'_{n - k} \} = \{ 1, 2, \ldots, n \} \backslash \{ j_1, j_2, \ldots, j_k \}
	$$
	则记子式 $A \begin{pmatrix} i_1, i_2, \ldots, i_k \\ j_1, j_2, \ldots, j_k \end{pmatrix}$ 的余子式为 $A \begin{pmatrix} i'_1, i'_2, \ldots, i'_{n - k} \\ j'_1, j'_2, \ldots, j'_{n - k} \end{pmatrix}$。
\end{definition}

\begin{definition}{代数余子式}
	在余子式的前面乘以 $(-1)^{(i_1 + i_2 + \cdots + i_k) + (j_1 + j_2 + \cdots + j_k)}$,称为子式的\emph{代数余子式}。
\end{definition}

\begin{theorem}[拉普拉斯定理]
	在 $n$ 级矩阵 $A$ 中,取定第 $i_1, i_2, \ldots, i_k$ 行($i_1 < i_2 < \cdots < i_k$),则这 $k$ 行元素形成的所有 $k$ 阶子式与他们自己的代数余子式的乘积之和等于 $|A|$,即:
	$$
	\begin{aligned}
		|A| = \sum_{1 \le j_1 < j_2 < \cdots < j_k \le n} & (-1)^{(i_1 + i_2 + \cdots + i_k) + (j_1 + j_2 + \cdots + j_k)} \times
		\\&
		A \begin{pmatrix} i_1, i_2, \ldots, i_k \\ j_1, j_2, \ldots, j_k \end{pmatrix} \cdot A \begin{pmatrix} i'_1, i'_2, \ldots, i'_{n - k} \\ j'_1, j'_2, \ldots, j'_{n - k} \end{pmatrix}
	\end{aligned}
	$$
\end{theorem}

拉普拉斯定理即为行列式按 $k$ 行展开定理。同理,将“行”换成“列”也成立,称为行列式按 $k$ 列展开定理。

% TODO: 拉普拉斯定理的证明。太长了,先省略。