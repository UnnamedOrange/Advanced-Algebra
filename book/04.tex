% Licensed under the Creative Commons Attribution Share Alike 4.0 International.
% See the LICENCE file in the repository root for full licence text.

\section{克拉默法则}

我们直接研究 $n$ 个方程的 $n$ 元线性方程组的增广矩阵,以期得出用方程组的系数和常数项表示的方程组的解。

设某 $n$ 个方程的 $n$ 元线性方程组为:
$$
\left[\begin{array}{c|c}
	\begin{matrix}
		a_{11} & \cdots & a_{1n}
		\\
		a_{21} & \cdots & a_{2n}
		\\
		\vdots & & \vdots
		\\
		a_{n1} & \cdots & a_{nn}
	\end{matrix}
	&
	\begin{matrix}
		b_1 \\ b_2 \\ \vdots \\ b_n
	\end{matrix}
\end{array}\right]
$$

我们用一个列向量的等式表示原方程组。这是首次使用这种方法,但该方法非常重要。
$$
x_1 \cdot \begin{bmatrix} a_{11} \\ a_{21} \\ \vdots \\ a_{n1} \end{bmatrix} +
x_2 \cdot \begin{bmatrix} a_{12} \\ a_{22} \\ \vdots \\ a_{n2} \end{bmatrix} +
\cdots +
x_n \cdot \begin{bmatrix} a_{1n} \\ a_{2n} \\ \vdots \\ a_{nn} \end{bmatrix} =
\begin{bmatrix} b_1 \\ b_2 \\ \vdots \\ b_n \end{bmatrix}
$$

设:
$$
\begin{matrix}
	\vec{\beta_i} = \begin{bmatrix} a_{1i} \\ a_{2i} \\ \vdots \\ a_{ni} \end{bmatrix},
	&
	\vec{\beta} = \begin{bmatrix} b_1 \\ b_2 \\ \vdots \\ b_n \end{bmatrix}
\end{matrix}
$$

则原方程组可记为:
$$
x_1 \vec \beta_1 + x_2 \vec \beta_2 + \cdots + x_n \vec \beta_n = \vec \beta
$$

移项可得:
$$
x_1 \vec \beta_1 + x_2 \vec \beta_2 + \cdots + x_n \vec \beta_n - \vec \beta =
\begin{bmatrix} 0 \\ 0 \\ \vdots \\ 0 \end{bmatrix}
\triangleq \vec 0
$$

希望得到 $x_i$ 的表达式,于是将上式记为:
$$
x_1 \vec \beta_1 + \cdots + (x_i \vec \beta_i - \vec \beta) + \cdots + x_n \vec \beta_n = \vec 0
$$

设 $n \times n$ 的矩阵 $P$ 为:
$$
P =
\begin{bmatrix} \vec \beta_1 & \cdots & \vec \beta_{i - 1} & x_i \vec \beta_i - \vec \beta & \vec \beta_{i + 1} & \cdots & \vec \beta_n \end{bmatrix}
$$

则由行列式的多线性,$\det P$ 为:
$$
\det P = x_i \cdot
\begin{vmatrix} \vec \beta_1 & \cdots & \vec \beta_n \end{vmatrix}
-
\begin{vmatrix} \vec \beta_1 & \cdots & \vec \beta_{i - 1} & \vec \beta & \vec \beta_{i + 1} & \cdots & \vec \beta_n \end{vmatrix}
$$

由于 $-(x_i \vec \beta_i - \vec \beta)$ 是其他各列向量的和,因此 $\det P = 0$。当 $\begin{vmatrix} \vec \beta_1 & \cdots & \vec \beta_n \end{vmatrix} \ne 0$ 时,可得:
$$
x_i = \dfrac{\begin{vmatrix} \vec \beta_1 & \cdots & \vec \beta_{i - 1} &\vec \beta & \vec \beta_{i + 1} & \cdots & \vec \beta_n \end{vmatrix}}{\begin{vmatrix} \vec \beta_1 & \cdots & \vec \beta_n \end{vmatrix}}
$$

总结以上结论,我们得到以下定理。

\begin{theorem}[克拉默法则]
	设 $A$ 是某 $n$ 个方程的 $n$ 元线性方程组的系数矩阵,$B_i$ 是把系数矩阵 $A$ 的第 $i$ 列换成常数项得到的矩阵。当 $|A| \ne 0$ 时,原方程组的唯一解是:
	$$
	\left( \dfrac{|B_i|}{|A|}, \dfrac{|B_2|}{|A|}, \ldots, \dfrac{|B_n|}{|A|} \right)'
	$$
\end{theorem}

在知道了克拉默法则的结论后,我们也可以用以下方法证明该结论。

\begin{proof}
	把 $x_j = \dfrac{|B_j|}{|A|} \pod{j = 1, 2, \ldots, n}$ 代入第 $i$ 个方程的左侧,得:
	$$
	\begin{aligned}&
		a_{i1} \dfrac{|B_1|}{|A|} + a_{i2} \dfrac{|B_2|}{|A|} + \cdots + a_{in} \dfrac{|B_n|}{|A|}
		\\=~&
		\sum\limits_{j = 1}^n a_{ij} \dfrac{|B_j|}{|A|}
		\\=~&
		\dfrac{1}{|A|} \sum\limits_{j = 1}^n a_{ij} \biggl( \sum\limits_{k = 1}^n b_k A_{kj} \biggr) \pod{\text{按第 $j$ 列展开}}
		\\=~&
		\dfrac{1}{|A|} \sum\limits_{k = 1}^n b_k \biggl( \sum\limits_{j = 1}^n a_{ij} A_{kj} \biggr)
		\\=~&
		b_i \pod{\text{某行元素与某行代数余子式乘积之和}}
	\end{aligned}
	$$

	即说明 $\left( \dfrac{|B_i|}{|A|}, \dfrac{|B_2|}{|A|}, \ldots, \dfrac{|B_n|}{|A|} \right)'$ 是原线性方程组的一组解。
\end{proof}