% Licensed under the Creative Commons Attribution Share Alike 4.0 International.
% See the LICENCE file in the repository root for full licence text.

\section{矩阵的秩与矩阵的运算}

作为本章的结尾,我们讨论矩阵的某些运算对矩阵的秩的影响。

\subsection{矩阵乘积的秩的上界}

设 $A \in M_{m \times n}(\mathbb K)$,$B \in M_{n \times s}(\mathbb K)$,则由分块矩阵,可以将 $AB$ 记为:
$$
AB = A
\begin{bmatrix} \vec \beta_1 & \vec \beta_2 & \cdots & \vec \beta_s \end{bmatrix}
=
\begin{bmatrix} A \vec \beta_1 & A \vec \beta_2 & \cdots & A \vec \beta_s \end{bmatrix}
$$$$
AB =
\begin{bmatrix} \vec \alpha_1 \\ \vec \alpha_2 \\ \vdots \\ \vec \alpha_m \end{bmatrix}
B =
\begin{bmatrix} \vec \alpha_1 B \\ \vec \alpha_2 B \\ \vdots \\ \vec \alpha_m \end{bmatrix}
$$

以上两式的意义是,$A \vec \beta_i$ 是 $A$ 的列向量的线性组合,$\vec \alpha_i B$ 是 $B$ 的行向量的线性组合。因此,我们有以下定理成立\footnote{设向量组 $\vec \beta_1, \vec \beta_2, \ldots, \vec \beta_r$ 可以由向量组 $\vec \alpha_1, \vec \alpha_2, \ldots, \vec \alpha_s$ 线性表出,如果 $\vec \beta_1, \vec \beta_2, \ldots, \vec \beta_r$ 线性无关,那么 $r \le s$。}。

\begin{theorem}[矩阵乘积的秩的上界]
	$$
	\operatorname{rank}(AB) \le \operatorname{rank}(A)
	$$$$
	\operatorname{rank}(AB) \le \operatorname{rank}(B)
	$$
\end{theorem}

\subsection{Sylvester 秩不等式}

Sylvester 秩不等式描述的是矩阵乘积的秩的下界。相比矩阵乘积的秩的上界,该定理不是那么显然。

\begin{theorem}[Sylvester 秩不等式]
	设 $A$、$B$ 分别是 $s \times n$,$n \times m$ 矩阵,则 $\operatorname{rank}(AB) \ge \operatorname{rank}(A) + \operatorname{rank}(B) - n$。
\end{theorem}

如果只是要证明 Sylvester 秩不等式,难度并不大,见以下代数证明方法。

\begin{proof}[代数证明方法]
	构造分块矩阵 $\begin{bmatrix} I_n & 0 \\ 0 & AB \end{bmatrix}$,可知 $\operatorname{rank}\left(\begin{bmatrix} I_n & 0 \\ 0 & AB \end{bmatrix}\right) = n + \operatorname{rank}(AB)$。现对该分块矩阵进行分块行(列)初等变换。
	$$
	\begin{bmatrix} I_n & 0 \\ 0 & AB \end{bmatrix} \to
	\begin{bmatrix} I_n & 0 \\ -A & AB \end{bmatrix} \to
	\begin{bmatrix} I_n & B \\ -A & 0 \end{bmatrix}
	$$

	注意到:
	$$
	\operatorname{rank}\left(\begin{bmatrix} I_n & 0 \\ 0 & AB \end{bmatrix}\right) = \operatorname{rank}\left(\begin{bmatrix} I_n & B \\ -A & 0 \end{bmatrix}\right) \ge \operatorname{rank}(A) + \operatorname{rank}(B)
	$$
	这是因为该矩阵必然存在一个不为零的 $\operatorname{rank}(A) + \operatorname{rank}(B)$ 阶子式。证毕。
\end{proof}

\bigskip

重点需要掌握下面介绍的两个引理,以及由此产生的几何证明方法\footnote{尽管这是考试的重点,但在应用中这一方法过于抽象,反而代数证明方法更重要了。下面内容的严谨性会因此有所降低。}。

% TODO: 补充例题。

\begin{theorem}
	设 $W \subseteq \mathbb K^m$ 是一个线性子空间,$\varphi \colon \mathbb K^m \to \mathbb K^n$ 是一个线性映射,则满足维数等式:
	$$
	\dim (W \cap \operatorname{Ker} \varphi) + \dim (\varphi(W)) = \dim W
	$$

	如果设:
	$$
	\begin{aligned} \tilde \varphi \colon & W \to \varphi(W) \\ & \vec \gamma \mapsto \varphi(\vec \gamma) \end{aligned}
	$$

	则该定理描述的是 $\tilde \varphi$ 的零空间的维数加上其象空间的维数等于其定义域对应子空间的维数,即:
	$$
	\dim \operatorname{Ker} \tilde \varphi + \dim \operatorname{Im} \tilde \varphi = \dim W
	$$
\end{theorem}

特别地,当 $W = \mathbb K^m$ 时,以上定理即为:
$$
\dim \operatorname{Im} \phi + \dim \operatorname{Ker} \phi = \dim \mathbb K^m = m
$$

% TODO: 补充证明。

\begin{theorem}
	设 $A, B$ 是两个线性映射(或表示其对应的矩阵),且 $BA$ 存在,则:
	$$
	\dim \operatorname{Ker}(BA) \le \dim \operatorname{Ker} A + \dim \operatorname{Ker} B
	$$
\end{theorem}

\begin{proof}
	设线性映射 $A \colon V \to U$,$B \colon U \to W$(其中 $V, U$ 是线性空间),则线性映射 $BA \colon V \to W$。可知:
	$$
	\dim V = \dim \operatorname{Ker}(BA) + \dim \operatorname{Im}(BA)
	$$

	由于:
	$$
	\operatorname{Im} (BA) = B \bigl( \operatorname{Im} A \bigr)
	$$

	因此\footnote{$\dim (W \cap \operatorname{Ker} \varphi) + \dim (\varphi(W)) = \dim W$}:
	$$
	\dim (\operatorname{Im} A \cap \operatorname{Ker} B) + \dim \operatorname{Im}(BA) = \dim \operatorname{Im} A
	$$

	综上,可以计算:
	$$
	\begin{aligned}
		\dim \operatorname{Ker}(BA) &= \dim V - \dim \operatorname{Im}(BA)
		\\&=
		\dim V - \bigl( \dim \operatorname{Im} A - \dim (\operatorname{Im} A \cap \operatorname{Ker} B) \bigr)
		\\&\le
		\dim V - \dim \operatorname{Im} A + \dim \operatorname{Ker} B
		\\&=
		\dim \operatorname{Ker} A + \dim \operatorname{Ker} B
	\end{aligned}
	$$
\end{proof}

下面给出 Sylvester 秩不等式的几何证明方法。

\begin{proof}[几何证明方法]
	设 $A_{m \times n}, B_{n \times s}$ 对应的线性映射分别是 $A \colon U \to W$,$B \colon V \to U$。可知:
	$$
	\begin{aligned}
		\operatorname{rank}(AB) &= \dim \operatorname{Im}(AB)
		\\&=
		\dim V - \dim \operatorname{Ker}(AB)
		\\&\ge
		\dim V - \bigl( \dim \operatorname{Ker} A + \dim \operatorname{Ker} B \bigr)
		\\&=
		\dim \operatorname{Im} B - \dim \operatorname{Ker} A
		\\&=
		\dim \operatorname{Im} B - (\dim U - \dim \operatorname{Im} A)
		\\&=
		\operatorname{rank}(A) + \operatorname{rank}(B) - n
	\end{aligned}
	$$
\end{proof}