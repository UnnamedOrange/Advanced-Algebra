% Licensed under the Creative Commons Attribution Share Alike 4.0 International.
% See the LICENSE file in the repository root for full license text.

\section{重因式}

设多项式 $f(x)$ 为:
$$
f(x) = a p_1^{l_1}(x) p_2^{l_2}(x) \cdots p_s^{l_s}(x) \pod{l_i \ge 0, i = 1, 2, \ldots, s}
$$
其中 $p_1, \ldots, p_s$ 为不可约多项式。

我们自然想要研究 $l_i$。当 $l_i = 0$ 时,$p_i$ 不是 $f(x)$ 的因式。当 $l_i = 1$ 时,我们可以称 $p_i$ 为 $f(x)$ 的\emph{\idx{单因式}}。而当 $l_i > 1$ 时,我们则可以称 $p_i$ 为 $f(x)$ 的重因式。下面给出重因式的形式化定义。

\begin{definition}{重因式}
	设 $f(x) \in \mathbb K[x]$。如果不可约多项式 $p(x)$ 满足 $p^k(x) \mid f(x)$,但 $p^{k + 1}(x) \nmid f(x)$,则称不可约多项式 $p(x)$ 为 $f(x)$ 的 \emph{$k$ 重因式}。
\end{definition}

目前我们没有一个较好的判断多项式的重因式的方法。我们可以借用导数来解决这一问题。下面我们在形式上定义多项式 $f(x)$ 的导数。

\begin{definition}{导数,一阶导数}
	对于 $\mathbb K[x]$ 中的多项式:
	$$
	f(x) = a_n x^n + a_{n - 1} x^{n - 1} + \cdots + a_1 x + a_0
	$$

	我们把以下多项式:
	$$
	n a_n x^{n - 1} + (n - 1) a_{n - 1} x^{n - 2} + \cdots + a_1
	$$
	称为 $f(x)$ 的\emph{导数}(或\emph{一阶导数}),记作 $f'(x)$。
\end{definition}

$f(x)$ 的 $k$ 阶导数一般记作 $f^{(k)}(x)$。显然,若 $f(x)$ 是 $n$ 次多项式,则 $f'(x)$ 是 $n - 1$ 次多项式。特别地,非零数和零多项式的导数是零多项式。

可以验证,按以上形式定义多项式的导数,与数学分析中定义导数的结果一致,满足一系列导数运算的法则。于是我们自然地想到用连续求导的方式来求解多项式的重因式,下面的定理及推论说明了具体的方法。

\begin{proposition}
	在 $\mathbb K[x]$ 中,如果不可约多项式 $p(x)$ 是 $f(x)$ 的一个 $k \pod{k \ge 1}$ 重因式,那么 $p(x)$ 是 $f'(x)$ 的一个 $k - 1$ 重因式。特别地,$f(x)$ 的单因式不是 $f'(x)$ 的因式。
\end{proposition}

\begin{proof}
	由 $p(x)$ 是 $f(x)$ 的 $k$ 重因式,可知存在 $g(x) \in \mathbb K[x]$,使得:
	$$
	f(x) = p^k(x) g(x) \pod{p(x) \nmid g(x)}
	$$

	于是:
	$$
	\begin{aligned}
		f(x) &= k p^{k - 1} p'(x) g(x) + p^k(x) g'(x)
		\\&=
		p^{k - 1}(x) \bigl( k p'(x) g(x) + p(x) g'(x) \bigr)
	\end{aligned}
	$$

	下面只需说明 $p(x) \nmid \bigl( k p'(x) g(x) + p(x) g'(x) \bigr)$。因为 $k \ne 0$,$p(x) \nmid p'(x)$,$p(x) \nmid g(x)$,又因为 $p(x)$ 不可约,所以\footnote{在 $\mathbb K[x]$ 中,从 $p(x) \mid f(x) g(x)$ 可推出 $p(x) \mid f(x)$ 或 $p(x) \mid g(x)$。}:
	$$
	p(x) \nmid k p'(x) g(x)
	$$

	从而:
	$$
	p(x) \nmid \bigl( k p'(x) g(x) + p(x) g'(x) \bigr)
	$$
\end{proof}

\begin{proposition}
	在 $\mathbb K[x]$ 中,不可约多项式 $p(x)$ 是 $f(x)$ 的一个重因式,当且仅当 $p(x)$ 是 $f(x)$ 与 $f'(x)$ 的一个公因式。
\end{proposition}

\begin{proposition}
	在 $\mathbb K[x]$ 中,次数大于 $0$ 的多项式 $f(x)$ 有重因式的充分必要条件是 $f(x)$ 与 $f'(x)$ 有次数大于 $0$ 的公因式。次数大于 $0$ 的多项式没有重因式的充分必要条件是 $f(x)$ 与 $f'(x)$ 互素。
\end{proposition}

如果 $\mathbb K[x]$ 中次数大于 $0$ 的多项式 $f(x)$ 有重因式,我们如何求出一个多项式 $g(x)$,使得 $g(x)$ 与 $f(x)$ 含有完全相同的不可约因式(不计重数),但 $g(x)$ 不含重因式?

\begin{proposition}
	在 $\mathbb K[x]$ 中,若设 $f(x)$ 的唯一因式分解为:
	$$
	f(x) = a p_1^{l_1}(x) p_2^{l_2}(x) \cdots p_s^{l_s}(x) \pod{s \ge 1}
	$$

	则 $\dfrac{f(x)}{(f(x), f'(x))}$ 为:
	$$
	\dfrac{f(x)}{(f(x), f'(x))} = a p_1(x) \cdots p_s(x)
	$$
\end{proposition}

\begin{proof}
	对 $f(x)$ 求导得:
	$$
	f'(x) = p_1^{l_1 - 1}(x) p_2^{l_2 - 1}(x) \cdots p_s^{l_s - 1}(x) h(x)
	$$

	其中 $h(x)$ 不能被任何 $p_i(x) \pod{i = 1, 2, \ldots, s}$ 整除,于是:
	$$
	(f(x), f'(x)) = p_1^{l_1 - 1}(x) p_2^{l_2 - 1}(x) \cdots p_s^{l_s - 1}(x)
	$$

	相除即证得原命题。
\end{proof}

\subsection{例题}

\begin{exercise}
	设 $\Q[x]$ 中的一个多项式为:
	$$
	f(x) = 1 + x + \dfrac{x^2}{2!} + \cdots + \dfrac{x^n}{n!}
	$$

	求证 $f(x)$ 没有重因式。
\end{exercise}

\begin{proof}
	求导得:
	$$
	f'(x) = 1 + x + \dfrac{x^2}{2!} + \cdots + \dfrac{x^{n - 1}}{(n - 1)!}
	$$

	所以有:
	$$
	(f(x), f'(x)) = \biggl( f'(x), \dfrac{x^n}{n!} \biggr)
	$$

	由于 $\dfrac{x^n}{n!}$ 的不可约因式只有 $x$(不计重数),而 $x \nmid f'(x)$,所以 $(f(x), f'(x)) = 1$,因此 $f(x)$ 没有重因式\footnote{次数大于 $0$ 的多项式没有重因式的充分必要条件是 $f(x)$ 与 $f'(x)$ 互素。}。
\end{proof}