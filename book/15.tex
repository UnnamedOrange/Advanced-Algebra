% Licensed under the Creative Commons Attribution Share Alike 4.0 International.
% See the LICENSE file in the repository root for full license text.

\section{矩阵乘积的行列式}

矩阵乘积的行列式有什么性质?可以将下面介绍的比内-柯西公式作为求矩阵乘积的行列式的基本定理。

\begin{theorem}[比内-柯西公式]
	设 $A = (a_{ij})_{s \times n}$,$B = (b_{ij})_{n \times s}$:

	\begin{enumerate}
		\item 如果 $s > n$,那么 $|AB| = 0$。
		\item 如果 $s \le n$,那么 $|AB|$ 等于 $A$ 的所有 $s$ 阶子式与 $B$ 的相应 $s$ 阶子式的乘积之和,即:
		$$
		|AB| = \sum\limits_{1 \le v_1 < v_2 < \cdots < v_s \le n} A \begin{pmatrix} 1, 2, \ldots, s \\ v_1, v_2, \ldots, v_s \end{pmatrix} B \begin{pmatrix} v_1, v_2, \ldots, v_s \\ 1, 2, \ldots, s \end{pmatrix}
		$$
	\end{enumerate}
\end{theorem}

\begin{proof}
	设:
	$$
	A =
	\begin{bmatrix}
		\vec \alpha_1 & \cdots & \vec \alpha_n
	\end{bmatrix}_{s \times n}
	$$$$
	B =
	\begin{bmatrix}
		\vec \beta_1 & \cdots & \vec \beta_s
	\end{bmatrix}_{n \times s}
	$$

	则可记 $AB$ 为:
	$$
	\begin{aligned}
		AB &=
		\begin{bmatrix}
			A \vec \beta_1 & \cdots & A \vec \beta_s
		\end{bmatrix}
		\\&=
		\begin{bmatrix}
			\sum\limits_{k_1 = 1}^n b_{k_1 1} \vec \alpha_{k_1} & \cdots & \sum\limits_{k_s = 1}^n b_{k_s s} \vec \alpha_{k_s}
		\end{bmatrix}
	\end{aligned}
	$$

	由行列式的多线性,得:
	$$
	|AB| = \sum\limits_{k_1 = 1}^n \cdots \sum\limits_{k_s = 1}^n b_{k_1 1} \cdots b_{k_s s} \cdot \begin{vmatrix} \vec \alpha_{k_1} & \cdots & \vec \alpha_{k_s} \end{vmatrix}
	$$

	当 $s > n$ 时,$\begin{bmatrix} \vec \alpha_{k_1} & \cdots & \vec \alpha_{k_s} \end{bmatrix}$ 中一定有重复的列,此时 $|AB| = 0$。当 $s \le n$ 时,只用考虑 $k_1, k_2, \ldots, k_s$ 互不相同的项。当它们互不相同时,设它们从小到大排序的结果为 $k_1^\circ, k_2^\circ, \ldots, k_s^\circ$,则有:
	$$
	\begin{aligned}&
		|AB|
		\\=~&
		\sum\limits_{1 \le k_1^\circ < k_2^\circ < \cdots < k_s^\circ \le n} \sum\limits_{k_1 k_2 \ldots k_n} \begin{vmatrix} \vec \alpha_{k_1^\circ} & \cdots & \vec \alpha_{k_s^\circ} \end{vmatrix} \cdot (-1)^{\tau(k_1 k_2 \ldots k_s)} b_{k_1 1} b_{k_2 2} \cdots b_{k_s s}
	\end{aligned}
	$$

	上式实际上就是:
	$$
	|AB| = \sum\limits_{1 \le v_1 < v_2 < \cdots < v_s \le n} A \begin{pmatrix} 1, 2, \ldots, s \\ v_1, v_2, \ldots, v_s \end{pmatrix} B \begin{pmatrix} v_1, v_2, \ldots, v_s \\ 1, 2, \ldots, s \end{pmatrix}
	$$
\end{proof}

% TODO: 补充基于拉普拉斯展开的证明。

基于比内-柯西公式,我们可以得到矩阵乘积的子式的公式。

\begin{theorem}
	设 $A = (a_{ij})_{s \times n}$,$B = (b_{ij})_{n \times s}$,正整数 $r \le s$。如果 $r > n$,那么 $AB$ 的所有 $r$ 阶子式都等于 $0$。如果 $r \le n$,那么 $AB$ 的任一子式为:
	$$
	(AB) \begin{pmatrix} i_1, \ldots, i_r \\ j_1, \ldots, j_r \end{pmatrix} = \sum\limits_{1 \le v_1 < \cdots < v_r \le n} A \begin{pmatrix} i_1, \ldots, i_r \\ v_1, \ldots, v_r \end{pmatrix} B \begin{pmatrix} v_1, \ldots, v_r \\ j_1, \ldots, j_r \end{pmatrix}
	$$
\end{theorem}

\begin{proof}
	$AB$ 的任一 $r$ 阶子式为:
	$$
	\begin{aligned}
		(AB) \begin{pmatrix} i_1, \ldots, i_r \\ j_1, \ldots, j_r \end{pmatrix} &=
		\begin{vmatrix}
			(AB)(i_1; j_1) & \cdots & (AB)(i_1; j_r)
			\\
			\vdots & & \vdots
			\\
			(AB)(i_r; j_1) & \cdots & (AB)(i_r; j_r)
		\end{vmatrix}
		\\&=
		\left|
		\begin{bmatrix}
			a_{i_1 1} & \cdots & a_{i_1 n}
			\\
			\vdots & & \vdots
			\\
			a_{i_r 1} & \cdots & a_{i_r n}
		\end{bmatrix}
		\begin{bmatrix}
			b_{1 j_1} & \cdots & b_{1 j_r}
			\\
			\vdots & & \vdots
			\\
			b_{n j_1} & \cdots & b_{n j_r}
		\end{bmatrix}
		\right|
	\end{aligned}
	$$

	由比内-柯西公式,如果 $r > n$,那么上式右端的两个矩阵的乘积的行列式等于 $0$,从而 $AB$ 的 $r$ 阶子式都等于 $0$;如果 $r \le n$,那么上式右端的两个矩阵乘积的行列式即等于:
	$$
	(AB) \begin{pmatrix} i_1, \ldots, i_r \\ j_1, \ldots, j_r \end{pmatrix} = \sum\limits_{1 \le v_1 < \cdots < v_r \le n} A \begin{pmatrix} i_1, \ldots, i_r \\ v_1, \ldots, v_r \end{pmatrix} B \begin{pmatrix} v_1, \ldots, v_r \\ j_1, \ldots, j_r \end{pmatrix}
	$$
\end{proof}

有关子式,我们在这里额外提出一个概念。

\begin{definition}{主子式}
	如果矩阵 $A$ 的一个子式的行指标与列指标相同,那么称它为 $A$ 的一个\emph{主子式}。$A$ 的一个 $r$ 阶主子式形如 $A \begin{pmatrix} i_1, \ldots, i_r \\ j_1, \ldots, j_r \end{pmatrix}$。
\end{definition}

最后,我们给出以下重要的定理。它是比内-柯西公式的一个特例。

\begin{theorem}
	设 $A, B$ 均为 $s$ 级方阵,则:
	$$
	|AB| = |A| |B| = |B| |A| = |BA|
	$$
\end{theorem}

可见,$n$ 级矩阵的行列式是从矩阵乘法的非交换性中提取的可交换的量。以后还会看到,矩阵乘法还有其他可交换的量。