% Licensed under the Creative Commons Attribution Share Alike 4.0 International.
% See the LICENCE file in the repository root for full licence text.

\section{矩阵的秩}

矩阵的行秩与列秩有何关系?下面我们研究该问题,并给出矩阵的秩的概念。

首先给出以下引理。

\begin{theorem}
	矩阵的初等列变换不改变矩阵的列秩。
\end{theorem}

具体证明略。证明思路是,对于初等列变换的三种操作,分别验证操作前后的两向量组可以互相表出,从而他们的线性相关性不变。

\begin{theorem}
	矩阵的初等行变换不改变矩阵列向量组的线性相关性,从而不改变矩阵的列秩。
\end{theorem}

\begin{proof}
	设原矩阵为 $C = \begin{bmatrix} \vec \alpha_1 & \cdots & \vec \alpha_s \end{bmatrix}$,经过初等行变换后,新矩阵为 $D = \begin{bmatrix} \vec \beta_1 & \cdots & \vec \beta_s \end{bmatrix}$。以 $C$ 为系数矩阵的齐次线性方程组和以 $D$ 为系数矩阵的齐次线性方程组同解。因此其中一个方程组有非零解当且仅当另一个方程组有非零解,于是 $C$ 线性相关当且仅当 $D$ 线性相关。

	设矩阵 $A$ 经过初等行变换变成矩阵 $B$,并且不妨设 $B$ 的前 $r$ 列构成 $B$ 的列向量组的一个极大线性无关组。由上一段可知,$A$ 的前 $r$ 个列向量构成的向量组线性无关。任取 $A$ 其余列中的一列,记为第 $l$ 列,由上一段以及对 $B$ 的假设可知,$A$ 的前 $r$ 列和第 $l$ 列组成的向量组线性相关,因此 $A$ 的前 $r$ 列构成 $A$ 的列向量组的一个极大线性无关组。

	综上所述,矩阵的初等行变换不改变矩阵的列秩。
\end{proof}

以上两个定理互换行与列显然也成立。

要进一步研究矩阵行向量组与列向量组的秩之间的关系,可以从一类特殊的矩阵入手。我们给出以下定理。

\begin{theorem}
	简化行阶梯形矩阵 $J$ 的行秩与列秩相等,它们都等于 $J$ 的非零行的个数。并且 $J$ 的主元所在列构成列向量组的一个极大线性无关组。
\end{theorem}

\begin{proof}
	设简化行阶梯形矩阵 $J$ 的主元所在列为 $j_1, j_2, \ldots, j_r$ 列。

	\begin{enumerate}
		\item 证明这 $r$ 个列向量构成的向量组线性无关。

		取出这 $r$ 列构成一个新的矩阵,并删去这个新矩阵的零行,则剩下 $r$ 行。该 $r \times r$ 矩阵的行列式非零,则这 $r$ 个缩短的列向量是线性无关的。由于线性无关的向量组的延伸组也线性无关,因此这 $r$ 个列向量构成的向量组线性无关。

		\item 证明这 $r$ 个列向量构成的向量组是 $J$ 的列向量组的一个极大线性无关组。

		注意到,对于第 $j_k \pod{k = 1, \ldots, r}$ 列,它的第 $k$ 行为 $1$,其余行为 $0$。而对于所有的列,$r$ 行之后都是零行。于是,从剩余的列中任选一列,该列总是可以由主元所在的 $r$ 列构成的向量组线性表出。故这 $r$ 列是 $J$ 的列向量组的一个极大线性无关组。

		\item 证明行秩也等于 $r$。

		显然 $J$ 的前 $r$ 行构成的行向量组是 $J$ 的行向量组的极大线性无关组。
	\end{enumerate}

	综上,原命题成立。
\end{proof}

综合以上结论,我们可以得到以下深刻而重要的结论。

\begin{theorem}
	任一矩阵 $A$ 的行秩等于它的列秩。
\end{theorem}

由此,我们定义矩阵的秩。

\begin{definition}{矩阵的秩}
	矩阵 $A$ 的行秩与列秩统称为 $A$ 的\emph{秩},记作 $\operatorname{rank}(A)$。
\end{definition}

根据以上证明过程及结论,我们得出以下重要推论。

\begin{theorem}
	设矩阵 $A$ 经过初等行变换化成简化行阶梯形矩阵 $J$。设 $J$ 的主元所在的列是第 $j_1, j_2, \ldots, j_r$ 列,则 $A$ 的第 $j_1, j_2, \ldots, j_r$ 列构成 $A$ 的列向量组的一个极大线性无关组。
\end{theorem}

\begin{theorem}
	矩阵的初等列变换不改变矩阵的秩。
\end{theorem}

\begin{theorem}
	若向量组 $A$ 可由 $B$ 表出,且 $\operatorname{rank}(A) = \operatorname{rank}(B)$,则 $A$ 与 $B$ 等价。
\end{theorem}

\begin{proof}
	设 $A' = \vec \alpha_{i_1}, \ldots, \vec \alpha_{i_r}$ 和 $B' = \vec \beta_{j_1}, \ldots, \vec \beta_{j_r}$ 分别是 $A$ 和 $B$ 的极大线性无关组。

	可知,向量组 $C_l = \vec \alpha_{i_1}, \ldots, \vec \alpha_{i_r}, \vec \beta_{jl} \pod{l = 1, \ldots, r}$ 可被 $B'$ 线性表出,故 $C_l$ 线性相关。由于 $A'$ 线性无关,因此 $B'$ 可被 $A'$ 线性表出。即可证得 $A$ 与 $B$ 等价。证毕。
\end{proof}

注意以上证明过程中运用了许多之前的定理但没有注明。

\subsection{子式与矩阵的秩}

矩阵的秩与其最高阶非零子式有着密切关系。

\begin{theorem}
	任一非零矩阵的秩等于它的不为零的子式的最高阶数。
\end{theorem}

\begin{proof}
	设矩阵 $s \times n$ 的矩阵 $A$ 的秩为 $r$。按以下思路进行证明:
	\begin{enumerate}
		\item 存在 $r$ 阶非零子式。

		$A$ 有 $r$ 行线性无关,它们组成一个矩阵 $A_1$。由于 $\operatorname{rank}(A_1) = r$,因此 $A_1$ 有 $r$ 列线性无关,于是 $A_1$ 的这 $r$ 列形成的行列式不为 $0$,而这是 $A$ 的一个 $r$ 阶子式。

		\item 不存在 $r + k \pod{k \ge 1}$ 阶非零子式。

		设 $m > r$,且 $m \le \min\{s, n\}$。设 $A$ 的第 $j_1, \ldots, j_r$ 列是其列向量组的极大线性无关组,则 $A$ 的任意 $m$ 列都可被这 $r$ 列线性表出。由于 $m > r$,因此这 $m$ 列线性相关,则这 $m$ 列的缩短组线性相关,即 $A$ 的任意 $m$ 阶子式等于 $0$。
	\end{enumerate}
\end{proof}

% TODO: 补充第二个证明。

\begin{theorem}
	设 $s \times n$ 矩阵 $A$ 的秩为 $r$,则 $A$ 的不等于零的 $r$ 阶子式所在的列(行)构成 $A$ 的列(行)向量组的一个极大线性无关组。
\end{theorem}

\begin{proof}
	$A$ 的不等于零的 $r$ 阶子式的列(行)向量组线性无关,从而它的延伸组也线性无关,即 $A$ 的相应的 $r$ 列(行)线性无关。由于 $A$ 的秩为 $r$,因此这 $r$ 列(行)构成 $A$ 的列(行)向量组的一个极大线性无关组。
\end{proof}

显然 $s \times n$ 矩阵的秩至多为 $\min \{s, n\}$,由此可以引出满秩矩阵的定义。

\begin{definition}{满秩矩阵}
	一个 $n$ 级矩阵 $A$ 的秩如果等于它的阶数 $n$,那么称 $A$ 为\emph{满秩矩阵}。
\end{definition}

注意满秩矩阵的概念限于方阵,但一般的矩阵也可以被称为是满秩的。

对于满秩矩阵,以下定理显然成立。

\begin{theorem}
	$n$ 级矩阵 $A$ 满秩的充分必要条件是 $|A| \ne 0$。
\end{theorem}

\subsection{线性方程组有解的充分必要条件}

利用矩阵的秩,我们给出以下有关线性方程组有解的充分必要条件的结论。

\begin{theorem}
	数域 $\mathbb K$ 上的线性方程组 $x_1 \vec \alpha_1 + x_2 \vec \alpha_2 + \cdots + x_n \vec \alpha_n = \vec \beta$ 有解的充分必要条件是它的系数矩阵与增广矩阵的秩相等。
\end{theorem}

\begin{proof}
	原命题 $\Longleftrightarrow$ $\vec \beta$ 可以由 $\vec \alpha_1, \vec \alpha_2, \ldots, \vec \alpha_n$ 线性表出 $\Longleftrightarrow$ 系数矩阵列向量组与增广矩阵列向量组等价 $\Longleftrightarrow$ 系数矩阵与增广矩阵的秩相等。
\end{proof}

\begin{theorem}
	数域 $\mathbb K$ 上的 $n$ 元线性方程组有解时,如果它的系数矩阵 $A$ 的秩等于 $n$,那么原方程组有唯一解;如果它的秩小于 $n$,那么原方程组有无穷多个解。
\end{theorem}

使用高斯约当算法判断解的情况,并判断主元个数与秩的关系,即可证明。具体过程略。

针对齐次线性方程组,有以下推论。

\begin{theorem}
	数域 $\mathbb K$ 上的 $n$ 元齐次线性方程组有非零解的充分必要条件是:它的系数矩阵的秩小于未知量的个数 $n$。
\end{theorem}