% Licensed under the Creative Commons Attribution Share Alike 4.0 International.
% See the LICENCE file in the repository root for full licence text.

\section{线性相关与线性无关的向量组}

在向量空间 $\mathbb K^2$ 中,设:
$$
	\vec e_1 = (1, 0)
$$$$
	\vec e_2 = (0, 1)
$$

显然,$\langle \vec e_1, \vec e_2 \rangle = \mathbb K^2$。进一步,我们想知道 $\mathbb K^2$ 上的其他向量组的张成是什么。举两例,$\langle (1, 0), (1, 1) \rangle = \mathbb K^2$,但 $\langle (1, 1), (2, 2) \rangle \ne \mathbb K^2$。可见,这两个向量组的张成不同,与向量组中的各向量有直接关系:第一例中,两向量“不共线”;而第二例中,两向量“共线”。我们将这一思路推广,引出以下定义。

\begin{definition}{线性相关}
	设 $\mathbb K^n$ 中一向量组为 $\vec \alpha_1, \ldots, \vec \alpha_s \pod{s \ge 1}$。如果 $\mathbb K$ 中有不全为 $0$ 的数 $k_1, \ldots, k_s$,使得 $k_1 \vec \alpha_1 + \cdots + k_s \vec \alpha_s = \vec 0$,则称该向量组是\emph{线性相关}的。
\end{definition}

\begin{definition}{线性无关}
	设 $\mathbb K^n$ 中一向量组为 $\vec \alpha_1, \ldots, \vec \alpha_s \pod{s \ge 1}$。如果该向量组不是线性相关的,则称该向量组是\emph{线性无关}的。

	换言之,如果可以由 $k_1 \vec \alpha_1 + \cdots + k_s \vec \alpha_s = \vec 0$ 得出 $k_1 = k_2 = \cdots = k_s = 0$,则该向量组是线性无关的。
\end{definition}

在以上定义的基础上,我们给出以下显然的结论。

\begin{theorem}
	包含零向量的向量组一定线性相关。
\end{theorem}

\begin{proof}
	因为 $1 \cdot \vec 0 + 0 \cdot \vec \alpha_2 + \cdots + 0 \cdot \vec \alpha_s = \vec 0$。
\end{proof}

\begin{theorem}
	单个向量 $\vec \alpha$ 组成的向量组线性相关当且仅当 $\vec \alpha = \vec 0$。
\end{theorem}

\begin{proof}
	因为“存在 $k \ne 0$ 使得 $k \vec \alpha = \vec 0$”是 $\vec \alpha = \vec 0$ 的充分必要条件。
\end{proof}

\begin{theorem}
	$\mathbb K^n$ 中,以下向量组成的向量组是线性无关的:
	$$
	\begin{aligned}
		\vec e_1 &= (1, 0, 0, \ldots, 0, 0)
		\\
		\vec e_2 &= (0, 1, 0, \ldots, 0, 0)
		\\
		\vdots
		\\
		\vec e_n &= (0, 0, 0, \ldots, 0, 1)
	\end{aligned}
	$$
\end{theorem}

由线性无关的定义即可得出该结论。

\subsection{从多个角度考察线性相关与线性无关的本质区别}

定义中,用“是否存在不全为零的一组满足方程的数”区别线性相关与线性无关。事实上,还可以从以下其他角度考察线性相关与线性无关,它们都非常重要。

\begin{enumerate}
	\item 从齐次线性方程组看:

	向量组 $\vec \alpha_1, \ldots, \vec \alpha_s \pod{s \ge 1}$ 线性相关 $\Longleftrightarrow$ 他们有系数不全为 $0$ 的线性组合等于零向量。

	向量组 $\vec \alpha_1, \ldots, \vec \alpha_s \pod{s \ge 1}$ 线性无关 $\Longleftrightarrow$ 他们只有系数全为 $0$ 的线性组合才会等于零向量。

	\item 从线性表出看:

	向量组 $\vec \alpha_1, \ldots, \vec \alpha_s \pod{s \ge 2}$ 线性相关 $\Longleftrightarrow$ 其中至少有一个向量可以由其余向量线性表出。

	\begin{proof}
		充分性。设 $\vec \alpha_1, \vec \alpha_2, \ldots, \vec \alpha_s$ 线性相关,则有不全为 $0$ 的数 $k_1, k_2, \ldots, k_s$,使得:
		$$
		k_1 \vec \alpha_1 + k_2 \vec \alpha_2 + \cdots + k_s \vec \alpha_s = \vec 0
		$$

		设 $k_i \ne 0$,则由上式得:
		$$
		\vec \alpha_i = - \frac{k_1}{k_i} \vec \alpha_1 - \cdots - \frac{k_{i - 1}}{k_i} \vec \alpha_{i - 1} - \frac{k_{i + 1}}{k_i} \vec \alpha_{i + 1} - \cdots - \frac{k_s}{k_1} \vec \alpha_s
		$$

		\bigskip

		必要性。设 $\vec \alpha_j = l_1 \vec \alpha_1 + \cdots + l_{j - 1} \vec \alpha_{j - 1} + l_{j + 1} \vec \alpha_{j + 1} + \cdots + l_s \vec \alpha_s$,则:
		$$
		l_1 \vec \alpha_1 + \cdots + l_{j - 1} \vec \alpha_{j - 1} - \vec \alpha_j + l_{j + 1} \vec \alpha_{j + 1} + \cdots + l_s \vec \alpha_s = \vec 0
		$$

		从而 $\vec \alpha_1, \vec \alpha_2, \ldots, \vec \alpha_s$ 线性相关。
	\end{proof}

	向量组 $\vec \alpha_1, \ldots, \vec \alpha_s \pod{s \ge 2}$ 线性无关 $\Longleftrightarrow$ 其中每一个向量都不能由其余向量线性表出。

	\item 从齐次线性方程组看:

	列向量组 $\vec \alpha_1, \ldots, \vec \alpha_s \pod{s \ge 1}$ 线性相关 $\Longleftrightarrow$ 齐次线性方程组 $x_1 \vec \alpha_1 + \ldots + x_s \vec \alpha_s = \vec 0$ 有非零解。

	列向量组 $\vec \alpha_1, \ldots, \vec \alpha_s \pod{s \ge 1}$ 线性无关 $\Longleftrightarrow$ 齐次线性方程组 $x_1 \vec \alpha_1 + \ldots + x_s \vec \alpha_s = \vec 0$ 只有零解。

	\item 从行列式看:

	$n$ 个 $n$ 维列(行)向量 $\vec \alpha_1, \vec \alpha_2, \ldots, \vec \alpha_n$ 线性相关 $\Longleftrightarrow$ 以 $\vec \alpha_1, \vec \alpha_2, \ldots, \vec \alpha_n$ 为列(行)向量组的矩阵的行列式等于零。

	$n$ 个 $n$ 维列(行)向量 $\vec \alpha_1, \vec \alpha_2, \ldots, \vec \alpha_n$ 线性无关 $\Longleftrightarrow$ 以 $\vec \alpha_1, \vec \alpha_2, \ldots, \vec \alpha_n$ 为列(行)向量组的矩阵的行列式不等于零。

	\item 从向量组线性表出一个向量的方式看:

	设向量 $\vec \beta$ 可以由向量组 $\vec \alpha_1, \ldots, \vec \alpha_s$ 线性表出。

	向量组 $\vec \alpha_1, \ldots, \vec \alpha_s$ 线性无关 $\Longleftrightarrow$ 表出方式唯一。

	\begin{proof}
		设 $\vec \beta = b_1 \vec \alpha_1 + \cdots + b_s \vec \alpha_s$。

		\bigskip

		充分性。已知 $\vec \alpha_1, \ldots, \vec \alpha_s$ 线性无关。如果还有 $\vec \beta = c_1 \vec \alpha_1 + \cdots + c_s \vec \alpha_s$,那么:
		$$
		b_1 \vec \alpha_1 + \cdots + b_s \vec \alpha_s = c_1 \vec \alpha_1 + \cdots + c_s \vec \alpha_s
		$$

		从而:
		$$
		(b_1 - c_1) \vec \alpha_1 + \cdots + (b_s - c_s) \vec \alpha_s = \vec 0
		$$

		由于 $\vec \alpha_1, \ldots, \vec \alpha_s$ 线性无关,因而有:
		$$
		b_1 - c_1 = \cdots = b_s - c_s = 0
		$$

		即 $b_1 = c_1, \ldots, b_s = c_s$,因此 $\vec \beta$ 由 $\vec \alpha_1, \ldots, \vec \alpha_s$ 线性表出的方式唯一。

		\bigskip

		必要性。已知 $\vec \beta$ 由 $\vec \alpha_1, \ldots, \vec \alpha_s$ 线性表出的方式唯一。假如 $\vec \alpha_1, \ldots, \vec \alpha_s$ 线性相关,则有不全为 $0$ 的数 $k_1, \ldots, k_s$,使得 $k_1 \vec \alpha_1 + \cdots + k_s \vec \alpha_s = \vec 0$。则可得:$\vec \beta = (b_1 + k_1) \vec \alpha_1 + \cdots + (b_s + k_s) \vec \alpha_s$。由于 $k_1, \ldots, k_s$ 不全为 $0$,所以 $(b_1 + k_1, \ldots, b_s + k_s) \ne (b_1, \ldots, b_s)$,于是 $\vec \beta$ 由 $\vec \alpha_1, \ldots, \vec \alpha_s$ 线性表出的方式至少有两种,这与表出方式唯一矛盾,因此 $\vec \alpha_1, \ldots, \vec \alpha_s$ 线性无关。
	\end{proof}

	向量组 $\vec \alpha_1, \ldots, \vec \alpha_s$ 线性相关 $\Longleftrightarrow$ 表出方式有无穷多种。

	\item 从向量组与它的部分组的关系看:

	如果向量组的一个部分组线性相关,那么整个向量组也线性相关。

	如果向量组线性无关,那么它的任何一个部分组也线性无关。

	\item 从向量组与它的延伸组或缩短组的关系看:

	如果向量组线性无关,那么把每个向量添上 $m$ 个分量(所添分量的位置对于每个向量都一样)得到的延伸组也线性无关。(例如,向量 $(1, 2)$ 增加两个分量变为 $(1, 2, 3, 4)$)

	\begin{proof}
		设 $\vec \alpha_1, \ldots, \vec \alpha_s$ 的一个延伸组为 $\tilde \alpha_1, \ldots, \tilde \alpha_s$,则从 $k_1 \tilde \alpha_1 + \cdots + k_s \tilde \alpha_s = \vec 0$ 可得出:
		$$
		k_1 \vec \alpha_1 + \cdots + k_s \vec \alpha_s = \vec 0
		$$

		若 $\vec \alpha_1, \ldots, \vec \alpha_s$ 线性无关,则从上式得 $k_1 = \cdots = k_s = 0$,从而 $\tilde \alpha_1, \ldots, \tilde \alpha_s$ 也线性无关。
	\end{proof}

	如果向量组线性相关,那么把每个向量去掉 $m$ 个分量(去掉的分量的位置对于每个向量都一样)得到的缩短组也线性相关。注意这是上述命题的逆否命题。
\end{enumerate}

我们要研究一个向量是否能由一个向量组线性表出,首先需要研究向量组线性无关的情形,因此给出以下定理。

\begin{theorem}
	设向量组 $\vec \alpha_1, \ldots, \vec \alpha_s$ 线性无关,则向量 $\vec \beta$ 可以由 $\vec \alpha_1, \ldots, \vec \alpha_s$ 线性表出的充分必要条件是 $\vec \alpha_1, \vec \alpha_2, \ldots, \vec \alpha_s, \vec \beta$ 线性相关。
\end{theorem}

\begin{proof}
	必要性显然,下面证充分性。设 $\vec \alpha_1, \vec \alpha_2, \ldots, \vec \alpha_s, \vec \beta$ 线性相关,则有 $\mathbb K$ 中不全为 $0$ 的数 $k_1, k_2, \ldots, k_s, l$,使得:
	$$
	k_1 \vec \alpha_1 + k_2 \vec \alpha_2 + \cdots + k_s \vec \alpha_s + l \vec \beta = \vec 0
	$$

	假设 $l = 0$,则 $k_1, k_2, \ldots, k_s$ 不全为 $0$,并且从上式可得:
	$$
	k_1 \vec \alpha_1 + k_2 \vec \alpha_2 + \cdots + k_s \vec \alpha_s = \vec 0
	$$

	与向量组线性无关矛盾,因此 $l$ 始终不为 $0$,从而:
	$$
	\vec \beta = -\dfrac{k_1}{l} \vec \alpha_1 - \dfrac{k_2}{l} \vec \alpha_2 - \cdots - \dfrac{k_s}{l} \vec \alpha_s
	$$

	即 $\vec \beta$ 可由 $\vec \alpha_1, \ldots, \vec \alpha_s$ 线性表出。证毕。
\end{proof}

下面的定理是上述定理的逆否命题。

\begin{theorem}
	设向量组 $\vec \alpha_1, \ldots, \vec \alpha_s$ 线性无关,则向量 $\vec \beta$ 不能由 $\vec \alpha_1, \ldots, \vec \alpha_s$ 线性表出的充分必要条件是 $\vec \alpha_1, \vec \alpha_2, \ldots, \vec \alpha_s, \vec \beta$ 线性无关。
\end{theorem}